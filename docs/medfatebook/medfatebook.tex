\documentclass[]{book}
\usepackage{lmodern}
\usepackage{amssymb,amsmath}
\usepackage{ifxetex,ifluatex}
\usepackage{fixltx2e} % provides \textsubscript
\ifnum 0\ifxetex 1\fi\ifluatex 1\fi=0 % if pdftex
  \usepackage[T1]{fontenc}
  \usepackage[utf8]{inputenc}
\else % if luatex or xelatex
  \ifxetex
    \usepackage{mathspec}
  \else
    \usepackage{fontspec}
  \fi
  \defaultfontfeatures{Ligatures=TeX,Scale=MatchLowercase}
\fi
% use upquote if available, for straight quotes in verbatim environments
\IfFileExists{upquote.sty}{\usepackage{upquote}}{}
% use microtype if available
\IfFileExists{microtype.sty}{%
\usepackage{microtype}
\UseMicrotypeSet[protrusion]{basicmath} % disable protrusion for tt fonts
}{}
\usepackage{hyperref}
\hypersetup{unicode=true,
            pdftitle={The medfate reference book},
            pdfborder={0 0 0},
            breaklinks=true}
\urlstyle{same}  % don't use monospace font for urls
\usepackage{natbib}
\bibliographystyle{apalike}
\usepackage{color}
\usepackage{fancyvrb}
\newcommand{\VerbBar}{|}
\newcommand{\VERB}{\Verb[commandchars=\\\{\}]}
\DefineVerbatimEnvironment{Highlighting}{Verbatim}{commandchars=\\\{\}}
% Add ',fontsize=\small' for more characters per line
\usepackage{framed}
\definecolor{shadecolor}{RGB}{248,248,248}
\newenvironment{Shaded}{\begin{snugshade}}{\end{snugshade}}
\newcommand{\KeywordTok}[1]{\textcolor[rgb]{0.13,0.29,0.53}{\textbf{#1}}}
\newcommand{\DataTypeTok}[1]{\textcolor[rgb]{0.13,0.29,0.53}{#1}}
\newcommand{\DecValTok}[1]{\textcolor[rgb]{0.00,0.00,0.81}{#1}}
\newcommand{\BaseNTok}[1]{\textcolor[rgb]{0.00,0.00,0.81}{#1}}
\newcommand{\FloatTok}[1]{\textcolor[rgb]{0.00,0.00,0.81}{#1}}
\newcommand{\ConstantTok}[1]{\textcolor[rgb]{0.00,0.00,0.00}{#1}}
\newcommand{\CharTok}[1]{\textcolor[rgb]{0.31,0.60,0.02}{#1}}
\newcommand{\SpecialCharTok}[1]{\textcolor[rgb]{0.00,0.00,0.00}{#1}}
\newcommand{\StringTok}[1]{\textcolor[rgb]{0.31,0.60,0.02}{#1}}
\newcommand{\VerbatimStringTok}[1]{\textcolor[rgb]{0.31,0.60,0.02}{#1}}
\newcommand{\SpecialStringTok}[1]{\textcolor[rgb]{0.31,0.60,0.02}{#1}}
\newcommand{\ImportTok}[1]{#1}
\newcommand{\CommentTok}[1]{\textcolor[rgb]{0.56,0.35,0.01}{\textit{#1}}}
\newcommand{\DocumentationTok}[1]{\textcolor[rgb]{0.56,0.35,0.01}{\textbf{\textit{#1}}}}
\newcommand{\AnnotationTok}[1]{\textcolor[rgb]{0.56,0.35,0.01}{\textbf{\textit{#1}}}}
\newcommand{\CommentVarTok}[1]{\textcolor[rgb]{0.56,0.35,0.01}{\textbf{\textit{#1}}}}
\newcommand{\OtherTok}[1]{\textcolor[rgb]{0.56,0.35,0.01}{#1}}
\newcommand{\FunctionTok}[1]{\textcolor[rgb]{0.00,0.00,0.00}{#1}}
\newcommand{\VariableTok}[1]{\textcolor[rgb]{0.00,0.00,0.00}{#1}}
\newcommand{\ControlFlowTok}[1]{\textcolor[rgb]{0.13,0.29,0.53}{\textbf{#1}}}
\newcommand{\OperatorTok}[1]{\textcolor[rgb]{0.81,0.36,0.00}{\textbf{#1}}}
\newcommand{\BuiltInTok}[1]{#1}
\newcommand{\ExtensionTok}[1]{#1}
\newcommand{\PreprocessorTok}[1]{\textcolor[rgb]{0.56,0.35,0.01}{\textit{#1}}}
\newcommand{\AttributeTok}[1]{\textcolor[rgb]{0.77,0.63,0.00}{#1}}
\newcommand{\RegionMarkerTok}[1]{#1}
\newcommand{\InformationTok}[1]{\textcolor[rgb]{0.56,0.35,0.01}{\textbf{\textit{#1}}}}
\newcommand{\WarningTok}[1]{\textcolor[rgb]{0.56,0.35,0.01}{\textbf{\textit{#1}}}}
\newcommand{\AlertTok}[1]{\textcolor[rgb]{0.94,0.16,0.16}{#1}}
\newcommand{\ErrorTok}[1]{\textcolor[rgb]{0.64,0.00,0.00}{\textbf{#1}}}
\newcommand{\NormalTok}[1]{#1}
\usepackage{longtable,booktabs}
\usepackage{graphicx,grffile}
\makeatletter
\def\maxwidth{\ifdim\Gin@nat@width>\linewidth\linewidth\else\Gin@nat@width\fi}
\def\maxheight{\ifdim\Gin@nat@height>\textheight\textheight\else\Gin@nat@height\fi}
\makeatother
% Scale images if necessary, so that they will not overflow the page
% margins by default, and it is still possible to overwrite the defaults
% using explicit options in \includegraphics[width, height, ...]{}
\setkeys{Gin}{width=\maxwidth,height=\maxheight,keepaspectratio}
\IfFileExists{parskip.sty}{%
\usepackage{parskip}
}{% else
\setlength{\parindent}{0pt}
\setlength{\parskip}{6pt plus 2pt minus 1pt}
}
\setlength{\emergencystretch}{3em}  % prevent overfull lines
\providecommand{\tightlist}{%
  \setlength{\itemsep}{0pt}\setlength{\parskip}{0pt}}
\setcounter{secnumdepth}{5}
% Redefines (sub)paragraphs to behave more like sections
\ifx\paragraph\undefined\else
\let\oldparagraph\paragraph
\renewcommand{\paragraph}[1]{\oldparagraph{#1}\mbox{}}
\fi
\ifx\subparagraph\undefined\else
\let\oldsubparagraph\subparagraph
\renewcommand{\subparagraph}[1]{\oldsubparagraph{#1}\mbox{}}
\fi

%%% Use protect on footnotes to avoid problems with footnotes in titles
\let\rmarkdownfootnote\footnote%
\def\footnote{\protect\rmarkdownfootnote}

%%% Change title format to be more compact
\usepackage{titling}

% Create subtitle command for use in maketitle
\providecommand{\subtitle}[1]{
  \posttitle{
    \begin{center}\large#1\end{center}
    }
}

\setlength{\droptitle}{-2em}

  \title{The medfate reference book}
    \pretitle{\vspace{\droptitle}\centering\huge}
  \posttitle{\par}
    \author{true}
    \preauthor{\centering\large\emph}
  \postauthor{\par}
      \predate{\centering\large\emph}
  \postdate{\par}
    \date{2019-05-30}

\usepackage{booktabs}
\usepackage{amsthm}
\makeatletter
\def\thm@space@setup{%
  \thm@preskip=8pt plus 2pt minus 4pt
  \thm@postskip=\thm@preskip
}
\makeatother

\begin{document}
\maketitle

{
\setcounter{tocdepth}{1}
\tableofcontents
}
\chapter*{Preface}\label{preface}
\addcontentsline{toc}{chapter}{Preface}

This is a reference book for the models implemented in the R package
\textbf{medfate}. This package provides functions for the simulation of
functioning and dynamics of Mediterranean forests.

\begin{center}\includegraphics[width=0.2\linewidth]{LOGO_Group} \end{center}

The online version of this book is licensed under the Creative Commons
Attribution-NonCommercial-ShareAlike 4.0 International License.

\part{Preliminaries}\label{part-preliminaries}

\chapter{Introduction}\label{intro}

\section{Purpose}\label{purpose}

Being able to anticipate the impact of global change on ecosystems is
one of the major environmental challenges in contemporary societies.
However, uncertainties in how ecological systems function and practical
constraints in how to integrate available information prevent the
development of robust and reliable predictive models. Despite the amount
of knowledge accumulated about the functioning and dynamics of
Mediterranean forests, scientists should make coordinate their efforts
to address the challenge of integrating the different global change
drivers in a modelling framework useful for research and applications.

The R package \textbf{medfate} has been designed to simulate the
functioning and dynamics of Mediterranean forest stands at temporal
scales from days to years. Fire and drought impacts are the main
subjects covered by the package. Eco-hydrological processes are
fundamental for the simulation models included in the package. In
particular, the package allows the simulation of water balance of soils
and plants within forest stands. It also allows simulating plant growth
of a set of cohorts competing for light and water within a forest stand.
Finally, the package also includes functions to relate the amount of
plant biomass and the relative water content of plant tissues to live
fuel moisture content, hence, fire hazard.

The version of the reference manual that you are reading is intended to
reflect \textbf{version 0.8.3} of the package.

\section{Package installation}\label{package-installation}

Package \textbf{medfate} can be found at
\href{https://CRAN.R-project.org/package=medfate}{CRAN}, where it is
updated every few months. Hence, it can be installed using:

\begin{Shaded}
\begin{Highlighting}[]
\KeywordTok{install.packages}\NormalTok{(}\StringTok{"medfate"}\NormalTok{)}
\end{Highlighting}
\end{Shaded}

Users can also download and install the latest stable versions GitHub as
follows (required package \texttt{devtools} should be installed/updated
first):

\begin{Shaded}
\begin{Highlighting}[]
\NormalTok{devtools}\OperatorTok{::}\KeywordTok{install_github}\NormalTok{(}\StringTok{"miquelcaceres/medfate"}\NormalTok{)}
\end{Highlighting}
\end{Shaded}

When installing from GitHub, may need to force the installation of
package vignettes, by using:

\begin{Shaded}
\begin{Highlighting}[]
\NormalTok{devtools}\OperatorTok{::}\KeywordTok{install_github}\NormalTok{(}\StringTok{"miquelcaceres/medfate"}\NormalTok{, }
                         \DataTypeTok{build_opts =} \KeywordTok{c}\NormalTok{(}\StringTok{"--no-resave-data"}\NormalTok{, }\StringTok{"--no-manual"}\NormalTok{))}
\end{Highlighting}
\end{Shaded}

\section{Package functions}\label{package-functions}

\subsection{Dynamic simulation
functions}\label{dynamic-simulation-functions}

Soil water balance can be studied for a given forest stand using
function \texttt{spwb()}, with the following purposes:

\begin{enumerate}
\def\labelenumi{\arabic{enumi}.}
\tightlist
\item
  Monitor or forecast temporal variation in soil water content in
  particular stands (for example to estimate mushroom yield).
\item
  Monitor or forecast temporal variation of plant drought stress in
  particular stands (for example to anticipate mortality events).
\item
  Monitor or forecast temporal variation of fuel moisture in particular
  stands (for example to monitor wildfire risk).
\end{enumerate}

Changes in leaf area and plant growth are key to evaluate the influence
of climatic conditions on forest structure and function. Function
\texttt{growth()} allows simulating growth of a set of plant cohorts
competing for light and water in a given forest stand, with the
following purposes:

\begin{enumerate}
\def\labelenumi{\arabic{enumi}.}
\tightlist
\item
  Monitor or forecast temporal variation in water fluxes and soil water
  content in forest stands (for example to estimate regulation ecosystem
  services) taking into account processes determining leaf area changes
  and plant growth.
\item
  Monitor or forecast temporal variation of plant size (i.e.~growth) in
  forest stands.
\item
  Monitor or forecast temporal variation of live fuel moisture and the
  amount of standing dead and live fuels in forest stands (for example
  to monitor wildfire risk).
\end{enumerate}

\subsection{Sub-model functions}\label{sub-model-functions}

Many of the functions included in \textbf{medfate} are internally called
by simulation functions. Some of them are made available to the user, to
facilitate understanding the different sub-models and to facilitate a
more creative use of the package. Sub-model functions are grouped by
\emph{subject}, which is included in the name of the function. The
different sub-model functions are (by subject):

\begin{itemize}
\tightlist
\item
  \texttt{biophysics\_*}: Physical and biophysical utility functions.
\item
  \texttt{hydraulics\_*}: Plant hydraulics.
\item
  \texttt{hydrology\_*}: Canopy and soil hydrology (rainfall
  interception, soil evaporation, soil infiltration).
\item
  \texttt{light\_*}: Light extinction and absortion.
\item
  \texttt{moisture\_*}: Live tissue moisture.
\item
  \texttt{pheno\_*}: Leaf phenology.
\item
  \texttt{photo\_*}: Leaf photosynthesis.
\item
  \texttt{root\_*}: Root distribution and conductance calculations.
\item
  \texttt{soil\_*}: Soil hydraulics and thermodynamics.
\item
  \texttt{spwb\_*}: Soil water balance parameter
  optimization/calibration routines.
\item
  \texttt{transp\_*}: Stomatal regulation and resulting
  transpiration/photosynthesis.
\end{itemize}

\subsection{Static functions}\label{static-functions}

Package \textbf{medfate} include a number of functions to examine
properties of the plants conforming forests, summary functions at the
stand level or vertical profiles of several physical properties:

\begin{itemize}
\tightlist
\item
  \texttt{plant\_*}: Cohort-level information (species name, id, leaf
  area, height\ldots{}).
\item
  \texttt{species\_*}: Cohort-level attributes aggregated by species
  (e.g.~basal area).
\item
  \texttt{forest\_*}: Forest-level attributes (e.g.~basal area).
\item
  \texttt{vprofile\_*}: Vertical profiles (light, wind, fuel density,
  leaf area density).
\end{itemize}

Vegetation functioning and dynamics have strong, but complex, effects on
fire hazard. On one hand, growth and death of organs and individuals
changes the amount of standing live and dead fuels, as well as downed
dead fuels. On the other, day-to-day changes in soil and plant water
content changes the physical properties of fuel, notably fuel moisture
content. Package \textbf{medfate} provides functions to estimate fuel
properties and potential fire behaviour in forest inventory plots.
Specifically, function \texttt{fuel\_stratification()} provides a
stratification of the stand into understory and canopy strata; and
\texttt{fuel\_FCCS()} calculates fuel characteristics. Function
\texttt{fuel\_cohortFineFMC()} allows obtaining daily fuel moisture
content estimates corresponding to the water status of plants, as
returned by function \texttt{spwb()}. A fire behaviour model is
implemented in function \texttt{fire\_FCCS()} to calculate the intensity
of surface fire reaction and the rate of fire spread of surface fires
assuming a steady-state fire. Fuel and fire behaviour functions allow
obtaining the following:

\begin{enumerate}
\def\labelenumi{\arabic{enumi}.}
\tightlist
\item
  Fuel characteristics by stratum.
\item
  Surface fire behavior (i.e.~reaction intensity, rate of spread,
  fireline intensity and flame length).
\item
  Crown fire behavior.
\item
  Fire potential ratings of surface fire behavior and crown fire
  behavior.
\end{enumerate}

\section{How to use this book}\label{how-to-use-this-book}

This is a \emph{reference book} for simulation and static models
included in \textbf{medfate}. Hands-on user guides to run simulation
functions and static functions can be found as \emph{package vignettes}
within the package. Our aim is to continuously update the reference book
along with package developments, so that users have detailed information
about the models at the time functions are run. As the manual will
follow package updates, after a given model application users should
store a \textbf{PDF version} of the reference manual to be sure it
matches the version the package reported in their application report or
article.

The following chapter describes the soil, vegetation and meteorology
inputs that apply to all models. After that, the book presents each
simulation model using a set of chapters. The first chapter introduces
the model and the remaining describe the details of formulations by
processes. Those process formulations that are common to more than one
model (like hydrology) are described only once. Static models and their
functions are presented grouped at the end of the book, before the
appendices.

In this book we use \texttt{code} or \texttt{variablename} to indicate R
code or to refer to variable names within R, and \texttt{functionname()}
to refer to a package function. When relevant, we will indicate the
correspondence between mathematical symbols, their units and variable
names within R.

\chapter{Model inputs}\label{model-inputs}

Data input for process-based models of forest functioning and dynamics
can be described using three broad categories. First, a description of
the target \textbf{vegetation} is needed, at a level of detail
appropriate for the model design. This includes not only species
composition and/or structure, but also functional parameters of the
plants. Second, most often \textbf{soil} information is required,
specially if underground resources (i.e.~water and/or nutrients) are
important for the processes included in the model. While soil and
vegetation features are sometimes considered static parameters in model
simulations, \textbf{meteorology} is most often a dynamic input which
defines the temporal variation of environmental conditions in which
biological processes occur. This chapter describes the requirements of
\textbf{medfate} package with respect to each of these three kinds of
input. The last section describes additional parameters that are used to
control the behaviour of simulation functions.

\section{Soil description}\label{soil-description}

Soils can be described in \textbf{medfate} using between 1 and 5 soil
layers. The number and size of layers may reflect changes in soil
properties, but also can be chosen to reflect different plant rooting
depths. The physical properties of the soil are needed to estimate its
hydraulic properties. Most soil properties are considered static
parameters in simulations, except those that depend on soil water
balance (e.g.~soil moisture content, soil water potential and soil
conductivity) and soil temperature, if the soil energy balance is
considered in simulations.

\subsection{Physical properties}\label{physical-properties}

Physical soil properties, like soil texture (i.e.~percent of sand, silt
and clay), bulk density or rock fragment content, can differ between
soil layers, and this has important consequences for water retention
capacity and soil hydraulics. In addition, specifying a deep rocky layer
may be important in some biomes like the Mediterranean one, because
plants often extend their roots into cracks existing in the parent rock
\citep{Ruffault2013}. For each soil layer \(s\) the following physical
parameters are needed:

\begin{longtable}[]{@{}llll@{}}
\toprule
Symbol & Units & R & Description\tabularnewline
\midrule
\endhead
\(d_{s}\) & \(mm\) & \texttt{widths} & Soil layer width\tabularnewline
\(P_{clay,s}\) & \% & \texttt{clay} & Percent of clay (within soil
particles)\tabularnewline
\(P_{sand,s}\) & \% & \texttt{sand} & Percent of sand (within soil
particles)\tabularnewline
\(OM\) & \% & \texttt{om} & Percentage of organic mater per dry weight
(can be missing)\tabularnewline
\(BD_{s}\) & \(g \cdot cm^{-3}\) & \texttt{bd} & Bulk
density\tabularnewline
\(P_{rocks,s}\) & \% & \texttt{rfc} & Rock fragment content as percent
per soil volume\tabularnewline
\bottomrule
\end{longtable}

\emph{Soil depth} is defined as the sum of soil layer widths. If
possible, soil physical properties should be measured in soil profiles
conducted at the target forest plot. Soil input data should be arranged
in a \texttt{data.frame} with soil layers in rows and physical variables
in columns (see function \texttt{defaultSoilParams()}). The package
includes function \texttt{soilgridsParams()} to fetch soil information
from \href{https://soilgrids.org/}{SoilGrids.org}, a global soil
database currently providing soil data at 250m scale. This can be
helpful to users lacking local soil measurements, but the uncertainty of
SoilGrids estimates can be high for some areas and soil properties,
especially soil depth and rock fragment content.

\subsection{Water retention curves}\label{water-retention-curves}

The water retention curve of a soil is the relationship between
volumetric soil moisture (\(\theta\), in \(m^3 \cdot m^{-3}\) of soil,
excluding rock fragments) and the soil water potential (\(\Psi\), in
MPa). This curve is characteristic for different types of soil, and is
also called the \emph{soil moisture characteristic curve}. Two water
retention curve models are available in \textbf{medfate}:

\begin{enumerate}
\def\labelenumi{\arabic{enumi}.}
\tightlist
\item
  \emph{Saxton model}: In this model, volumetric soil moisture
  \(\theta(\Psi)\) corresponding to a given water potential \(\Psi\) (in
  MPa) below \(\Psi_{fc}\) is calculated using:

  \begin{equation}\theta(\Psi) = (\Psi/A)^{(1/B)}\end{equation}

  where \(A\) and \(B\) depend on the texture and, if available, organic
  matter in the soil. If organic matter is available, \(A\) and \(B\)
  are calculated from \(P_{clay}\), \(P_{sand}\) and \(OM\) following
  \citet{Saxton2006}. Otherwise, they are calculated from \(P_{clay}\)
  and \(P_{sand}\) as indicated in \citet{Saxton1986}. Volumetric
  changes between field capacity and saturation are estimated using a
  linear function.
\item
  \emph{Van Genuchten model}: The well known van Genuchten
  \citeyearpar{Genuchten1980} model is:

  \begin{equation}\theta(\Psi) = \theta_{res}+\frac{\theta_{sat}-\theta_{res}}{\left[1+ (\alpha \cdot \Psi)^n \right]^{1-1/n}}\end{equation}

  where \(\theta(\psi)\) is the water retention, \(\theta_{sat}\) is the
  saturated water content, \(\theta_{res}\) is the residual water
  content, \(\alpha\) is related to the inverse of the air entry
  pressure (and here has to be expressed in \(MPa^{-1}\)), and \(n\) is
  a measure of the pore-size distribution.
\end{enumerate}

\subsection{Soil initialization}\label{soil-initialization}

Soil initialization is needed to estimate soil hydrological parameters
for each soil layer \(s\) from their physical attributes. Initialization
is done using function \texttt{soil()}, which adds the following
information to the physical soil description:

\begin{longtable}[]{@{}llll@{}}
\toprule
\begin{minipage}[b]{0.11\columnwidth}\raggedright\strut
Symbol\strut
\end{minipage} & \begin{minipage}[b]{0.10\columnwidth}\raggedright\strut
Units\strut
\end{minipage} & \begin{minipage}[b]{0.06\columnwidth}\raggedright\strut
R\strut
\end{minipage} & \begin{minipage}[b]{0.53\columnwidth}\raggedright\strut
Description\strut
\end{minipage}\tabularnewline
\midrule
\endhead
\begin{minipage}[t]{0.11\columnwidth}\raggedright\strut
\(P_{macro, s}\)\strut
\end{minipage} & \begin{minipage}[t]{0.10\columnwidth}\raggedright\strut
\%\strut
\end{minipage} & \begin{minipage}[t]{0.06\columnwidth}\raggedright\strut
\texttt{macro}\strut
\end{minipage} & \begin{minipage}[t]{0.53\columnwidth}\raggedright\strut
Percentage of macroporosity corresponding to each soil layers\strut
\end{minipage}\tabularnewline
\begin{minipage}[t]{0.11\columnwidth}\raggedright\strut
\(\gamma_{soil}\)\strut
\end{minipage} & \begin{minipage}[t]{0.10\columnwidth}\raggedright\strut
\(mm \cdot day^{-1}\)\strut
\end{minipage} & \begin{minipage}[t]{0.06\columnwidth}\raggedright\strut
\texttt{Gsoil}\strut
\end{minipage} & \begin{minipage}[t]{0.53\columnwidth}\raggedright\strut
The maximum daily evaporation rate from soil\strut
\end{minipage}\tabularnewline
\begin{minipage}[t]{0.11\columnwidth}\raggedright\strut
\(\kappa_{soil}\)\strut
\end{minipage} & \begin{minipage}[t]{0.10\columnwidth}\raggedright\strut
\strut
\end{minipage} & \begin{minipage}[t]{0.06\columnwidth}\raggedright\strut
\texttt{Ksoil}\strut
\end{minipage} & \begin{minipage}[t]{0.53\columnwidth}\raggedright\strut
Extinction parameter to regulate the amount of water extracted from each
soil layer when simulating evaporation from bare soil\strut
\end{minipage}\tabularnewline
\begin{minipage}[t]{0.11\columnwidth}\raggedright\strut
\(\theta_{sat, s}\)\strut
\end{minipage} & \begin{minipage}[t]{0.10\columnwidth}\raggedright\strut
\(m^3 \cdot m^{-3}\)\strut
\end{minipage} & \begin{minipage}[t]{0.06\columnwidth}\raggedright\strut
\texttt{VG\_theta\_sat}\strut
\end{minipage} & \begin{minipage}[t]{0.53\columnwidth}\raggedright\strut
Volumetric moisture at saturation\strut
\end{minipage}\tabularnewline
\begin{minipage}[t]{0.11\columnwidth}\raggedright\strut
\(\theta_{res, s}\)\strut
\end{minipage} & \begin{minipage}[t]{0.10\columnwidth}\raggedright\strut
\(m^3 \cdot m^{-3}\)\strut
\end{minipage} & \begin{minipage}[t]{0.06\columnwidth}\raggedright\strut
\texttt{VG\_theta\_res}\strut
\end{minipage} & \begin{minipage}[t]{0.53\columnwidth}\raggedright\strut
Residual volumetric moisture\strut
\end{minipage}\tabularnewline
\begin{minipage}[t]{0.11\columnwidth}\raggedright\strut
\(n_s\)\strut
\end{minipage} & \begin{minipage}[t]{0.10\columnwidth}\raggedright\strut
\strut
\end{minipage} & \begin{minipage}[t]{0.06\columnwidth}\raggedright\strut
\texttt{VG\_n}\strut
\end{minipage} & \begin{minipage}[t]{0.53\columnwidth}\raggedright\strut
Parameter of the Van Genuchten \citeyearpar{Genuchten1980} model\strut
\end{minipage}\tabularnewline
\begin{minipage}[t]{0.11\columnwidth}\raggedright\strut
\(\alpha_s\)\strut
\end{minipage} & \begin{minipage}[t]{0.10\columnwidth}\raggedright\strut
\strut
\end{minipage} & \begin{minipage}[t]{0.06\columnwidth}\raggedright\strut
\texttt{VG\_alpha}\strut
\end{minipage} & \begin{minipage}[t]{0.53\columnwidth}\raggedright\strut
Parameter of the Van Genuchten \citeyearpar{Genuchten1980} model\strut
\end{minipage}\tabularnewline
\bottomrule
\end{longtable}

Macroporosity values are calculated using the equations given in
\citet{Stolf2011}. Parameters of the van Genuchten model are estimated
from the physical description of the soil using one of two pedotransfer
functions (see help for \texttt{soil()}):

\begin{enumerate}
\def\labelenumi{\arabic{enumi}.}
\tightlist
\item
  Using the USDA texture classification and the average texture class
  parameters given by \citet{Carsel1988}.
\item
  Directly from the soil texture, organic matter and bulk density, using
  the pedotransfer functions in \citet{Toth2015}.
\end{enumerate}

Users can also edit the \texttt{soil} object manually, for example to
provide specific parameters of the Van Genuchten retention curve
calibrated from soil samples.

\subsection{Water content and water table
depth}\label{water-content-and-water-table-depth}

The \emph{water content} (\(V_s\) in mm) of a soil layer \(s\) is
calculated from its water potential and water retention curve using:

\begin{equation}
V_s(\Psi) = d_s\cdot ((100-P_{rocks,s})/100)\cdot\theta_{s}(\Psi)
\end{equation}

where \(d_s\) is the depth of the soil layer (in mm) and \(P_{rocks,s}\)
is the percentage of rock fragments. Besides this general formula, a
number of contents are important to remember.

\begin{itemize}
\tightlist
\item
  \emph{Water holding capacity} (\(V_{fc, s}\), in mm) of the soil layer
  \(s\) is defined as the volumetric water content at field capacity,
  i.e. -0.033 MPa:

  \begin{equation}
  V_{fc,s} = V_{s}(-0.033) = d_s\cdot ((100-P_{rocks,s})/100)\cdot\theta_{fc,s}
  \end{equation}

  where \(\theta_{fc,s} = \theta_s(-0.033)\).
\item
  \emph{Water content at saturation} is calculated anagolously, but
  replacing \(\theta_{fc,s}\) by \(\theta_{sat, s} = \theta_s(0)\).
\item
  \emph{Water content }at wilting point* (-1.5 MPa) is calculated
  replacing \(\theta_{fc,s}\) by \(\theta_{wp,s} = \theta_s(-1.5)\).
\item
  The \emph{amount of extractable water} is the difference between water
  content at field capacity and a the water content at a conventional
  minimum water potential, which can be set at wilting point or lower
  values (by default at -5 MPa in \textbf{medfate}).
\end{itemize}

The depth of water table is the depth of saturated soil. In medfate,
\emph{water table depth} (\(WT\), in mm) equals soil depth when all soil
layers are below field capacity. When some layers are between field
capacity and saturation, water table depth is calculated as:

\begin{equation}
WT = \sum_{s}{d_s \cdot \min\left[1,(\theta_{sat,s} - \theta(\Psi_s))/(\theta_{sat,s}-\theta_{fc,s})\right]}
\end{equation}

\section{Vegetation description}\label{vegetation-description}

Representation of vegetation is not spatially-explicit in
\textbf{medfate} (i.e.~trees or shrubs do not have explicit coordinates
within forest stands). However, differences in tree or shrub size
(i.e.~height) are explicitly taken into account. The taxonomic identity
of plants is also considered, and parameter values need to be provided
for each taxonomic entity (but the package could be used with functional
groups). This representation is chosen so that package functions can be
easily applied to forest plot data, for example from national forest
inventories.

Vegetation in the stand is described using a set of plant cohorts, in an
object of class \texttt{spwbInput}. Each plant cohort \(i\) is primarly
defined by its species identity (\(SP_i\); with R name
{[}\texttt{SP}{]}).

\subsection{Aboveground parameters}\label{aboveground-parameters}

The aboveground structure if each cohort is defined using a set
attributes defined in the following table, where columns \textbf{spwb}
and \textbf{growth} indicate whether attributes are required by
functions \texttt{spwb()} and \texttt{growth()}, respectively:

\begin{longtable}[]{@{}llllll@{}}
\toprule
\begin{minipage}[b]{0.10\columnwidth}\raggedright\strut
Symbol\strut
\end{minipage} & \begin{minipage}[b]{0.09\columnwidth}\raggedright\strut
Units\strut
\end{minipage} & \begin{minipage}[b]{0.06\columnwidth}\raggedright\strut
R\strut
\end{minipage} & \begin{minipage}[b]{0.43\columnwidth}\raggedright\strut
Description\strut
\end{minipage} & \begin{minipage}[b]{0.07\columnwidth}\raggedright\strut
spwb\strut
\end{minipage} & \begin{minipage}[b]{0.07\columnwidth}\raggedright\strut
growth\strut
\end{minipage}\tabularnewline
\midrule
\endhead
\begin{minipage}[t]{0.10\columnwidth}\raggedright\strut
\(H_i\)\strut
\end{minipage} & \begin{minipage}[t]{0.09\columnwidth}\raggedright\strut
\(cm\)\strut
\end{minipage} & \begin{minipage}[t]{0.06\columnwidth}\raggedright\strut
\texttt{H}\strut
\end{minipage} & \begin{minipage}[t]{0.43\columnwidth}\raggedright\strut
Total tree or shrub height\strut
\end{minipage} & \begin{minipage}[t]{0.07\columnwidth}\raggedright\strut
Y\strut
\end{minipage} & \begin{minipage}[t]{0.07\columnwidth}\raggedright\strut
Y\strut
\end{minipage}\tabularnewline
\begin{minipage}[t]{0.10\columnwidth}\raggedright\strut
\(CR_i\)\strut
\end{minipage} & \begin{minipage}[t]{0.09\columnwidth}\raggedright\strut
\strut
\end{minipage} & \begin{minipage}[t]{0.06\columnwidth}\raggedright\strut
\texttt{CR}\strut
\end{minipage} & \begin{minipage}[t]{0.43\columnwidth}\raggedright\strut
Crown ratio (i.e.~ratio between crown length and total height)\strut
\end{minipage} & \begin{minipage}[t]{0.07\columnwidth}\raggedright\strut
Y\strut
\end{minipage} & \begin{minipage}[t]{0.07\columnwidth}\raggedright\strut
Y\strut
\end{minipage}\tabularnewline
\begin{minipage}[t]{0.10\columnwidth}\raggedright\strut
\(N\)\strut
\end{minipage} & \begin{minipage}[t]{0.09\columnwidth}\raggedright\strut
\(ind · ha^{-1}\)\strut
\end{minipage} & \begin{minipage}[t]{0.06\columnwidth}\raggedright\strut
\texttt{N}\strut
\end{minipage} & \begin{minipage}[t]{0.43\columnwidth}\raggedright\strut
The density of tree individuals\strut
\end{minipage} & \begin{minipage}[t]{0.07\columnwidth}\raggedright\strut
N\strut
\end{minipage} & \begin{minipage}[t]{0.07\columnwidth}\raggedright\strut
Y\strut
\end{minipage}\tabularnewline
\begin{minipage}[t]{0.10\columnwidth}\raggedright\strut
\(DBH\)\strut
\end{minipage} & \begin{minipage}[t]{0.09\columnwidth}\raggedright\strut
\(cm\)\strut
\end{minipage} & \begin{minipage}[t]{0.06\columnwidth}\raggedright\strut
\texttt{DBH}\strut
\end{minipage} & \begin{minipage}[t]{0.43\columnwidth}\raggedright\strut
Tree diameter at breast height\strut
\end{minipage} & \begin{minipage}[t]{0.07\columnwidth}\raggedright\strut
N\strut
\end{minipage} & \begin{minipage}[t]{0.07\columnwidth}\raggedright\strut
Y\strut
\end{minipage}\tabularnewline
\begin{minipage}[t]{0.10\columnwidth}\raggedright\strut
\(Cover\)\strut
\end{minipage} & \begin{minipage}[t]{0.09\columnwidth}\raggedright\strut
\%\strut
\end{minipage} & \begin{minipage}[t]{0.06\columnwidth}\raggedright\strut
\texttt{Cover}\strut
\end{minipage} & \begin{minipage}[t]{0.43\columnwidth}\raggedright\strut
Shrub percent cover\strut
\end{minipage} & \begin{minipage}[t]{0.07\columnwidth}\raggedright\strut
N\strut
\end{minipage} & \begin{minipage}[t]{0.07\columnwidth}\raggedright\strut
Y\strut
\end{minipage}\tabularnewline
\begin{minipage}[t]{0.10\columnwidth}\raggedright\strut
\(LAI^{live}_i\)\strut
\end{minipage} & \begin{minipage}[t]{0.09\columnwidth}\raggedright\strut
\(m^2 \cdot m^{-2}\)\strut
\end{minipage} & \begin{minipage}[t]{0.06\columnwidth}\raggedright\strut
\texttt{LAI\_live}\strut
\end{minipage} & \begin{minipage}[t]{0.43\columnwidth}\raggedright\strut
(Maximum) leaf area index\strut
\end{minipage} & \begin{minipage}[t]{0.07\columnwidth}\raggedright\strut
Y\strut
\end{minipage} & \begin{minipage}[t]{0.07\columnwidth}\raggedright\strut
Y\strut
\end{minipage}\tabularnewline
\begin{minipage}[t]{0.10\columnwidth}\raggedright\strut
\(LAI^{dead}_i\)\strut
\end{minipage} & \begin{minipage}[t]{0.09\columnwidth}\raggedright\strut
\(m^2 \cdot m^{-2}\)\strut
\end{minipage} & \begin{minipage}[t]{0.06\columnwidth}\raggedright\strut
\texttt{LAI\_dead}\strut
\end{minipage} & \begin{minipage}[t]{0.43\columnwidth}\raggedright\strut
Dead leaf area index\strut
\end{minipage} & \begin{minipage}[t]{0.07\columnwidth}\raggedright\strut
Y\strut
\end{minipage} & \begin{minipage}[t]{0.07\columnwidth}\raggedright\strut
Y\strut
\end{minipage}\tabularnewline
\begin{minipage}[t]{0.10\columnwidth}\raggedright\strut
\(LAI^{\phi}_i\)\strut
\end{minipage} & \begin{minipage}[t]{0.09\columnwidth}\raggedright\strut
\(m^2 \cdot m^{-2}\)\strut
\end{minipage} & \begin{minipage}[t]{0.06\columnwidth}\raggedright\strut
\texttt{LAI\_expanded}\strut
\end{minipage} & \begin{minipage}[t]{0.43\columnwidth}\raggedright\strut
Current expanded leaf area index\strut
\end{minipage} & \begin{minipage}[t]{0.07\columnwidth}\raggedright\strut
Y\strut
\end{minipage} & \begin{minipage}[t]{0.07\columnwidth}\raggedright\strut
Y\strut
\end{minipage}\tabularnewline
\bottomrule
\end{longtable}

Height values refer to average height of individuals included in the
cohort, and the same for crown ratio and \(DBH\). \(LAI\) variables
refer to one-side leaf area of plants in the cohort per surface area of
the stand. While plant height is relatively easy to measure, it is
normally quite hard to estimate leaf area indices for stands and
species. Package \textbf{medfate} provides some utilities to provide
estimates crown ratio and \(LAI\) from forest inventory data
(e.g.~heights, DBH and density for measured trees), using allometric
relationships calibrated for catalonia (see chapter \ref{leafbiomass}).

Vegetation characteristics stay constant during simulations using
function \texttt{spwb()}, although the actual expanded leaf area and
dead leaf area may vary is the species is winter deciduous. Function
\texttt{growth()} can modify any of the vegetation attributes.

\subsection{Belowground parameters}\label{belowground-parameters}

The root system of each plant cohorts is described using the proportion
of fine roots in each soil layer:

\begin{itemize}
\tightlist
\item
  \(v_{i,s}\) {[}\texttt{V{[}i,s{]}}{]}: The proportion of fine roots in
  each soil layer \(s\).
\end{itemize}

The rooting system of each cohort \(i\) (i.e.~the proportions
\(v_{i,s}\)) can be defined assuming conic distribution of fine roots
(see \texttt{root\_conicDistribution()}). In this case, only rooting
depth parameter is needed to determine fine root proportions.
Alternatively, one can adopt the linear dose response model (Collins and
Bras, 2007; Schenk and Jackson, 2002):

\begin{equation}
Y_i(z)=\frac{1}{1+(z/Z_{50,i})^{c_i}}
\end{equation}

where \(Y_i(z)\) is the cumulative fraction of fine root mass located
between surface and depth \(z\); \(Z_{50,i}\) is the depth above which
50\% of the root mass is located; and \(c_i\) is a shape parameter
related to \(Z_{50,i}\) and \(Z_{95,i}\) as
\(c_i = 2.94 / \ln(Z_{50,i} / Z_{95,i})\) (see
\texttt{root\_ldrDistribution()}).

\begin{figure}
\centering
\includegraphics{medfatebook_files/figure-latex/unnamed-chunk-4-1.pdf}
\caption{\label{fig:unnamed-chunk-4}Examples of root density profile
according to a conic distribution (left) and the linear dose response
model (right).}
\end{figure}

\section{Metereological input}\label{meteoinput}

Weather input data must include variables calculated at the
\textbf{daily} scale. Data should be arranged in a data frame with days
in rows and variables in columns. The following table indicates the
symbols, units, definitions and the variable name in R.

\begin{longtable}[]{@{}llll@{}}
\toprule
\begin{minipage}[b]{0.11\columnwidth}\raggedright\strut
Symbol\strut
\end{minipage} & \begin{minipage}[b]{0.10\columnwidth}\raggedright\strut
Units\strut
\end{minipage} & \begin{minipage}[b]{0.12\columnwidth}\raggedright\strut
R param\strut
\end{minipage} & \begin{minipage}[b]{0.45\columnwidth}\raggedright\strut
Description\strut
\end{minipage}\tabularnewline
\midrule
\endhead
\begin{minipage}[t]{0.11\columnwidth}\raggedright\strut
\(T_{mean}\)\strut
\end{minipage} & \begin{minipage}[t]{0.10\columnwidth}\raggedright\strut
\(^{\circ} \mathrm{C}\)\strut
\end{minipage} & \begin{minipage}[t]{0.12\columnwidth}\raggedright\strut
\texttt{MeanTemperature}\strut
\end{minipage} & \begin{minipage}[t]{0.45\columnwidth}\raggedright\strut
Mean temperature\strut
\end{minipage}\tabularnewline
\begin{minipage}[t]{0.11\columnwidth}\raggedright\strut
\(T_{min}\)\strut
\end{minipage} & \begin{minipage}[t]{0.10\columnwidth}\raggedright\strut
\(^{\circ} \mathrm{C}\)\strut
\end{minipage} & \begin{minipage}[t]{0.12\columnwidth}\raggedright\strut
\texttt{MinTemperature}\strut
\end{minipage} & \begin{minipage}[t]{0.45\columnwidth}\raggedright\strut
Minimum temperature\strut
\end{minipage}\tabularnewline
\begin{minipage}[t]{0.11\columnwidth}\raggedright\strut
\(T_{max}\)\strut
\end{minipage} & \begin{minipage}[t]{0.10\columnwidth}\raggedright\strut
\(^{\circ} \mathrm{C}\)\strut
\end{minipage} & \begin{minipage}[t]{0.12\columnwidth}\raggedright\strut
\texttt{MaxTemperature}\strut
\end{minipage} & \begin{minipage}[t]{0.45\columnwidth}\raggedright\strut
Maximum temperature\strut
\end{minipage}\tabularnewline
\begin{minipage}[t]{0.11\columnwidth}\raggedright\strut
\(RH_{min}\)\strut
\end{minipage} & \begin{minipage}[t]{0.10\columnwidth}\raggedright\strut
\%\strut
\end{minipage} & \begin{minipage}[t]{0.12\columnwidth}\raggedright\strut
\texttt{MinRelativeHumidity}\strut
\end{minipage} & \begin{minipage}[t]{0.45\columnwidth}\raggedright\strut
Minimum relative humidity\strut
\end{minipage}\tabularnewline
\begin{minipage}[t]{0.11\columnwidth}\raggedright\strut
\(RH_{max}\)\strut
\end{minipage} & \begin{minipage}[t]{0.10\columnwidth}\raggedright\strut
\%\strut
\end{minipage} & \begin{minipage}[t]{0.12\columnwidth}\raggedright\strut
\texttt{MaxRelativeHumidity}\strut
\end{minipage} & \begin{minipage}[t]{0.45\columnwidth}\raggedright\strut
Maximum relative humidity\strut
\end{minipage}\tabularnewline
\begin{minipage}[t]{0.11\columnwidth}\raggedright\strut
\(P\)\strut
\end{minipage} & \begin{minipage}[t]{0.10\columnwidth}\raggedright\strut
\(L \cdot m^{-2} = mm\)\strut
\end{minipage} & \begin{minipage}[t]{0.12\columnwidth}\raggedright\strut
\texttt{Precipitation}\strut
\end{minipage} & \begin{minipage}[t]{0.45\columnwidth}\raggedright\strut
Precipitation\strut
\end{minipage}\tabularnewline
\begin{minipage}[t]{0.11\columnwidth}\raggedright\strut
\(PET\)\strut
\end{minipage} & \begin{minipage}[t]{0.10\columnwidth}\raggedright\strut
\(L \cdot m^{-2} = mm\)\strut
\end{minipage} & \begin{minipage}[t]{0.12\columnwidth}\raggedright\strut
\texttt{PET}\strut
\end{minipage} & \begin{minipage}[t]{0.45\columnwidth}\raggedright\strut
Potential evapotranspiration, preferably calculated using Penman's
equation\strut
\end{minipage}\tabularnewline
\begin{minipage}[t]{0.11\columnwidth}\raggedright\strut
\(Rad\)\strut
\end{minipage} & \begin{minipage}[t]{0.10\columnwidth}\raggedright\strut
\(MJ \cdot m^{-2}\)\strut
\end{minipage} & \begin{minipage}[t]{0.12\columnwidth}\raggedright\strut
\texttt{Radiation}\strut
\end{minipage} & \begin{minipage}[t]{0.45\columnwidth}\raggedright\strut
Solar radiation after accounting for clouds\strut
\end{minipage}\tabularnewline
\begin{minipage}[t]{0.11\columnwidth}\raggedright\strut
\(u\)\strut
\end{minipage} & \begin{minipage}[t]{0.10\columnwidth}\raggedright\strut
\(m \cdot s^{-1}\)\strut
\end{minipage} & \begin{minipage}[t]{0.12\columnwidth}\raggedright\strut
\texttt{WindSpeed}\strut
\end{minipage} & \begin{minipage}[t]{0.45\columnwidth}\raggedright\strut
Wind speed\strut
\end{minipage}\tabularnewline
\bottomrule
\end{longtable}

Not all models require all weather variables, this will be indicated in
the corresponding model chapters. For convenience, we recommend
meteorological input to be generated using package \textbf{meteoland}
\citep{DeCaceres2018}.

\section{Control parameters}\label{control-parameters}

Control parameters are a list of parameter values, initialized using
function \texttt{defaultControl()}, that the user can modify to change
the general behavior of model functions. Not all control parameters are
relevant to all model functions

\part{Basic water balance
modelling}\label{part-basic-water-balance-modelling}

\chapter{Basic water balance model}\label{basic-water-balance-model}

\section{Design principles}\label{design-principles}

The water balance model described in \citet{DeCaceres2015} calculates
temporal variations in soil moisture on a daily step basis for the input
forest stand and for the period corresponding to input weather data.

The model considers only the vertical spatial dimension of the stand and
not the horizontal distribution of plants within it. In other words, the
model is not spatially explicit (i.e., plants do not interact for space
explicitly). Still, the stand is divided into groups of plants, here
referred to as \emph{plant cohorts} of different species, height and
leaf area index (\(LAI\)). Height and \(LAI\) values determine
competition for light. \(LAI\) values also drive competition for water,
along with root distribution. The soil water balance follows the design
principles of SIERRA \citep{Mouillot2001, Ruffault2014, Ruffault2013}
and BILJOU \citetext{\citealp{Granier1999}; \citeyear{Granier2007}},
although some features are taken from other models \citep{Kergoat1998}.
Potential evapotranspiration (\(PET\)) is given as input and the model
determines maximum canopy transpiration (\(Tr_{\max}\)) using an
empirical relationship between the \(LAI\) of the stand and the ratio
\(Tr_{\max}/PET\) \citep{Granier1999}. Actual plant transpiration is
estimated using a function that depends on current soil moisture levels,
species-specific drought resistance, fine root distribution and the
degree of shading of the target plant cohort.

\section{State variables}\label{state-variables}

The model has the following state variables:

\begin{itemize}
\tightlist
\item
  Cumulative growth degree days, \(GDD\), are tracked by the model and
  determine leaf phenological status of winter deciduous species.
\item
  Daily soil moisture content dynamics on each layer \(s\) are tracked
  using \(W_s = \theta(\Psi_s)/ \theta_{fc,s}\), the \textbf{proportion
  of volumetric soil moisture in relation to field capacity}, where
  field capacity, \(\theta_{fc,s}\), is assumed to correspond to
  \(\Psi_{fc} = -0.033\) MPa. Note that \(W_s\) values larger than one
  are possible if the soil is between field capacity and saturation
  (which can happen if deep drainage is not allowed).
\item
  If snow accumulation/melting is considered, the model tracks
  \(S_{snow}\), the snow water equivalent (in mm) of the snow pack
  storage over the surface.
\item
  If irreversible cavitation is activated, the model tracks
  \(P_{embolized,i}\), the maximum value of daily drought stress so far
  experienced.
\end{itemize}

\section{Water balance}\label{water-balance}

Daily variations in soil water content (mm) can be summarized as:

\begin{equation}
\Delta{SWC} = Pr + Sm - In - Ru - Dd - Es -Tr
\end{equation}

where \(Pr\) is precipitation as rainfall, \(Sm\) is water from melting
the snow pack (if considered), \(In\) is the interception loss (i.e.,
water evaporated after being intercepted by the canopy), \(Ru\) is
surface runoff, \(Dd\) is deep drainage (i.e.~water percolated to layers
beyond soil depth), \(Es\) is evaporation from soil and \(Tr\) is plant
transpiration. If snow dynamics are considered, the water balance of the
snow pack is defined as:

\begin{equation}
\Delta{S_{snow}} = Ps - Sm
\end{equation}

where \(Ps\) is precipitation as snowfall and \(Sm\) is snow melt.

\section{Process scheduling}\label{process-scheduling}

Every day water balance is perfomed as follows. The model first updates
leaf area values according to the phenology of species and calculates
light extinction. After that, the model updates soil water content of
soil layers in several steps:

\begin{itemize}
\tightlist
\item
  Update leaf area values according to the phenology of species (section
  \ref{leafphenology}).
\item
  If snow dynamics are included, increase snow pack from snow
  precipitation (\(Ps\)) and decreases it from snow melting (\(Sm\))
  (section \ref{snowpack}).
\item
  Increase soil moisture due to rainfall (\(Pr\)), surface runon
  (\(Ro\)) and snow melting (\(Sm\)), after accounting for interception
  loss (\(In\)), surface runoff (\(Ru\)) and deep drainage (\(Dd\))
  (sections \ref{interception} and \ref{runoff})
\item
  Decrease water content due to bare soil evaporation (\(Es\)) (section
  \ref{soilevaporation}).
\item
  Determine transpiration, photosynthesis and drought stress for each
  plant cohort and decrease water content due to plant transpiration
  (\(Tr\)) (chapter \ref{transpirationgranier}).
\end{itemize}

\section{Model input}\label{model-input}

\subsection{Metereological input}\label{metereological-input}

The minimum weather variables required to run the model are mean
temperature (\(T_{mean}\)), precipitation (\(P\)) and potential
evapotranspiration (\(PET\)). Solar radiation (\(Rad\)) is required if
snow pack dynamics have to be simulated. Optionally, the user may also
specify values of wind speed (\(u\)) that are used to control leaf fall
in autumn. Definitions and units of these variables are given in section
\ref{meteoinput}.

\subsection{Control parameters}\label{control-parameters-1}

The control values relevant for the simple water balance model are:

\begin{itemize}
\tightlist
\item
  \texttt{verbose\ {[}=TRUE{]}}: Whether extra console output is desired
  during simulations.
\item
  \texttt{soilFunctions\ {[}=\ "SX"{]}}: Water retention curve model,
  either ``SX'' (for Saxton) or ``VG'' (for Van Genuchten).
\item
  \texttt{leafPhenology\ {[}=\ TRUE{]}}: Whether leaf phenology is
  simulated.
\item
  \texttt{snowpack\ {[}=\ TRUE{]}}: Whether dynamics of snow pack are
  included.
\item
  \texttt{drainage\ {[}=\ TRUE{]}}: Whether water exceeding the field
  capacity of the bottom layer is lost by deep drainage into a
  non-reachable compartment.
\item
  \texttt{transpirationMode\ {[}=\ "Granier"{]}}: Transpiration model,
  in this case ``Granier''.
\item
  \texttt{defaultWindSpeed\ {[}=\ 2.5{]}}: Default value for wind speed
  (in \(m \cdot s^{-1}\)) when this is missing (only used for leaf
  fall).
\item
  \texttt{cavitationRefill\ {[}=\ TRUE{]}}: If \texttt{FALSE} the model
  operates in a irreversible cavitation mode.
\end{itemize}

\section{Model output}\label{model-output}

Function \texttt{spwb} returns a list object whose elements are data
tables. Each table has dates as rows and has different output variables:

\begin{itemize}
\tightlist
\item
  \texttt{WaterBalance}: Components of the climatic and soil daily water
  balance (i.e.~net precipitation, infiltration, runoff, plant
  transpiration\ldots{}).
\item
  \texttt{Soil}: Daily variation of soil variables (volume, moisture
  relative to field capacity, water potential) for soil layers.
\item
  \texttt{PlantLAI}: Daily leaf area index for each plant cohort.
\item
  \texttt{PlantTranspiration}: Daily transpiration (in mm) for each
  plant cohort.
\item
  \texttt{PlantPhotosynthesis}: Daily net photosynthesis (in
  \(g C \cdot m^{-2}\)) for each plant cohort.
\item
  \texttt{PlantPsi}: Daily water potential of each plant (in MPa).
\item
  \texttt{PlantStress}: Daily stress level suffered by each plant cohort
  (relative whole-plant conductance).
\end{itemize}

\chapter{Leaf phenology and light
extinction}\label{leaf-phenology-and-light-extinction}

\section{Leaf phenology and leaf fall}\label{leafphenology}

Given a base temperature (\(T_{base} = 5^{\circ} \mathrm{C}\)), the
growth degree days (\(GDD\)) are zero for all those days where mean
temperature \(T_{mean}\) is below \(T_{base}\) and start increasing when
temperatures become warmer than this threshold. In other words, the
\(GDD\) function accumulates \(\max(0.0, T_{mean} - T_{base})\) for all
days previous to the current one. At the end of a year the cummulative
value is set again to zero. Plant species can have either evergreen or
winter-deciduous phenology. Evergreen plants maintain constant leaf area
over the year, whereas in deciduous plants leaf-phenological status is
updated daily, represented by \(\phi_i\), the fraction of maximum leaf
area. Leaf area index (LAI) values of deciduous plants are adjusted for
leaf phenology following \citep{Prentice1993, Sitch2003}:

\begin{equation}
LAI_{i}^{\phi}=LAI^{live}_i\,\cdot\phi_i
\end{equation}

Budburst occurs when daily temperature exceeds \(T_{base}\) and
\(\phi_i\) increases linearly from 0 to 1 as function of the degree days
above \(T_{base}\), until a the value \(S_{GDD,i}\) is reached
(i.e.~until \(GDD > S_{GDD,i}\)). In autumn, \(\phi_i\) drops to 0 when
average daily temperature falls again below \(T_{base}\)
\citep{Sitch2003}. The drop of \(\phi_i\) causes live expanded leaves to
become dead leaves. To avoid a sudden decrease of leaf area, dead leaves
are kept in the canopy and they are reduced daily using a negative
exponential function of wind speed:

\begin{equation}
LAI^{dead}_i=LAI^{dead}_i\,\cdot e^{- u/10}
\end{equation}

where \(u\) is wind speed in (m·s\(^{-1}\)).

After updating the leaf area of each cohort, the model updates the total
leaf area of the stand. To simplify the notation, let us call
\(LAI^{all}_{i}\) the sum of dead and live expanded leaves of a cohort
\(i\):

\begin{equation}
LAI^{all}_{i} = LAI^{\phi}_{i}+LAI^{dead}_{i}
\end{equation}

If there are \(c\) plant cohorts, the leaf area index of the whole
stand, \(LAI_{stand}\) is then:

\begin{equation}
LAI_{stand} = \sum_{i=1}^c{LAI_{i}^{all}}= \sum_{i=1}^c{LAI^{\phi}_{i}+LAI^{dead}_{i}}
\end{equation}

Although simulations will normally start in winter and \(GDD\) will be
zero at the beginning, the user can specify a starting non-zero value
for \(GDD\) in the object created by function \texttt{spwbInput()}.

\section{Light extinction}\label{light-extinction}

The proportion of photosynthetic active radiation (PAR) decreases with
leaf area following thee Beer-Lambert's light extinction equation. To
calculate the proportion of PAR available for a given plant cohort one
must accumulate the light extinction caused by cohorts whose crown is
above that of the target cohort:

\begin{equation}
L^{PAR}_i=e^{-\sum_{h=1}^{c}{k_{PAR,h} \cdot LAI_{h}^{all} \cdot p_{ih}}}
\end{equation}

where \(k_{PAR,h}\) is the PAR extinction coefficient of cohort \(h\).
Because plant cohorts may differ in height only slightly, leaf area is
multiplied by \(p_{ih}\), the proportion of the crown of cohort \(h\)
that overtops that of cohort \(i\):

\begin{equation}
p_{ih}=\max(0,\min(1,(H_h-H_i\cdot (1 - CR_i))/(H_h-H_h\cdot (1 - CR_h))))
\end{equation}

where \(CR_i\) and \(CR_h\) are the crown ratio of cohorts \(i\) and
\(h\). In other terms, cohorts whose crown is completely above that of
\(i\) reduce the amount of light available more strongly by than cohorts
that are only slightly taller. \(L^{PAR}_{ground}\), the proportion of
PAR that reaches the ground, is calculated as:

\begin{equation}
L^{PAR}_{ground}=e^{-\sum_{i=1}^{c}{k_{PAR,i} \cdot LAI_{i}^{all}}}
\end{equation}

The shortwave radiation (SWR; 400-3000 nm) energy absorbed by each plant
cohort needs to be calculated to determine plant transpiration, and the
radiation absorbed by the soil is needed to calculate soil evaporation.
Foliage absorbs a higher proportion of PAR than SWR; thus, the
extinction coefficient is higher for PAR than for SWR. However, values
for the ratio of extinction coefficients are rather constant. Following
Friend et al. \citeyearpar{Friend1997} here it is assumed that the
extinction coefficient for PAR is 1.35 times larger than that for SWR
(i.e. \(k_{SWR,i} = k_{PAR,i}/1.35\)).

To calculate radiation absorption, where the vertical dimension of the
plot is divided into 1 m deep layers, and the SWR absorbed is calculated
for each plant cohort in each layer. Let \(n\) be the number of canopy
layers. The fraction of radiation incident on layer \(j\) that is
absorbed in the same layer is:

\begin{equation}
f_j=1 - e^{-\sum_{i=1}^{c}{k_{SWR,i} \cdot LAI_{i,j}^{all}}}
\end{equation}

where \(LAI_{i,j}^{all} = LAI_{i,j}^{\phi}+LAI_{i,j}^{dead}\) is the
leaf area index of cohort \(i\) in layer \(j\). Hence, the fraction
transmitted is \((1-f_j)\). The fraction of radiation incident on layer
\(j\) that is absorbed by expanded leaves of plant cohort \(i\) in that
layer (\(f_{ij}\)) is calculated from the relative contribution of these
leaves to the total absorption in the layer:

\begin{equation}
f_{ij} = f_j \cdot \frac{k_{SWR,i}\cdot LAI_{i,j}^{\phi}}{\sum_{h=1}^{c}{k_{SWR,h} \cdot LAI_{h,j}^{all}}}
\end{equation}

The fraction of canopy radiation absorbed by a plant cohort \(i\) across
all layers is found by adding the fraction absorbed in each layer:

\begin{equation}
f_i = \sum_{j=1}^{n}{f_{ij}\cdot \prod_{h>j}^{n}{(1-f_h)}}
\end{equation}

where for each layer the fraction of the radiation incident in the
canopy that reaches the layer is found by multiplying the transmitted
fractions across the layers above it. The proportion of (shortwave) net
radiation absorbed by the ground is simply:

\begin{equation}
L^{SWR}_{ground} = 1 - \sum_{j}^{n}{f_j}
\end{equation}

\chapter{Hydrology}\label{hydrology}

\section{Snow pack dynamics}\label{snowpack}

Precipitation is considered be snow when \(T_{mean}<0\). Interception of
snow by the canopy is neglected, and all snow is assumed to accumulate
in a single storage compartment \(S_{snow}\) over the soil (i.e.~canopy
snow storage capacity is neglected). A very simple snow submodel is used
for snow pack dynamics (accumulation and melt), taken from
\citet{Kergoat1998}. When mean air temperature is above 0 Celsius
\(T_{mean}>0\), a simple energy budget relates snow melt, \(Sm\) (mm),
to air temperature and soil-level radiation (see function
\texttt{hydrology\_snowMelt}):

\begin{equation}
Sm = \frac{Rad\cdot L^{SWR}_{ground}\cdot (1-\alpha_{ice}) + \tau_{day} \cdot T_{mean} \cdot \rho \cdot C_p/r_{s}}{\lambda_{ice}}
\end{equation}

where \(Rad\) is solar radiation (MJ·m\(^{-2}\)), \(L^{SWR}_{ground}\)
is the fraction of (short-wave) radiation reaching the ground,
\(\alpha_{ice} = 0.9\) is the albedo of snow, \(\tau_{day} = 86400\) is
the day duration in s, \(\rho\) is the air density (depending on
temperature and elevation), \(C_{p} = 1013.86 \cdot 10^{-6}\)
MJ·kg\(^{-1}\)·C\(^{-1}\) is the specific heat capacity of the air and
\(r_{s} = 100\) s·m\(^{-1}\) is the snow aerodynamic resistance and
\(\lambda_{ice} = 0.33355\)MJ·kg is the (latent) heat of fusion of snow.

\section{Rainfall interception loss}\label{interception}

Interception loss is only modelled for water precipitation (i.e.~snow
interception is not modelled). Rainfall interception loss, \(In\), is
estimated using the \citet{Gash1995} analytical interception model for
sparse canopies, where rain is assumed to fall in a single event during
the day. First, the amount of rainfall needed to saturate the canopy is
calculated:

\begin{equation}
P_G = - \frac{S/C}{(E/R)} \cdot \ln(1-(E/R))
\end{equation}

where \(S\) is the canopy water storage capacity (in mm) -- i.e.~the
minimum amount of water needed to saturate the canopy --, \(C\) is the
canopy cover and \((E/R)\) is the ratio of evaporation rate to rainfall
rate during the rainfall event. Simplifying assumptions are made to
determine \((E/R)\). Values of the Evaporation-to-rainfall ratio are
calculated from daily potential evapotranspiration and rainfall, while
accounting for seasonal variation in rainfall intensity (mm/h). Minimum
values for rainfall intensity are assumed for convective storms (5.6
mm/h) and synoptic storms (1.5 mm/h) from \citet{Miralles2010}. Synoptic
storms are assumed between December and June, and convective storms are
assumed for the remaining months, as typical in the Mediterranean Basin.

The amount of water evaporated from interception, \(I\) (mm), is
calculated as:

\begin{eqnarray}
In = C\cdot P_G+C\cdot(E/R)\cdot(P-P_G) \: {if}\: P > P_G \\
In = C\cdot P\: {if}\: P \leq P_G
\end{eqnarray}

where \(P\) is the daily gross precipitation (in mm). Net rainfall,
\(Pr_{net}\), is calculated as the difference between gross rainfall and
interception loss. Although interception models are normally applied to
single-canopy stands, we apply the sparse Gash model to the whole stand
(including shrubs). Moreover, in our implementation stem interception is
lumped with canopy interception, so that \(S\) represents both.
Following \citet{Watanabe1996} we estimate \(S\), the canopy water
storage capacity, from adjusted LAI values:

\begin{equation}
S=\sum_{i}{s_{water,i}\cdot LAI_{i}^{\phi}}
\end{equation}

where \(s_{water,i}\) is the depth of water that can be retained by
leaves and trunks of a species \(i\) per unit of leaf area index
(mm·LAI\(^{-1}\)). To estimate the stand cover, \(C\), we use the
complement of the percentage of PAR that reaches the ground, i.e.
\(C = 1 - L^{PAR}_{ground}\) \citep{Deguchi2006}. Fig. 1 below shows
examples of relative throughfall, calculated according to the
interception model, under different situations (see function
\texttt{hydrology\_rainInterception()}).

\begin{figure}
\centering
\includegraphics{medfatebook_files/figure-latex/unnamed-chunk-5-1.pdf}
\caption{\label{fig:unnamed-chunk-5}Examples of canopy interception with
different S (canopy water storage capacity), E/R (ratio between
evaporation and rainfall rates) and p (throughfall coefficient; p = 1 -
C)}
\end{figure}

\section{Runoff, infiltration and percolation}\label{runoff}

The water that reaches the soil is the sum of net rainfall
(\(Pr_{net}\)), runon (\(Ro\)) and melted snow (\(Sm\)). The amount of
water infiltrating into the soil is \(Pr_{net} + Sm + Ro - Ru\), where
\(Ru\) is the water lost by surface runoff (see function
\texttt{hydrology\_infiltrationAmount}). Surface runoff (in mm), is
calculated using the USDA SCS curve number method, as in
\citet{Boughton1989}:

\begin{equation}
Ru=\frac{(Pr_{net} + Ro + Sm - 0.2 \cdot V_{soil})^2}{(Pr_{net} + Ro + Sm - 0.8 \cdot V_{soil})}
\end{equation}

where \(V_{soil}\) (in mm) is the overall soil water retention capacity.
Following \citet{Granier1999}, part of the water reaching one soil layer
percolates quickly through the macropores. The amount of water reaching
each layer through macropores is modelled using an extinction function
(see function \texttt{hydrology\_infiltrationRepartition}). The
remaining water is retained by the micropores refilling the current soil
layer. When this soil layer reaches its field capacity the excess of
water also percolates to the soil layer below.

\begin{figure}
\centering
\includegraphics{medfatebook_files/figure-latex/unnamed-chunk-6-1.pdf}
\caption{\label{fig:unnamed-chunk-6}Examples of infiltration/runoff
calculation for different values of net rainfall and overall retention
capacity, \(V_{soil}\), calculated from different soil depths
(topsoil+subsoil) and assuming that soil texture is 15\% clay and 25\%
sand. Rock fragment content was 25\% and 40\% for the topsoil and
subsoil, respectively.}
\end{figure}

If \texttt{drainage\ =\ TRUE}, the water percolating from the lowest
layer is considered deep drainage, \(Dd\), and soil layers are never
above field capacity. If \texttt{drainage\ =\ FALSE}, the excess of
water in the lowest soil layer is not lost via drainage, so that this
layer (and those above) can be filled up to saturation. This additional
filling is done starting from below, and when a given layer reaches
saturation, the model adds the remaining water to the layer above. If
all layers are filled to saturation, the remaining water is assumed to
be infiltration excess and is added to the surface runoff.

\section{Bare soil evaporation}\label{soilevaporation}

Evaporation from the soil surface is the last component of the soil
water balance to be calculated. There is a difference in the way that
soil evaporative demand is calculated depending on the transpiration
mode. Potential evaporation from the soil (\(PE_{soil}\); in
\(mm\cdot day^{-1}\)) is defined as the product between \(PET\) and
\(L^{SWR}_{ground}\), the proportion of SWR absorbed by the ground:

\begin{equation}
PE_{soil} =  PET\cdot L^{SWR}_{ground}
\end{equation}

Actual evaporation from the soil surface is modeled as in
\citet{Mouillot2001}, who followed \citet{Ritchie1972}. First, the model
determines the time needed to evaporate the current water deficit
(difference between field capacity and current moisture) in the surface
soil layer:

\begin{equation}
t = \left \{ \frac{V_1\cdot(1- W_1)}{\gamma_{soil}} \right \}
\end{equation}

where \(\gamma_{soil}\) is the maximum daily evaporation
(mm·day\(^{-1}\)). The calculated time is used to determine the
`supplied' evaporation, \(SE_{soil}\):

\begin{equation}
SE_{soil} = \gamma_{soil} \cdot (\sqrt{t+1}-\sqrt{1})
\end{equation}

The amount of water evaporated from the soil, \(E_{soil}\), is then
calculated as the minimum between supply and demand \citep{Federer1982},
the latter being the product of PET and the proportion of light that
reaches the ground (see function
\texttt{hydrology\_soilEvaporationAmount}):

\begin{equation}
E_{soil} = \min(PE_{soil}, SE_{soil})
\end{equation}

Finally, \(E_{soil}\) is distributed along the soil profile according to
an exponential decay function with an extinction coefficient
\(\kappa_{soil}\) \citep{Mouillot2001}.

\begin{figure}
\centering
\includegraphics{medfatebook_files/figure-latex/unnamed-chunk-7-1.pdf}
\caption{\label{fig:unnamed-chunk-7}Cumulative bare soil evaporation for
different values of maximum evaporation rate and extinction coefficient.
Three soil layers (0 -- 30 cm; 30 -- 150 cm; 150 -- 400 cm) are
initialized at field capacity (\(V_1 = 50 mm\); \(V_2 = 201 mm\);
\(V_3 = 35 mm\)). \(PE_{soil}\) was assumed not to be limiting. When the
extinction coefficient is smaller a higher proportion of the evaporated
water is removed from the subsoil and less from the topsoil. This causes
more water being available to calculate t in the next step.}
\end{figure}

\chapter{Transpiration and photosynthesis under Granier's
model}\label{transpirationgranier}

\section{PET and maximum canopy
transpiration}\label{pet-and-maximum-canopy-transpiration}

In this model daily potential evapotranspiration (\(PET\); in
mm·day\(^{-1}\); the amount of evaporation that would occur if a
sufficient water source was available) is part of the meteorological
input and has to be calculated externally. \(PET\) is assumed to
represent open water evaporation potential (like in Penman's formula).
Maximum canopy transpiration \(Tr_{\max}\) not only depends on \(PET\)
but also on the amount of transpirating surface. To estimate
\(Tr_{\max}\) we take the approach of \citet{Granier1999}, where
\(Tr_{\max}/PET\) is a function of \(LAI_{stand}\) -- the cumulative
leaf area of the forest stand --, according to the empirical equation:

\begin{equation}
\frac{Tr_{\max}}{PET}= -0.006\,LAI_{stand}^2+0.134\,LAI_{stand}+0.036
\end{equation}

This equation has already been adopted for Mediterranean biomes (Fyllas
and Troumbis, 2009; Ruffault et al., 2013).

\begin{figure}

{\centering \includegraphics{medfatebook_files/figure-latex/unnamed-chunk-8-1} 

}

\caption{Empirical relationship between $Tr_max/PET$ and $LAI_{stand}$}\label{fig:unnamed-chunk-8}
\end{figure}

The maximum transpiration for a given plant cohort \(i\) is calculated
as the portion of \(Tr_{\max}\) defined by the fraction of total
absorbed SWR that is due to cohort \(i\):

\begin{equation}
Tr_{\max, i} = Tr_{\max} \cdot \frac{f_i}{\sum_{j}{f_j}}
\end{equation}

\section{Actual plant transpiration}\label{actual-plant-transpiration}

Actual plant transpiration depends on soil moisture and is calculated
for each plant cohort and each soil layer separately. \(Tr_{i,s}\) (in
mm·day\(^{-1}\)) represents the transpiration made by cohort \(i\) from
layer \(s\). Actual plant transpiration from a given layer is regulated
by soil moisture and the resistance to water flow through the plant. For
each plant cohort \(i\) and soil layer \(s\), the model first estimates
the a whole-plant relative water conductance, \(K_{i,s}\), which varies
between 0 and 1 depending on \(\Psi_{extract,i}\), the potential at
which conductance is 50\% of maximum, and \(\Psi_s\), the water
potential in layer \(s\) (see function \texttt{hydraulics\_psi2K()}):

\begin{equation}
K_{i,s}=K_{i}(\Psi_s) = \exp \left \{\ln{(0.5)}\cdot \left[ \frac{\Psi_s}{\Psi_{extract,i}} \right] ^r \right \} 
\end{equation}

where \(r\) is an exponent that modulates the steepness of the decrease
in relative conductance when soil potential becomes negative (by
default, \(r = 3\)) and \(\ln(0.5)\) is used to ensure that
\(K_{i}(\Psi_{extract,i}) = 0.5\) (Fig. 4).

\begin{figure}

{\centering \includegraphics{medfatebook_files/figure-latex/unnamed-chunk-9-1} 

}

\caption{Whole-plant relative water conductance functions for different water potential values ($r = 3$ in all cases).}\label{fig:unnamed-chunk-9}
\end{figure}

Actual transpiration of plant cohort \(i\) from a given soil layer
\(s\), \(Tr_{i,s}\), is defined as the product of (Mouillot et al.,
2001): (i) the maximum transpiration of the plant cohort; (ii) the
relative whole-plant conductance, \(K_{i,s}\), corresponding to the
species and water potential in layer \(s\); (iii) the proportion of
plant fine roots in layer \(s\), \(v_{i,s}\):

\begin{equation}
Tr_{i,s} =  Tr_{\max,i} \cdot K_{i,s} \cdot v_{i,s}
\end{equation}

The total amount of water transpired by plants, \(Tr\) (in
mm·day\(^{-1}\)), is the sum of \(Tr_{i,s}\) values over all plant
cohorts and soil layers:

\begin{equation}
Tr =\sum_{s}\sum_{i}{Tr_{i,s}}
\end{equation}

Assuming no water limitations (i.e. \(K_{i,s} = 1\)), we have that
\(Tr = Tr_{\max}\). Total stand transpiration will be lower than
\(Tr_{\max}\) if soil water potential in any layer is negative enough to
cause a significant reduction in whole-plant conductance. At the plant
level, the transpiration of a given plant cohort will be lower than that
of others if: (1) the cohort is under the shade (it reduces \(f_i\) and
hence \(Tr_{\max,i}\)); (2) the cohort has a lower amount of leaf area
(it reduces \(f_i\) and hence \(Tr_{\max,i}\)); (3) the soil layers
exploited by the cohort have more negative water potentials (it reduces
\(K_{i,s}\)).

\section{Plant photosynthesis}\label{plant-photosynthesis}

Because it is useful for growth and forest dynamics, and for
compatibility with the `Sperry' transpiration mode, the `Granier'
transpiration mode also calculates net assimilated carbon. Assuming a
constant maximum water use efficiency (WUE), photosynthesis for a given
plant cohort \(i\) (in g C · m\(^{-2}\) · day\(^{-1}\)) is estimated as
a function of transpiration \citep{Mouillot2001}:

\begin{equation}
A_n = \alpha \cdot WUE_{\max} \cdot Tr_i
\end{equation}

where \(Tr_i\) is the transpiration of plant cohort \(i\),
\(WUE_{\max}\) is the maximum water use efficiency of the corresponding
species (in g C · mm\(^{-1}\)) and \(\alpha = T_{mean}/20\) is bounded
between 0 and 1.

\subsubsection{Plant drought stress and plant water potential}

Similarly to \citet{Mouillot2002}, daily drought stress of a given plant
cohort \(i\), \(DDS_i\), is defined as the complement of relative
whole-plant conductance and is aggregated across soil layers using the
proportion of fine roots in each layer as weights:

\begin{equation}
DDS_i=\phi_i \cdot \sum_{s}{(1-K_{i,s})\cdot v_{i,s}}
\end{equation}

Leaf-phenological status is included to prevent winter deciduous plants
from suffering drought stress during winter. Daily drought stress values
can be later used to define drought stress indices for larger temporal
scales, as presented in the main text.

Granier's transpiration model does not allow estimating a water
potential drop from soil to the leaf. Moreover, in a multilayered soil
it is difficult to know what would be the water potential of the plant.
Despite these limitations, a gross surrogate of (pre-dawn) `leaf water
potential' (\(\Psi_{leaf,i}\); in MPa) may be obtained averaging
whole-plant relative conductance values:

\begin{equation}
\Psi_{leaf,i}=K^{-1}(K_i) = K^{-1}\left(\sum_{s}{K_{i,s}\cdot v_{i,s}}\right)
\end{equation}

where \(K_i\) is the average whole-plant relative conductance obtained
from the scalar product of conductances and fine root proportions and
\(K^{-1}\) is the inverse of the whole-plant conductance function (see
function \texttt{hydraulics\_K2Psi()}).

\section{Irreversible cavitation and hydraulic
disconnection}\label{irreversible-cavitation-and-hydraulic-disconnection}

The water balance model is normally run assuming that although soil
drought may reduce transpiration, embolized xylem conduits are
automatically refilled when soil moisture recovers (in other words,
cavitation is reversible). It is possible to simulate irreversible
cavitation by setting \texttt{cavitationRefill\ =\ FALSE} in the control
parameters. This option causes the model to keep track of the maximum
value of drought stress so far experienced:

\begin{equation}
P_{embolized,i}= \max \{P_{embolized,i}, DDS_i \}
\end{equation}

and then \(K_{i,s}\) cannot be larger than the complement of this
maximum drought stress:

\begin{equation}
K_{i,s} = \min \{K_{i}(\Psi_s), 1.0 - P_{embolized,i} \}
\end{equation}

Another optional behavior consists in allowing the plant to disconnect
from the soil when its potential becomes too negative. This may be
advantageous for a cavitation-sensitive plant that is competing for
water with another plant with higher extraction capacity. Parameter
\(K_{rootdisc,i}\) can be used to specify the minimum relative
conductance value that the plant will tolerate without disconnecting
hydraulically from the soil (by default \(K_{rootdisc,i} = 0\)). If,
after possibly accounting for irreversible cavitation,
\(K_{i,s}<K_{rootdisc,i}\) for a given soil layer, then the model
assumes that transpiration from this soil layer is absent. Moreover,
\(K_{i,s}\) is assumed equal to \(K_{rootdisc,i}\) for the sake of
determining plant water potential.

\part{Advanced water balance
modelling}\label{part-advanced-water-balance-modelling}

\chapter{Advanced water balance
model}\label{advanced-water-balance-model}

\section{Design principles}\label{design-principles-1}

The model performs soil water balance and energy balance for the input
forest stand and for the period corresponding to input weather data.
Soil water balance is calculated on a daily step basis for the input
forest stand and for the period corresponding to input weather data. The
model considers only the vertical spatial dimension of the stand, and
not the horizontal distribution of plants within it. Still, the stand is
divided into groups of plants (here referred to as `plant cohorts') of
different species, height and leaf area index (\(LAI\)). As hydrological
processes, the model includes water interception loss \citep{Gash1995},
plant transpiration and hydraulic redistribution, evaporation from soil
\citep{Ritchie1972} and the partition between infiltration and runoff
\citep{Boughton1989}. Water exceeding soil water holding capacity is
lost via deep drainage. For plant transpiration, the model determines
subdaily regulation of leaf water conductance and actual transpiration
involving detailed calculations of hydraulics and photosynthesis
\citep{Sperry2016}. This level of complexity allows a precise estimation
of photosynthesis and hydraulic redistribution of water among soil
layers.

Radiation and energy balances are conducted subdaily at two levels:
canopy/soil and leaf. One one hand the model keeps track of temperature
variation of the air in the canopy (i.e.~canopy energy balance) and in
the soil (i.e.~soil energy balance) as the result of energy exchanges
between them and with the atmosphere. These energy balance equations are
very similar to those of Best et al. \citeyearpar{Best2011} for model
JULES. On the other, the model performs the energy balance at the leaf
level to determine transpiration \citep{Sperry2016}. At this scale,
radiation inputs include shortwave radiation from the atmosphere
absorbed by the leaf and absorbed longwave radiation coming from both
the atmosphere and the canopy itself. Leaf temperature is determined
assuming that the temperature of the air is that of the canopy. After
performing stomatal regulation, the model upscales the transpiration
flux at the canopy level and the corresponding latent heat is used to
complete the calculation of the energy balance at the canopy level. Note
that the latent heat fluxes from evaporation from the soil and of
intercepted water are not currently included in the energy balance.

\section{State variables}\label{state-variables-1}

The model has the following state variables: + Cumulative growth degree
days, \(GDD\), are tracked by the model and determine leaf phenological
status of winter deciduous species. + Daily soil moisture content
dynamics on each layer \(s\) are tracked using
\(W = \theta(\Psi_s)/ \theta_{fc,s}\), the \textbf{proportion of
volumetric soil moisture in relation to field capacity}, where field
capacity, \(\theta_{fc,s}\), is assumed to correspond to
\(\Psi_{fc} = -0.033\) MPa. Soils are not allowed to contain more water
than dictated by their field capacity. + The temperature of the canopy
and of each soil layer (\(T_c\) and \(T_s\); both in ºC) are tracked for
every subdaily step. + If irreversible cavitation is activated, the
model also tracks \(P_{embolized,i}\), the maximum value of daily
drought stress so far experienced.

\section{Water and energy balances}\label{water-and-energy-balances}

\subsection{Water balance}\label{water-balance-1}

Daily variations in soil water content (mm) can be summarized as:

\begin{equation}
\Delta{SWC} = Pr + Sm - In - Ru - Dd - Es -Tr
\end{equation}

where \(Pr\) is precipitation as rainfall, \(Sm\) is water from melting
the snow pack (if considered), \(In\) is the interception loss (i.e.,
water evaporated after being intercepted by the canopy), \(Ru\) is
surface runoff, \(Dd\) is deep drainage (i.e.~water percolated to layers
beyond soil depth), \(Es\) is evaporation from soil and \(Tr\) is plant
transpiration. If snow dynamics are considered, the water balance of the
snow pack is defined as:

\begin{equation}
\Delta{S_{snow}} = Ps - Sm
\end{equation}

where \(Ps\) is precipitation as snowfall and \(Sm\) is snow melt.

\subsection{Energy balance}\label{energy-balance}

The canopy absorbs shortwave radiation from the atmosphere
(\(K_{abs,ca}\)). It also absorbs longwave counterradiation from the
atmosphere (\(L_{abs,ca}\)) and longwave radiation emmited from the soil
(\(L_{abs,cs}\)). These inputs are counterbalanced by the longwave
radiation emmited by the canopy (\(L_{em,c}\)) in both cases. Other
energy fluxes considered are convective exchanges between the canopy and
atmosphere (\(H_{ca}\)) and between the canopy and the soil
(\(H_{cs}\)). Finally, energy is released to the atmosphere through
latent heat (\(LE_{c}\)) produced via transpiration (so far,
interception and soil evaporation are not included). The instantaneous
energy balance equation for the canopy is thus:

\begin{equation}
  C_{c} \cdot \frac{\delta T_{c}}{\delta t} = K_{abs,ca} + (L_{abs,ca} - L_{em,c}) + (L_{abs,cs} - L_{em,c}) - LE_{c} - H_{ca} - H_{cs} 
\end{equation}

where \(C_{c}\) is the canopy thermal capacitance. Like the canopy, the
soil absorbs short- and longwave radiation from the atmosphere
(\(K_{abs,sa}\) and \(L_{abs,sa}\)). It also absorbs longwave radiation
released by the canopy (\(L_{abs,sc} = L_{em,c}\)) and emmits longwave
radiation (\(L_{em,s}\)). Note that \(L_{abs,cs} \leq L_{em,s}\) because
part of the radiation emmitted by the soil is sent to the atmosphere
without being intercepted by the canopy. Finally, it also exchanges
convective energy with the canopy (\(H_{cs}\)). The energy balance
equation for the soil is:

\begin{equation}
  C_{s} \cdot \frac{\delta T_{s}}{\delta t} = K_{abs,sa} + L_{abs,sa} - L_{em,s} + L_{abs,sc} + H_{cs} 
\end{equation}

where \(C_{s}\) is the soil thermal capacitance.

\section{Process scheduling}\label{process-scheduling-1}

Every day the model performs the following actions:

\begin{itemize}
\tightlist
\item
  Update leaf area values according to the phenology of species.
\item
  Increase soil moisture due to precipitation after accounting for
  canopy interception loss, surface runoff and deep drainage.
\item
  Determine subdaily temperature and direct/diffuse irradiance
  variations.
\item
  Determine short- and longwave radiation absorved by the canopy and the
  soil at subdaily steps.
\item
  Determine plant transpiration, photosynthesis and close soil/canopy
  energy balance at subdaily steps. This involves the following
  sub-actions:

  \begin{itemize}
  \tightlist
  \item
    Determine longwave radiation emmited by soil and canopy, according
    to their temperature.
  \item
    Update the water supply function of each plant cohort, according to
    the hydraulic model and the current soil water potential.
  \item
    Calculate leaf temperature and photosynthesis, for shade and sunlit
    leaves of each plant cohort, corresponding to each transpiration
    value of the supply function.
  \item
    Determine stomatal conductance, transpiration and photosynthesis on
    shade and sunlit leaves of each plant cohort according to a profit
    maximization strategy.
  \item
    Update soil moisture after scaling transpiration from leaf to canopy
    level.
  \item
    Complete energy balance of the canopy and the soil (after
    translating plant transpiration to latent heat and calculating
    convective heat exchange for both the canopy and the soil).
  \end{itemize}
\item
  Determine drought stress index for each plant cohort.
\item
  Decrease water content due to bare soil evaporation, closing the water
  balance.
\end{itemize}

\section{Model input}\label{model-input-1}

\subsection{Meteorological input}\label{meteorological-input}

The minimum weather variables required to run the model are min/max
temperatures (\(T_{min}\) and \(T_{max}\)), min/max relative humidity
(\(RH_{min}\) and \(RH_{max}\)), precipitation (\(P\)) and solar
radiation (\(Rad\)). Wind speed (\(u\)) is also needed for the model,
but the user ma introduce missing values if not available (then, a
default value will be used). Definitions and units of these variables
are given in section \ref{meteoinput}.

\subsection{Control parameters}\label{control-parameters-2}

The control values relevant for the advanced water balance model are:

\begin{itemize}
\tightlist
\item
  \texttt{verbose\ {[}=TRUE{]}}: Whether extra console output is desired
  during simulations.
\item
  \texttt{soilFunctions\ {[}=\ "SX"{]}}: Water retention curve model,
  either ``SX'' (for Saxton) or ``VG'' (for Van Genuchten).
\item
  \texttt{leafPhenology\ {[}=\ TRUE{]}}: Whether leaf phenology is
  simulated.
\item
  \texttt{snowpack\ {[}=\ TRUE{]}}: Whether dynamics of snow pack are
  included.
\item
  \texttt{drainage\ {[}=\ TRUE{]}}: Whether water exceeding the field
  capacity of the bottom layer is lost by deep drainage into a
  non-reachable compartment.
\item
  \texttt{transpirationMode\ {[}=\ "Granier"{]}}: Transpiration model,
  in this case should be ``Sperry''.
\item
  \texttt{defaultWindSpeed\ {[}=\ 2.5{]}}: Default value for wind speed
  (in \(m \cdot s^{-1}\)) when this is missing (only used for leaf
  fall).
\item
  \texttt{hydraulicCostFunction\ {[}=\ 1{]}}: Variant of the hydraulic
  cost function used in the stomatal regulation model of
  \citet{Sperry2016}. Values accepted are 1 (original cost function
  based on the derivative of supply function), 2 (leaf vulnerability
  curve).
\item
  \texttt{ndailysteps\ {[}=24{]}}: Number of daily substeps.
\item
  \texttt{thermalCapacityLAI\ {[}=\ 1000000{]}}: Canopy thermal
  capacitance per LAI unit.
\item
  \texttt{verticalLayerSize\ {[}=\ 100{]}}: The size of vertical layers
  (in cm) for photosynthesis calculation.
\item
  \texttt{cavitationRefill\ {[}=\ TRUE{]}}: If \texttt{FALSE} the model
  operates in a irreversible cavitation mode.
\item
  \texttt{taper\ {[}=\ TRUE{]}}: Whether taper of xylem conduits is
  accounted for when calculating aboveground stem conductance from xylem
  conductivity.
\item
  \texttt{averageFracRhizosphereResistance\ {[}=\ 0.15{]}}: Fraction to
  total continuum (stem+root+rhizosphere) resistance that corresponds to
  rhizosphere (averaged across soil water potential values).
\item
  \texttt{numericParams}: A list with params for numerical approximation
  routines.
\item
  \texttt{Catm\ {[}=386{]}}: Atmospheric CO2 concentration (in micromol
  \(CO_2 \cdot mol^{-1}\) = ppm).
\end{itemize}

\subsection{Species parameters}\label{species-parameters}

Vegetation functional attributes are normally filled for each cohort by
function \texttt{spwbInput()} from species identity. However, different
parameters can be specified for different cohort of the same species if
desired. Basic functional parameters relate to light extinction, water
interception and leaf phenology:

\begin{itemize}
\tightlist
\item
  \(k_{PAR,i}\) {[}\texttt{k}{]}: PAR extinction coefficient.
\item
  \(s_{water, i}\) {[}\texttt{g}{]}: Crown water storage capacity
  (i.e.~depth of water that can be retained by leaves and branches) per
  LAI unit (in \(mmH_2O·LAI^{-1}\)).
\item
  \(S_{GDD,i}\) {[}\texttt{Sgdd}{]}: Growth degree days corresponding to
  leave budburst (in degrees Celsius).
\end{itemize}

A second set of plant functional parameters are related to plant
transpiration and photosynthesis:

\begin{itemize}
\tightlist
\item
  \(g_{wmin,i}\) {[}\texttt{Gwmin}{]}: Minimum leaf water conductance
  (in \(mol·s^{-1}·m^{-2}\)).
\item
  \(g_{wmax,i}\) {[}\texttt{Gwmax}{]}: Maximum leaf water conductance
  (in \(mol·s^{-1}·m^{-2}\)).
\item
  \(k_{rhizo,i,s}\) {[}\texttt{VGrhizo\_kmax}{]}: Maximum hydraulic
  conductance of the rhizosphere for each soil layer.
\item
  \(k_{\max root,i,s}\) {[}\texttt{VCroot\_kmax}{]}: Maximum hydraulic
  conductance of the root xylem for each soil layer.
\item
  \(c_{root,i}\) and \(d_{root,i}\) {[}\texttt{VCroot\_c} and
  \texttt{VCroot\_c}{]}: Parameters of the root xylem vulnerability
  curve.
\item
  \(k_{\max stem,i}\) {[}\texttt{VCstem\_kmax}{]}: Maximum hydraulic
  conductance of the stem xylem.
\item
  \(c_{stem,i}\) and \(d_{stem,i}\) {[}\texttt{VCstem\_c} and
  \texttt{VCstem\_c}{]}: Parameters of the stem xylem vulnerability
  curve.
\item
  \(V298_{max,i}\) {[}\texttt{Vmax298}{]}: Maximum Rubisco carboxylation
  rate at 25ºC (in \(\mu mol CO_2·s^{-1}·m^{-2}\)).
\item
  \(J298_{max,i}\) {[}\texttt{Jmax298}{]}: Maximum rate of electron
  transport at 25ºC (in \(\mu mol e ·s^{-1}·m^{-2}\)).\}
\item
  \(P_{rootdisc,i}\) {[}\texttt{pRootDisc}{]}: Relative conductance of
  roots that leads to hydraulic disconnection from soil.
\item
  \(k_{rhizo,i,s}\) {[}\texttt{VGrhizo\_kmax}{]}: Maximum rhizosphere
  conductance values for each soil layer.
\item
  \(k_{root,i,s}\) {[}\texttt{VCroot\_kmax}{]}: Maximum root xylem
  conductance values for each soil layer.
\end{itemize}

\section{Model output}\label{model-output-1}

Function \texttt{spwb} with \texttt{transpirationMode\ =\ "Sperry"}
returns a list object whose elements are data tables. Each table has
dates as rows and has different output variables:

\begin{itemize}
\tightlist
\item
  \texttt{WaterBalance}: Components of the climatic and soil daily water
  balance (i.e.~net precipitation, infiltration, runoff, plant
  transpiration\ldots{}).
\item
  \texttt{Soil}: Daily variation of soil variables (volume, moisture
  relative to field capacity, water potential) for soil layers.
\item
  \texttt{EnergyBalance}: Components of the atmosphere-canopy-soil
  energy balance, including latent heat, convective heat, conduction
  heat and radiation exchanges.
\item
  \texttt{PlantLAI}: Daily leaf area index for each plant cohort.
\item
  \texttt{PlantTranspiration}: Daily transpiration (in mm) for each
  plant cohort.
\item
  \texttt{PlantPhotosynthesis}: Daily net photosynthesis (in g C·m-2)
  for each plant cohort.
\item
  \texttt{PlantPsi}: Daily water potential of each plant (in MPa).
\item
  \texttt{PlantStress}: Daily stress level suffered by each plant cohort
  (relative whole-plant conductance).
\end{itemize}

\chapter{Subdaily temperature and light
variations}\label{subdaily-temperature-and-light-variations}

\section{Air temperature}\label{air-temperature}

Diurnal above-canopy air temperature (\(T_a\)) variations are modeled
assuming a sinusoidal pattern with \(T_a = T_{\min}\) at sunrise and
\(T_a = (T_{\min}+T_{\max})/2\) at sunset. Air temperature varies
linearly between sunset and sunrise \citep{McMurtrie1990}.

\section{Diffuse and direct
radiation}\label{diffuse-and-direct-radiation}

Daily global radiation (in MJ·m\(^{-2}\)) is assumed to include both
direct and diffuse shortwave radiation (SWR). Using latitude information
and whether is a rainy day, this quantity is partitioned into
instantaneous direct and diffuse SWR and PAR for different daily
substeps. Values of instantaneous direct and diffuse SWR and PAR above
the canopy (i.e. \(I_{beam}\) and \(I_{dif}\)) are calculated using the
methods described in \citet{Spitters1986}, which involve comparing daily
global radiation with daily potential radiation.

\section{Longwave radiation}\label{longwave-radiation}

Longwave radiation (LWR) coming from the atmosphere is calculated
following \citet{Campbell1998}:

\begin{equation}
L_{a} = \epsilon_{a} \cdot \sigma \cdot (T_{a} + 273.16)^{4.0}
\end{equation}

where \(T_{a}\) is air temperature, \(\sigma = 5.67 \cdot 10^{-8.0}\)
W·K\(^{-4}\)·m\(^{-2}\) is the Stephan-Boltzmann constant and
\(\epsilon_{a}\) is the emmissivity of the atmosphere, calculated using:

\begin{eqnarray}
\epsilon_{a} &=& (1 - 0.84 \cdot c) \cdot \epsilon_{ac} + 0.84 \cdot c \\
\epsilon_{ac} &=& 1.72 \cdot \left(\frac{vp_{day}}{T_{a} + 273.16} \right)^{1/7}
\end{eqnarray}

where \(vp_{day}\) is the average daily vapor pressure (in kPa) and
\(c\) is the proportion of clouds.

\chapter{Radiation balance}\label{radiation-balance}

The following subsection detail the calculation of radiation components
of the energy balance equations.

\section{Shortwave radiation absorbed by the
canopy}\label{shortwave-radiation-absorbed-by-the-canopy}

The canopy is divided into vertical layers (whose size is determined by
the control parameter \texttt{verticalLayerSize}), and the expanded and
dead leaf area index of each cohort within each layer is determined. Let
\(n\) be the number of canopy layers. And let
\(LAI_{i,j}^{all} = LAI_{i,j}^{\phi}+LAI_{i,j}^{dead}\) be the leaf area
index of cohort \(i\) in layer \(j\). Furthermore, it is generally
accepted that sunlit and shade leaves need to be treated separately
\citep{DePury1997}. This separation is necessary because photosynthesis
of shade leaves has an essentially linear response to irradiance, while
photosynthesis of leaves in sunflecks is often light saturated and
independent of irradiance.

The average irradiance reaching the top of each canopy layer \(j\) is
calculated separately for direct beam and diffuse radiation:

\begin{eqnarray}
I_{beam,j} &=& (1 - \gamma) \cdot I_{beam} \cdot \exp\left[ \sum_{h=j+1}^{n}{\sum_{i}^{c}{-k_{b,i}\cdot \alpha_i^{0.5}\cdot LAI^{all}_{i,h}}}\right]\\
I_{dif,j} &=& (1 - \gamma) \cdot I_{dif} \cdot \exp\left[ \sum_{h=j+1}^{n}{\sum_{i}^{c}{-k_{d,i}\cdot \alpha_i^{0.5}\cdot LAI^{all}_{i,h}}}\right]
\end{eqnarray}

where \(I_{beam}\) and \(I_{dif}\) are the direct and diffuse
irrradiance at the top of the canopy, \(\gamma\) is the leaf reflectance
(\(\gamma_{PAR} = 0.04\), \(\gamma_{SWR} = 0.05\)), \(k_{b,i}\) is the
extinction coefficient of cohort \(i\) for direct light
(\(k_{b,i} = 0.8\)), \(k_{d,i}\) is the extinction coefficient of cohort
\(i\) for diffuse light (i.e. \(k_{PAR}\) or \(k_{SWR}\)) and
\(\alpha_i\) is the absorbance coefficient (\(\alpha_{i,PAR} = 0.9\),
\(\alpha_{i,SWR} = 0.7\)).

The proportion of sunlit leaves, i.e.~leaves in a canopy layer that the
direct light beams (sunflecks) reach is:

\begin{equation}
f_{SL, j}  = \exp\left( \sum_{k>j}^{n}{\sum_{i}^{c}{-k_{b,i} \cdot LAI^{all}_{i,k}}}\right) \cdot \exp\left( \sum_{i}^{c}{-k_{b,i} \cdot 0.5\cdot LAI^{all}_{i,j}}\right)
\end{equation}

As an example we will consider a canopy of one species of LAI = 2,
divided into ten layers with constant leaf density.

This canopy definition leads to a percentage of the above-canopy
irradiance reaching each layer \citep{Anten2016}. Extinction of direct
radiation also defines the proportion of leaves of each layer that are
affected by sunflecks (i.e.~the proportion of sunlit leaves).

\includegraphics{medfatebook_files/figure-latex/unnamed-chunk-11-1.pdf}

The amount of absorved diffuse radiation per leaf area unit of cohort
\(i\) within a given canopy layer \(j\) is calculated as:

\begin{equation}
I_{dif,i,j} = I_{dif,j} \cdot k_{d,i} \cdot \alpha_i^{0.5} \exp\left[ \sum_{h}^{c}{-k_{d,h}\cdot \alpha_h^{0.5}\cdot 0.5\cdot LAI^{all}_{h,j}}\right]
\end{equation}

The amount of absorved scattered beam radiation per leaf area unit of
cohort \(i\) within a given canopy layer \(j\) is calculated as:

\begin{equation}
% \begin{split}
I_{sca,i,j} = I_{b,j} \cdot k_{b,i} \left( \alpha_i^{0.5}\cdot \exp \left( \sum_{h}^{c}{-k_{b,h}\cdot \alpha_h\cdot 0.5\cdot LAI^{all}_{h,i}}\right) 
-\frac{\alpha_i}{(1-\gamma)}\cdot \exp\left( \sum_{h}^{c}{-k_{b,h}\cdot 0.5\cdot LAI^{all}_{h,i}}\right) \right)
% \end{split}
\end{equation}

Finally, the direct radiation absorbed by a unit of sunlit leaf area of
cohort \(i\) in a canopy layer \(j\) does not depend on the position of
the canopy layer and is:

\begin{equation}
I_{dir,i,j} = I_{beam} \cdot \alpha_i \cdot 0.5/\sin{\beta}
\end{equation}

where \(\beta\) is the solar elevation angle, which changes throughout
the day. The amount of light absorbed by shaded/sunlit foliage of cohort
\(i\) in layer \(j\) per leaf area unit (\(I_{SH,i,j}\) and
\(I_{SL,i,j}\), respectively) is:

\begin{eqnarray}
I_{SH,i,j} &=& I_{dif,i,j} + I_{sca,i,j} \\
I_{SL,i,j} &=& I_{dif,i,j} + I_{sca,i,j} + I_{dir,i,j}
\end{eqnarray}

The total amount of light absorbed by shaded/sunlit foliage of cohort
\(i\) per ground area unit is found by taking into account the
proportion of sunlit foliage:

\begin{eqnarray}
\Phi_{SH,i,j} &=& I_{SH,i,j}\cdot (1 - f_{SL,j}) \cdot LAI^{\phi}_{i,j}\\
\Phi_{SL,i,j} &=& I_{SL,i,j}\cdot f_{SL,j} \cdot LAI^{\phi}_{i,j}
\end{eqnarray}

\section{Shortwave radiation absorbed by the
soil}\label{shortwave-radiation-absorbed-by-the-soil}

The instantaneous shortwave radiation reaching the soil is calculated
separately for direct beam and diffuse radiation:

\begin{eqnarray}
K_{beam, soil} &=&  K_{beam} \cdot \exp\left[ \sum_{h=j+1}^{n}{\sum_{i}^{c}{-k_{b,i}\cdot \alpha_i^{0.5} \cdot LAI^{all}_{i,h}}}\right]\\
K_{dif, soil} &=& K_{dif} \cdot \exp\left[ \sum_{h=j+1}^{n}{\sum_{i}^{c}{-k_{d,i}\cdot LAI^{all}_{i,h}}}\right]
\end{eqnarray}

where \(K_{beam}\) and \(K_{dif}\) are the direct and diffuse
irrradiance at the top of the canopy, \(k_{b,i}\) is the extinction
coefficient of cohort \(i\) for direct light (\(k_{b,i} = 0.8\)) and
\(k_{d,i}\) is the extinction coefficient of cohort \(i\) for diffuse
SWR. From these, the SWR absorbed by the soil is found by:

\begin{equation}
K_{abs,sa} = (1 - \gamma_{SWR, soil})\cdot (K_{beam, soil} + K_{dif, soil})
\end{equation}

where \(\gamma_{SWR, soil} = 0.10\) is the SWR reflectance (10\% albedo)
of the soil. LWR is treated in the same way as diffuse SWR. The
instantaneous LWR reaching the soil is:

\begin{equation}
L_{abs,sa} = (1 - \gamma_{LWR, soil})\cdot L_{a} \cdot \exp\left[ \sum_{h=j+1}^{n}{\sum_{i}^{c}{-k_{LWR}\cdot LAI^{all}_{i,h}}}\right]
\end{equation}

where \(L_a\) is the atmosphere longwave irradiance, \(k_{LWR} = 0.8\)
is the extinction coefficient for LWR and \(\gamma_{LWR, soil} = 0.05\)
is LWR soil reflectance (5\% albedo) of the soil.

\section{Longwave soil-canopy radiation
exchange}\label{longwave-soil-canopy-radiation-exchange}

Longwave radiation (LWR) emmited by a surface at the canopy temperature
\(T_{can}\) would be:

\begin{equation}
L_{em} = 0.95 \cdot \sigma \cdot (T_{can} + 273.16)^{4.0}
\end{equation}

where \(0.95\) is emissivity and
\(\sigma = 5.67 \cdot 10^{-8.0} Wm^{-2}\) is the Stephan-Boltzmann
constant. However, the canopy may only partially cover the soil surface,
so the value must be reduced by the proportion of the canopy layer
actually exchanging energy. We take this as the proportion between
atmospheric and absorbed LWR:

\begin{eqnarray}
p_{exch} &=& \frac{L_{abs,ca}}{L_a} \\
L_{em, c} &=& p_{exch} \cdot L_{em}
\end{eqnarray}

\(L_{em, c}\) is discounted twice from the canopy energy balance: as
losses to the atmosphere and losses to the soil.

LWR emmited by the soil is calculated from the temperature of the first
soil layer:

\begin{equation}
L_{em, s} = 0.95 \cdot \sigma \cdot (T_{soil} + 273.16)^{4.0}
\end{equation}

Again, the canopy only absorbs part of this radiation, the remaining
going to the atmosphere:

\begin{equation}
L_{abs, cs} = p_{exch} \cdot L_{em,s}
\end{equation}

\chapter{Canopy and soil energy
balances}\label{canopy-and-soil-energy-balances}

\section{Convective energy}\label{convective-energy}

Convective energy fluxes between atmosphere and the canopy (\(H_{ca}\))
and between the canopy and the soil (\(H_{cs}\)) are determined as
follows:

\begin{eqnarray}
H_{ca} &=& \frac{\rho_{a} \cdot c_p}{r_{ca}}\cdot (T_{can} - T_a) \\
H_{cs} &=& \frac{\rho_{c} \cdot c_p}{r_{cs}}\cdot (T_{can} - T_{soil})
\end{eqnarray}

where \(\rho_{a}\) and \(\rho_{c}\) are the air density above-canopy and
inside-canopy, respectively, \(c_{p}\) = 1013.86 \(J·kg^{-1}·C^{-1}\) is
the specific heat capacity of the air. \(r_{ca}\) and \(r_{cs}\) are the
atmosphere-canopy and canopy-soil aerodynamic resistances (in
\(s·m^{-1}\)). These, in turn, are calculated using canopy height, total
LAI and above-canopy and below-canopy wind speeds.

\section{Latent heat}\label{latent-heat}

As mentioned above, the model only considers latent heat exchanged from
plant transpiration, neglecting energy fluxes corresponding to
evaporation from the soil and evaporation of rain intercepted by the
canopy. After determining stomatal regulation and transpiration for each
plant cohort, latent heat of transpiration is simply calculated as:

\begin{equation}
LE_{c} = \lambda_{T_{can}} \sum_{i}{E_i}
\end{equation}

where \(\lambda_{T_{can}}\) is the latent heat of vaporization at
temperature \(T_{can}\) (in J·kg\(^{-1}\)) and \(E_i\) is the
instanteous transpiration flux calculated for cohort \(i\).

\section{Canopy capacitance and temperature
changes}\label{canopy-capacitance-and-temperature-changes}

TO BE DONE

\section{Soil temperature changes}\label{soil-temperature-changes}

Instantaneous soil temperature changes on each soil layer depend on the
balance between incoming and outcoming energies (\(G_k\) and
\(G_{k-1}\)):

\begin{equation}
\frac{\delta T_{soil,k}}{\delta t} = \frac{G_k - G_{k-1}}{C_{soil,k} \cdot \Delta z_k}
\end{equation}

where \(\Delta z_k\) is the soil width of layer \(k\) and \(C_{soil,k}\)
is the thermal capacity of layer \(k\), depending on soil moisture and
texture (see function \texttt{soil\_thermalcapacity}).

Energy inflow to the first layer (i.e. \(G_0\)) is the result of the
soil energy balance explained above, while energy transfers between
layers (i.e. \(G_k\)) depend on the soil temperature gradient:

\begin{eqnarray}
G_0 &=& K_{abs,sa} + L_{abs,sa} - L_{em,s} + L_{abs,sc} + H_{cs}\\
G_k &=& \lambda_{soil,k} \cdot \frac{\delta T_{soil,k}}{\delta z}
\end{eqnarray}

where \(\lambda_{soil,k}\) is the thermal conductivity of layer \(k\),
depending on soil moisture and texture (see function
\texttt{soil\_thermalconductivity}). The gradient in the bottom layer is
calculated assuming a temperature of the earth (at 10 m) of 15.5
Celsius.

\section{Bare soil evaporation}\label{bare-soil-evaporation}

Evaporation from the soil surface is the last component of the soil
water balance to be calculated. Soil evaporative demand, i.e.~potential
evaporation from the soil (\(PE_{soil}\); in \(mm\cdot day^{-1}\)), is
calculated using the Penman-Monteith combination equation:

\begin{equation}
PE_{soil} = \frac{1}{\lambda} \cdot \frac{\Delta \cdot R_{n,soil} + D \cdot (\rho \cdot C_p/r_a)}{\Delta + \gamma \cdot (1 + r_{soil}/r_a)}
\end{equation}

where \(D\) is the vapour pressure deficit (in kPa), \(\Delta\) is the
slope of the saturated vapor pressure (in \(Pa \cdot K^{-1}\)),
\(\gamma\) is the psychrometer constant (in \(kPa\cdot K^{-1}\)),
\(\lambda\) is the latent heat vaporization of water (in
\(MJ\cdot kg^{-1}\)) and \(C_p\) is the specific heat of air (in
\(MJ\cdot kg^{-1}\cdot K^{-1}\)). \(r_{soil}\) is the resistance of the
soil surface, set to a constant value (\(r_{soil} = 200\)
\(s\cdot m^{-1}\)). For simplicity, aerodynamic resistance (\(r_a\)) in
the soil is currently set to \(r_a = 208.0/u\) where \(u\) is the input
wind speed.

\chapter{Plant hydraulics}\label{plant-hydraulics}

The supply-loss theory of plant hydraulics, presented by
\citet{Sperry2016} and used in \citet{Sperry2016}, uses the physics of
flow through soil and xylem to quantify how steady-state canopy water
supply declines with drought and ceases by hydraulic failure. The theory
builds on the hydraulic model of \citet{Sperry1998} and can be applied
to different segmentations of the soil-plant continuum. In our case we
considered a network of \((N \times 2 + S + 1)\) resistance elements,
with soil being represented in \(N\) different layers and \(S\)
different stem segments. For each soil layer there is a rhizosphere
element in series with a root xylem element. The \(N\) soil layers are
in parallel up to the root crown. From there there are \(S\) stem xylem
segments and a final leaf segment, all in series. Althougth the model
implements this network representation of the soil-plant continuum,
simpler one-element, two-element and three-element representations will
be used in this document to facilitate understanding.

\section{Vulnerability curves}\label{vulnerability-curves}

Each continuum element has a vulnerability curve that starts at maximum
hydraulic conductance (\(k_{max}\), flow rate per pressure drop) and
monotonically declines as water pressure (\(\Psi\)) becomes more
negative. Vulnerability curves form the basis of hydraulic calculations.

\subsection{Xylem vulnerability
curves}\label{xylem-vulnerability-curves}

Xylem tissues are assigned a two-parameter Weibull function as the
vulnerability curve \(k(\Psi)\):

\begin{equation}
k(\Psi) = k_{max}\cdot e^{-((\Psi/d)^c)}
\label{eq:xylemvulnerability}
\end{equation}

where \(k_{max}\) is the maximum hydraulic conductance (defined as flow
per leaf surface unit and per pressure drop), and \(c\) and \(d\) are
species-specific and tissue-specific parameters. Note that parameter
\(d\) is the water potential (in MPa) at which
\(k(\Psi)/k_{max} = e^{-1} = 0.367\). Parameter \(c\) controls the shape
of the vulnerability curve (`exponential' shape with no threshold has
\(c \leq 1\), sigmoidal threshold has \(c > 1\)).

For example, we define the following parameter values for a stem xylem
(\(k_{s,max}\) and parameters \(c_s\) and \(d_s\) of the vulnerability
curve):

\begin{Shaded}
\begin{Highlighting}[]
\NormalTok{kstemmax =}\StringTok{ }\FloatTok{5.0} \CommentTok{# mmol·m-2·s-1·MPa-1}
\NormalTok{stemc =}\StringTok{ }\DecValTok{3} 
\NormalTok{stemd =}\StringTok{ }\OperatorTok{-}\FloatTok{3.0} \CommentTok{# MPa}
\end{Highlighting}
\end{Shaded}

For root xylem (\(k_{r,max}\)), we may assume a higher conductance
(i.e.~higher efficiency) but also higher vulnerability to cavitation
(defined by parameters \(c_r\) and \(d_r\)):

\begin{Shaded}
\begin{Highlighting}[]
\NormalTok{krootmax =}\StringTok{ }\FloatTok{6.6} \CommentTok{# mmol·m-2·s-1·MPa-1}
\NormalTok{rootc =}\StringTok{ }\DecValTok{2}
\NormalTok{rootd =}\StringTok{ }\OperatorTok{-}\FloatTok{2.5} \CommentTok{#MPa}
\end{Highlighting}
\end{Shaded}

The concept of vulnerability curve can be used to specify the
relationship between pressure and conductance in any portion of the flow
path. Leaf vulnerability curve \(k_l(\Psi)\) can be modelled using the
same equation as for xylem:

\begin{equation}
k_l(\Psi) = k_{l,max}\cdot e^{-((\Psi/d_l)^{c_l})}
\label{eq:leafvulnerability}
\end{equation}

where \(k_{l,max}\) is the leaf maximum hydraulic conductance. Values
defined below specify higher conductance for leaves but also slightly
higher vulnerability:

\begin{Shaded}
\begin{Highlighting}[]
\NormalTok{kleafmax =}\StringTok{ }\DecValTok{10}
\NormalTok{leafc =}\StringTok{ }\DecValTok{2}
\NormalTok{leafd =}\StringTok{ }\OperatorTok{-}\DecValTok{2}
\end{Highlighting}
\end{Shaded}

With these parameter values, the vulnerability curves for root, stem and
leaf are (see \texttt{hydraulics\_xylemConductance()}):

\begin{center}\includegraphics{medfatebook_files/figure-latex/unnamed-chunk-17-1} \end{center}

The dotted line between 0 and \(\Psi_{cav} = -2.5\) MPa indicates the
modification of the stem xylem vulnerability curve when cavitation has
occurred (i.e., previous embolism limits a the maximum conductance
value), as indicated in \citet{Sperry2016}. The corresponding proportion
of conductance loss can be found using the stem vulnerability curve:

\begin{equation}
PLC(\Psi_{cav}) = 1.0 - \frac{k_s(\Psi_{cav})}{k_{s,max}} = 1.0 - e^{-((\Psi/{d_s})^{c_s})}
\end{equation}

\begin{Shaded}
\begin{Highlighting}[]
\FloatTok{1.0} \OperatorTok{-}\StringTok{ }\KeywordTok{exp}\NormalTok{(}\OperatorTok{-}\NormalTok{(}\OperatorTok{-}\FloatTok{2.5}\OperatorTok{/}\NormalTok{stemd)}\OperatorTok{^}\NormalTok{stemc)}
\end{Highlighting}
\end{Shaded}

\begin{verbatim}
## [1] 0.4393754
\end{verbatim}

Although root xylem are more vulnerable to the formation of emboli for a
given potential, it is generally accepted that the less negative
potentials of root xylem compared to the stem lead to cavitation
occurring more often in the stem. The constrain created by cavitation
has an effect on the calculation of the flow rates and derived
quantities (see below).

\subsection{Rhizosphere vulnerability
curve}\label{rhizosphere-vulnerability-curve}

The rhizosphere conductance function \(k_{rh}(\Psi)\) is modeled as a
van Genuchten function \citep{Genuchten1980}:

\begin{eqnarray}
k_{rh}(\Psi) &=& k_{rh,max} \cdot v^{(n-1)/(2\cdot n)} \cdot ((1-v)^{(n-1)/n}-1)^2 \\
v &=& [(\alpha \Psi)^{n}+1]^{-1}
\label{eq:rhizovulnerability}
\end{eqnarray}

where \(k_{rh,max}\) is the maximum rhizosphere conductance, and \(n\)
and \(\alpha\) are texture-specific parameters
\citep{Leij1996, Carsel1988}. These are automatically set by function
\texttt{soil()} when initializing soil objects (see parameters
\texttt{VG\_alpha} and \texttt{VG\_n} in the output of \texttt{soil()}),
but we can use function \texttt{soil\_vanGenuchtenParamsCarsel()} to
derive them from texture types:

\begin{Shaded}
\begin{Highlighting}[]
\NormalTok{textures =}\StringTok{ }\KeywordTok{c}\NormalTok{(}\StringTok{"Sandy loam"}\NormalTok{,}\StringTok{"Silt loam"}\NormalTok{, }\StringTok{"Clay"}\NormalTok{)}
\CommentTok{#Textural parameters}
\CommentTok{#Sandy clay loam }
\NormalTok{p1 =}\StringTok{ }\KeywordTok{soil_vanGenuchtenParamsCarsel}\NormalTok{(textures[}\DecValTok{1}\NormalTok{])}
\NormalTok{p1}
\end{Highlighting}
\end{Shaded}

\begin{verbatim}
##     alpha         n theta_res theta_sat 
##   764.983     1.890     0.065     0.410
\end{verbatim}

\begin{Shaded}
\begin{Highlighting}[]
\NormalTok{alpha1 =}\StringTok{ }\NormalTok{p1[}\DecValTok{1}\NormalTok{]  }
\NormalTok{n1 =}\StringTok{ }\NormalTok{p1[}\DecValTok{2}\NormalTok{]}
\CommentTok{#Silt loam}
\NormalTok{p2 =}\StringTok{ }\KeywordTok{soil_vanGenuchtenParamsCarsel}\NormalTok{(textures[}\DecValTok{2}\NormalTok{])}
\NormalTok{p2}
\end{Highlighting}
\end{Shaded}

\begin{verbatim}
##     alpha         n theta_res theta_sat 
##  203.9955    1.4100    0.0670    0.4500
\end{verbatim}

\begin{Shaded}
\begin{Highlighting}[]
\NormalTok{alpha2 =}\StringTok{  }\NormalTok{p2[}\DecValTok{1}\NormalTok{]}
\NormalTok{n2 =}\StringTok{ }\NormalTok{p2[}\DecValTok{2}\NormalTok{]}
\CommentTok{#Silty clay}
\NormalTok{p3 =}\StringTok{ }\KeywordTok{soil_vanGenuchtenParamsCarsel}\NormalTok{(textures[}\DecValTok{3}\NormalTok{])}
\NormalTok{p3}
\end{Highlighting}
\end{Shaded}

\begin{verbatim}
##     alpha         n theta_res theta_sat 
##  81.59819   1.09000   0.06800   0.38000
\end{verbatim}

\begin{Shaded}
\begin{Highlighting}[]
\NormalTok{alpha3 =}\StringTok{  }\NormalTok{p3[}\DecValTok{1}\NormalTok{]}
\NormalTok{n3 =}\StringTok{ }\NormalTok{p3[}\DecValTok{2}\NormalTok{]}
\end{Highlighting}
\end{Shaded}

We can estimate maximum rhizosphere conductance values assuming that
they account for an average percentage of the resistance (e.g.~15\%)
across the continuum (see functions
\texttt{hydraulics\_averageRhizosphereResistancePercent()} and
\texttt{hydraulics\_findRhizosphereMaximumConductance()}):

\begin{Shaded}
\begin{Highlighting}[]
\NormalTok{percentResistance =}\StringTok{ }\DecValTok{15}
\CommentTok{#Sandy clay loam }
\NormalTok{krmax1 =}\KeywordTok{hydraulics_findRhizosphereMaximumConductance}\NormalTok{(percentResistance, }
\NormalTok{                      n1,alpha1, krootmax, rootc,rootd, kstemmax, stemc, stemd,}
\NormalTok{                      kleafmax, leafc, leafd)}
\NormalTok{krmax1}
\end{Highlighting}
\end{Shaded}

\begin{verbatim}
## [1] 7.375648e+14
\end{verbatim}

\begin{Shaded}
\begin{Highlighting}[]
\CommentTok{#Silt loam}
\NormalTok{krmax2 =}\KeywordTok{hydraulics_findRhizosphereMaximumConductance}\NormalTok{(percentResistance, }
\NormalTok{                      n2,alpha2, krootmax, rootc,rootd, kstemmax, stemc, stemd,}
\NormalTok{                      kleafmax, leafc, leafd)}
\NormalTok{krmax2}
\end{Highlighting}
\end{Shaded}

\begin{verbatim}
## [1] 3420747735
\end{verbatim}

\begin{Shaded}
\begin{Highlighting}[]
\CommentTok{#Silty clay}
\NormalTok{krmax3 =}\KeywordTok{hydraulics_findRhizosphereMaximumConductance}\NormalTok{(percentResistance, }
\NormalTok{                      n3,alpha3, krootmax, rootc,rootd, kstemmax, stemc, stemd,}
\NormalTok{                      kleafmax, leafc, leafd)}
\NormalTok{krmax3}
\end{Highlighting}
\end{Shaded}

\begin{verbatim}
## [1] 36831905
\end{verbatim}

With these parameters, the resulting \(k_{rh}(\Psi)\) functions can be
displayed using the function
\texttt{hydraulics\_vanGenuchtenConductance()}:

\begin{center}\includegraphics{medfatebook_files/figure-latex/unnamed-chunk-21-1} \end{center}

\section{Water content of plant
tissues}\label{water-content-of-plant-tissues}

In \textbf{medfate} the water content of leaves and stems is tracked
explicitly. Following \citet{Martin-StPaul2017}, we consider two sources
of water in plant segments \citep{Tyree1990}. The first comes from the
conduits (tracheids or vessels), which will release water due to
cavitation and may be refilled with water from adjacent living tissue.
The second source of water is formed by more elastic living cells
(i.e.~parenchyma) and can potentially be a large source of water during
relatively high water potentials. This source can be described using the
relative water content of a symplasmic tissue. This storage compartment
has its own water potential and exchanges water with the xylem conduits
according to the difference in water potential.

\subsection{Pressure-volume curves}\label{pressure-volume-curves}

A pressure-volume curve of a tissue relates a water potential against
relative water content (\(RWC\); \(kg H_2O \cdot kg^{-1}H_2O\) at
saturation) in drying tissues. Pressure-volume theory is usually applied
to leaves \citep{Bartlett2012}, but it can also be applied to other
tissues such as sapwood or cambium cells.

For living cells, the relationship between \(\Psi\) and \(RWC\) of the
symplasmic fraction (\(RWC_{sym}\)) is achieved by separating \(\Psi\)
into osmotic (solute) potential (\(\Psi_{S}\)) and the turgor potential
(\(\Psi_{P}\)):

\begin{equation}
\Psi = \Psi_{S} + \Psi_{P}
\end{equation}

The relationship for \(\Psi_{P}\) is:

\begin{equation}
\Psi_{P} = -\pi_0 -\epsilon\cdot (1.0 - RWC_{sym})
\end{equation}

where \(\pi_0\) (MPa) is the osmotic potential at full turgor (i.e.~when
\(RWC_{sym} = 1\)), and \(\epsilon\) is the modulus of elasticity
(i.e.~the slope of the relationship). Assuming constant solute content,
the relationship for \(\Psi_{S}\) is:

\begin{equation}
\Psi_{S} = \frac{-\pi_0}{RWC_{sym}} 
\end{equation}

When \(\Psi \leq \Psi_{tlp}\), the water potential at turgor loss point,
then \(\Psi_{P} = 0\) and \(\Psi = \Psi_{S}\). If \(\Psi > \Psi_{tlp}\)
then the two components are needed. The water potential at turgor loss
point (\(\Psi_{tlp}\)) can be found by \citep{Bartlett2012}:

\begin{equation}
\Psi_{tlp} = \frac{\pi_0 \cdot \epsilon}{\pi_0 + \epsilon}
\end{equation}

As an example, the following figure draws the pressure-volume curve for
a tissue with \(\epsilon = 12\) and \(\pi_0 = -3.0\)MPa:

\begin{center}\includegraphics{medfatebook_files/figure-latex/unnamed-chunk-22-1} \end{center}

To calculate \(RWC_{sym}\) from the water potential of a tissue, the
previous equations need to be combined and, after isolating
\(RWC_{sym}\), a quadratic relationship is obtained.

Apoplastic reservoirs (e.g.~sapwood) consist of inelastic cells that
release their water to the transpiration stream following embolism. As
in \citet{Holtta2009}, we equate the relative water content of the
apoplastic reservoir of a segment (leaves or stem) to the proportion of
maximum conductance in the vulnerability curve:

\begin{equation}
RWC_{apo}(\Psi) = \frac{k(\Psi)}{k_{max}} = e^{-((\Psi/d)^{c})}
\end{equation}

\begin{center}\includegraphics{medfatebook_files/figure-latex/unnamed-chunk-23-1} \end{center}

The average relative water content in a given compartment (\(RWC\)) can
be obtained from \(\Psi_{sym}\) and \(\Psi_{apo}\) by calculating
\(RWC_{sym}(\Psi_{sym})\) and \(RWC_{apo}(\Psi_{apo})\) followed by
assuming a constant apoplastic fraction \(f_{apo}\):

\begin{equation}
RWC = RWC_{apo}(\Psi_{apo}) \cdot f_{apo} + RWC_{sym}(\Psi_{sym}) \cdot (1 - f_{apo})
\end{equation}

\subsection{Water content and live fuel moisture
content}\label{water-content-and-live-fuel-moisture-content}

Pressure-volume curves are useful to determine the moisture content of
live fuel elements (leaves and twigs). Given an average relative water
content of a water compartment, its live fuel moisture content (\(LFMC\)
in \(g H_2O \cdot g^{-1}\) of dry tissue) can be calculated using:

\begin{equation}
LFMC = RWC \cdot \Theta \cdot \frac{\rho_{H_2O}}{\rho} = RWC \cdot LFMC_{max}
\end{equation}

where \(\Theta\) is the tissue porosity (\(cm^3\) of water per \(cm^3\)
of tissue), \(\rho\) is the density of the tissue and \(\rho_{H_2O}\) is
the density of water.

If we know \(RWC_{apo}(\Psi_{apo})\), the relative water content in
conduits, and \(V_{segment}\) (in \(m^3\)), the volume of conducting
tissue (sapwood) in the segment, then the mass of water that is stored
in conduits is:

\begin{equation}
S_{apo}(\Psi_{apo}) = V_{segment} \cdot f_{apo} \cdot RWC_{apo}(\Psi_{apo}) \cdot \rho_{w}
\end{equation}

where \(\rho_{w}\) is the density of water (\(kgH_2O \cdot m^{-3}\)) and
\(f_{apo,s}\) is the volume fraction of apoplastic tissue within
sapwood. Similarly, the amount of water stored in the symplastic tissue
of the segment at any time is:

\begin{equation}
S_{sym}(\Psi_{sym}) = V_{segment} \cdot (1 - f_{apo}) \cdot RWC_{sym}(\Psi_{sym}) \cdot \rho_{w}
\end{equation}

Finally, if we consider that both apoplastic and symplastic tissues are
at the same water potential, the water content in the segment will be:

\begin{equation}
S(\Psi) = V_{segment} \cdot (f_{apo} \cdot RWC_{apo}(\Psi) + (1 - f_{apo}) \cdot RWC_{sym}(\Psi)) \cdot \rho_{w}
\end{equation}

\subsection{Relative water content and
cavitation}\label{relative-water-content-and-cavitation}

In \textbf{medfate} we assume that cavitation in stems is
non-reversible. Within a given sapwood segment, we further assume that
the proportion of conductance loss (\(PLC\)) is related to the relative
water content (\(RWC\)) in its conduit vessels (i.e.~the water volume in
conduit vessels with respect to the maximum water volume):

\begin{equation}
PLC(\Psi_{cav}) = 1 - RWC_{apo}(\Psi_{cav})
\end{equation}

where \(RWC_{apo,s}\) is the function of relative water content for
apoplastic stem tissue. With this assumption one can relate changes in
PLC derived from cavitation with decreases in water content in xylem
conduits. When \(PLC\) increases the associated change in water content
is a source of water can be added to the transpirational stream
\citep{Martin-StPaul2017}.

\section{Supply functions}\label{supply-functions}

The supply function describes the rate of water supply (i.e.~flow) for
transpiration (\(E\)) as a function of pressure. The steady-state flow
rate \(E_i\) through each \(i\) element of the continuum is related to
the flow-induced drop in pressure across that element
(\(\Delta \Psi_i\)) by the integral transform of the element's
vulnerability curve \(k_i(\Psi)\) \citep{Sperry2015}:

\begin{equation}
E_i(\Delta \Psi_i) = \int_{\Psi_{up}}^{\Psi_{down}}{k_i(\Psi) d\Psi}
\label{eq:generalsupply}
\end{equation}

where \(\Psi_{up}\) and \(\Psi_{down}\) are the upstream and downstream
water potential values, respectively. The integral transform assumes
infinite discretization of the flow path. The supply function can be
defined for individual elements of the continuum or for the whole
soil-plant continuum using different topologies. In the following
subsections we illustrate the supply function for different cases.

\subsection{Supply function for single
elements}\label{supply-function-for-single-elements}

In the case of a single stem xylem element the supply function describes
the flow rate as a function of canopy pressure (\(\Psi_{canopy}\)). It
can be calculated by numerical integration or aproximated using an
incomplete gamma function. The shape of the supply function starting at
different root crown water potential values (\(\Psi_{rootcrown}\)) is
(see function \texttt{hydraulics\_EXylem()}):

\begin{center}\includegraphics{medfatebook_files/figure-latex/unnamed-chunk-24-1} \end{center}

Right pane shows the supply functions that are obtained in the case of a
cavitated xylem (i.e.~without refilling), assuming that the minimum
water potential experienced so far was -2.5 MPa. Note the linear part of
the flow rate between \(\Psi_{soil}\) and this limit.

The supply function of the rhizosphere element relates the flow rate to
the pressure inside the roots (\(\Psi_{root}\)). It is calculated by
numerical integration of the van Genuchten function (see function
\texttt{hydraulics\_EVanGenuchten()}), for which we use the analytical
approximation of \citet{VanLier2009}. Here we draw the supply function
for the rhizosphere starting at the four different values of bulk soil
pressure (\(\Psi_{soil}\)) and for the same three texture types:

\begin{center}\includegraphics{medfatebook_files/figure-latex/unnamed-chunk-25-1} \end{center}

The nearly vertical lines indicate that for many values of \(E_i\) the
corresponding drop in water potential through the rhizosphere will be
negligible. Only for increasingly negative soil water potential values
the decrease in water potential through the rhizosphere becomes
relevant. Both in the case of a xylem element or a rhyzosphere element
the derivative \(dE_i/d\Psi\) of the supply function is equal to the
corresponding vulnerability curve.

\subsection{Supply function of two elements in
series}\label{supply-function-of-two-elements-in-series}

Let us describe the soil-plant continuum is represented using \emph{two}
elements in series (rhizosphere + stem xylem). In this case, the supply
function has to be calculated by sequentially using the previous supply
functions. The \(E_i\) is identical for each element and equal to the
canopy \(E\). Since \(\Psi_{soil}\) is known, one first inverts the
supply function of the rhizosphere to find \(\Psi_{root}\) (see function
\texttt{hydraulics\_E2psiVanGenuchten()}) and then inverts the supply
function of the xylem to find \(\Psi_{canopy}\) (see function
\texttt{hydraulics\_E2psiXylem()}). The two operations can be summarized
in a single supply function describing the potential rate of water
supply for transpiration (\(E\)) as function of the canopy xylem
pressure (\(\Psi_{canopy}\)), starting from different bulk soil
(\(\Psi_{soil}\)) values (see function
\texttt{hydraulics\textbackslash{}\_supplyFunctionTwoElements()}):

\begin{center}\includegraphics{medfatebook_files/figure-latex/unnamed-chunk-26-1} \end{center}

The supply function for the whole continuum contains much information.
The \(\Psi\) intercept at \(E=0\) represents the predawn canopy sap
pressure which integrates the rooted soil moisture profile. As \(E\)
increments from zero, the disproportionately greater drop in
\(\Psi_{canopy}\) results from the loss of conductance. As the soil
dries the differences in flow due to soil texture become more apparent.
The derivative of the whole continuum supply function, \(dE/d\Psi\), is
not equal to either of the vulnerability curves and it has to be
obtained numerically. The derivative functions corresponding to the
supply functions shown in the previous figure are:

\begin{center}\includegraphics{medfatebook_files/figure-latex/unnamed-chunk-27-1} \end{center}

The derivative \(dE/d\Psi_{canopy}\) is the conductance if the entire
continuum was exposed to \(\Psi_{canopy}\) \citep{Sperry2015}. It
corresponds to the local loss of hydraulic conductance at the downstream
end of the flow path. It falls towards zero for asymptotic critical
values (\(E_{crit}\)). For a cavitated system \(dE/d\Psi_{canopy}\) can
be rather flat, in accordance with the close to linear part of the
supply function.

\subsection{Supply function of three elements in
series}\label{supply-function-of-three-elements-in-series}

If the soil-plant continuum is represented using \emph{three} elements
in series (rhizosphere + stem xylem + leaf), the resulting overall
conductance and resistance fractions (under wet conditions) are:

\begin{Shaded}
\begin{Highlighting}[]
\NormalTok{rstemmin =}\StringTok{ }\DecValTok{1}\OperatorTok{/}\NormalTok{kstemmax}
\NormalTok{rleafmin =}\StringTok{ }\DecValTok{1}\OperatorTok{/}\NormalTok{kleafmax}

\CommentTok{#Percentages of minimum resistance}
\NormalTok{rvec =}\StringTok{ }\KeywordTok{c}\NormalTok{(rstemmin,rleafmin)}
\DecValTok{100}\OperatorTok{*}\NormalTok{rvec}\OperatorTok{/}\KeywordTok{sum}\NormalTok{(rvec)}
\end{Highlighting}
\end{Shaded}

\begin{verbatim}
## [1] 66.66667 33.33333
\end{verbatim}

\begin{Shaded}
\begin{Highlighting}[]
\CommentTok{#Maximum overall conductance}
\DecValTok{1}\OperatorTok{/}\KeywordTok{sum}\NormalTok{(rvec)}
\end{Highlighting}
\end{Shaded}

\begin{verbatim}
## [1] 3.333333
\end{verbatim}

As before, the supply function has to be calculated by sequentially. The
\(E_i\) is identical for each element. Since \(\Psi_{soil}\) is known,
one first inverts the supply function of the rhizosphere to find
\(\Psi_{root}\) and then inverts the supply function of the xylem to
find \(\Psi_{stem}\). Finally, one inverts the supply function of the
leaf element to find \(\Psi_{leaf}\). As before, the three operations
can be summarized in a single supply function describing the potential
rate of water supply for transpiration (\(E\)) as function of the leaf
pressure (\(\Psi_{leaf}\)), starting from different bulk soil
(\(\Psi_{soil}\)) values (see function
\texttt{hydraulics\_supplyFunctionThreeElements()}):

\begin{center}\includegraphics{medfatebook_files/figure-latex/unnamed-chunk-29-1} \end{center}

Note that overall conductance and the maximum flow of the supply
function are smaller in this case than in the representation using two
elements in series. While the rhizosphere component only adds a
significant resistance when the soil dries, considering the leaf segment
(or a root xylem segment) increases the overall resistance of the
continuum. Higher vulnerability of leaves also makes the curve to
saturate for less negative soil water potentials. The derivative
functions corresponding to the supply functions shown in the previous
figure are (note the highest value being equal to the overall maximum
conductance):

\begin{center}\includegraphics{medfatebook_files/figure-latex/unnamed-chunk-30-1} \end{center}

\subsection{Supply function of a root
system}\label{supply-function-of-a-root-system}

So far we considered supply functions of elements in series, but
resistance elements will be in parallel if soil is represented using
\(N>1\) different layers. For each soil layer there is a rhizosphere
element in series with a root xylem element. The \(N\) soil layers are
in parallel up to the root crown.

Network of \(N\) rhizosphere components and root layers in parallel
there are \(N+1\) unknown pressures: the \(N\) root surface pressures
(\(\Psi_{rootsurf,1},\dots,\Psi_{rootsurf,N}\)) and the root crown
pressure at the downstream junction for all root components
(\(\Psi_{rootcrown}\)). The \(N+1\) unknown pressures are solved, for
each specified total flow value \(E\), using multidimensional
Newton-Raphson on a set of equations for steady-state flow
\citep{Sperry2016a}:

\begin{eqnarray}
   E_{k, rhizosphere}-E_{k,root} &=& 0 \\
   \sum_{k}^{n}{E_{k,root}}-E &=& 0
\end{eqnarray}

where \(E_{k, rhizosphere}\) and \(E_{k, root}\) are supply flows
calculated using the integrals of either van Genuchten or Weibull
function as vulnerability curves, respectively. In the case of
rhizosphere elements, \(\Psi_{up,k}=\Psi_{soil,k}\) and in the case of
root elements \(\Psi_{up,k}=\Psi_{rootsurf,k}\). Solving the
steady-state equations also provides values for flow across each of the
parallel paths \(E_{k, rhizosphere} = E_{k, root}\), which are useful to
conduct water balance operations on each layer.

\begin{figure}

{\centering \includegraphics[width=0.8\linewidth]{hydraulics_rs} 

}

\caption{Schematic representation of hydraulics in a root network}\label{fig:unnamed-chunk-31}
\end{figure}

As an example, we start by defining the water potential of three soil
layers corresponding to four situations (analogously with the soil water
potentials defined above):

\begin{Shaded}
\begin{Highlighting}[]
\NormalTok{ psiSoilLayers1 =}\StringTok{ }\KeywordTok{c}\NormalTok{(}\OperatorTok{-}\FloatTok{0.3}\NormalTok{,}\OperatorTok{-}\FloatTok{0.2}\NormalTok{,}\OperatorTok{-}\FloatTok{0.1}\NormalTok{)}
\NormalTok{ psiSoilLayers2 =}\StringTok{ }\KeywordTok{c}\NormalTok{(}\OperatorTok{-}\FloatTok{1.3}\NormalTok{,}\OperatorTok{-}\FloatTok{1.2}\NormalTok{,}\OperatorTok{-}\FloatTok{1.1}\NormalTok{)}
\NormalTok{ psiSoilLayers3 =}\StringTok{ }\KeywordTok{c}\NormalTok{(}\OperatorTok{-}\FloatTok{2.3}\NormalTok{,}\OperatorTok{-}\FloatTok{2.2}\NormalTok{,}\OperatorTok{-}\FloatTok{2.1}\NormalTok{)}
\NormalTok{ psiSoilLayers4 =}\StringTok{ }\KeywordTok{c}\NormalTok{(}\OperatorTok{-}\FloatTok{3.3}\NormalTok{,}\OperatorTok{-}\FloatTok{3.2}\NormalTok{,}\OperatorTok{-}\FloatTok{3.1}\NormalTok{)}
\end{Highlighting}
\end{Shaded}

In a network of several soil layers, one has to divide the total
rhizosphere and root xylem conductances among layers. Let layer depths
be:

\begin{Shaded}
\begin{Highlighting}[]
\NormalTok{d =}\StringTok{ }\KeywordTok{c}\NormalTok{(}\DecValTok{300}\NormalTok{,}\DecValTok{700}\NormalTok{,}\DecValTok{3000}\NormalTok{) }\CommentTok{#Soil layer widths in mm}
\end{Highlighting}
\end{Shaded}

Now let \(v_1\), \(v_2\) and \(v_3\) be the proportion of fine root
biomass in each soil layer.

\begin{Shaded}
\begin{Highlighting}[]
\NormalTok{Z50 =}\StringTok{ }\DecValTok{200} \CommentTok{#Parameter of LDR root distribution}
\NormalTok{Z95 =}\StringTok{ }\DecValTok{1200} \CommentTok{#Parameter of LDR root distribution}
\NormalTok{v =}\StringTok{ }\KeywordTok{root_ldrDistribution}\NormalTok{(Z50, Z95, d)}
\NormalTok{v}
\end{Highlighting}
\end{Shaded}

\begin{verbatim}
##           [,1]      [,2]       [,3]
## [1,] 0.6652935 0.2749944 0.05971209
\end{verbatim}

In the case of the rhizosphere conductances, we can simply define them
(for each soil texture type) as:

\begin{Shaded}
\begin{Highlighting}[]
\NormalTok{krhizomaxvec1 =}\StringTok{ }\NormalTok{krmax1}\OperatorTok{*}\NormalTok{v}
\NormalTok{krhizomaxvec2 =}\StringTok{ }\NormalTok{krmax2}\OperatorTok{*}\NormalTok{v}
\NormalTok{krhizomaxvec3 =}\StringTok{ }\NormalTok{krmax3}\OperatorTok{*}\NormalTok{v}
\end{Highlighting}
\end{Shaded}

To divide maximum root xylem conductance among soil layers we need
weights inversely proportional to the length of transport distances
\citep{Sperry2016a}. Vertical transport lengths can be calculated from
soil depths and radial spread can be calculated assuming cylinders with
volume proportional to the proportions of fine root biomass. The whole
process can be done using function
\texttt{root\_rootXylemConductanceProportions()}:

\begin{Shaded}
\begin{Highlighting}[]
\NormalTok{weights =}\StringTok{ }\KeywordTok{root_xylemConductanceProportions}\NormalTok{(v, d)}
\NormalTok{weights}
\end{Highlighting}
\end{Shaded}

\begin{verbatim}
## [1] 0.2369724 0.4214326 0.3415950
\end{verbatim}

Transport weights are quite different than the fine root biomass
proportions. This is because radial lengths are largest for the first
(top) layer and vertical lengths are largest for the third (bottom)
layer. The root xylem conductances are (in this case they do not depend
on soil texture):

\begin{Shaded}
\begin{Highlighting}[]
\NormalTok{krootmaxvec =}\StringTok{ }\NormalTok{krootmax}\OperatorTok{*}\NormalTok{weights}
\NormalTok{krootmaxvec}
\end{Highlighting}
\end{Shaded}

\begin{verbatim}
## [1] 1.564018 2.781455 2.254527
\end{verbatim}

Having all these maximum conductances, we can now build the supply
functions for each soil texture and starting from the different soil
water potential configurations (see function
\texttt{hydraulics\_supplyFunctionBelowground()}):

\begin{center}\includegraphics{medfatebook_files/figure-latex/unnamed-chunk-38-1} \end{center}

The derivative of \(dE/d\Psi_{rootcrown}\) for the supply function of
the root system is again obtained numerically. Solving the previous
system of equations provides the water potentials in different points of
the root system. Here we plot them for the results of silt loam texture
and the first and last soil potential vectors defined above:

\begin{center}\includegraphics{medfatebook_files/figure-latex/unnamed-chunk-39-1} \end{center}

Note that when soil is not dry (first situation) pressure drop in the
rhizosphere is negligible, but not the pressure drop in the root xylem.
For drier soils rhizosphere becomes more relevant. We can also plot the
flow rates across each of the parallel paths (again corresponding to the
results of silt loam texture and for the four soil potential vectors):

\begin{center}\includegraphics{medfatebook_files/figure-latex/unnamed-chunk-40-1} \end{center}

Note that the contribution of each soil layer depends on the soil
conditions and the total amount of flow. For a low total flow rate some
layers may have negative flows if their potential is lower than others,
which in a dynamic context will cause hydraulic redistribution of water
among soil layers.

\subsection{Supply function of the soil-plant
continuum}\label{supply-function-of-the-soil-plant-continuum}

We can use a network of \((N \times 2 + S + 1)\) resistance elements to
represent the soil-plant continuum, with soil being represented in \(N\)
different layers. As before, the \(N\) soil layers are in parallel up to
the root crown and each soil layer requires at least a rhizosphere and a
root segment. From the root crown there are \(S\) stem xylem elements
(normally \(S = 1\)) in series and a final leaf element. The whole
hydraulic network is illustrated in the figure below.

To build the supply function for the network, we proceed by calculating
water potentials in the network for each value of flow. For any given
\(E\) value we start by calculating flows and potentials within the root
system. After that, and assuming \(S = 1\), the water potential at the
upper end of the stem (\(\Psi_{stem}\)) is obtained using the inverse of
the stem supply function and setting \(\Psi_{up,k}=\Psi_{rootcrown}\).
If \(S > 1\), this is done for each of the stem segments (thus obtaining
\(\Psi_{stem, 1}\), \(\Psi_{stem, 2}\), \ldots{} \(\Psi_{stem, S}\)),
while using a maximum conductance for segments equal to \(k_{max,s}\)
times \(S\). Leaf water potential (\(\Psi_{leaf}\)) is finaly obtained
using the inverse of the leaf supply function and setting
\(\Psi_{up,k}=\Psi_{stem, S}\) and assuming a steady-state flow \(E\).
The whole supply function \(E(\Psi_{leaf}\)) is obtained repeating these
operations from \(E=0\) to a critical value \(E_{crit}\).

\begin{figure}

{\centering \includegraphics[width=0.8\linewidth]{hydraulics_nocap} 

}

\caption{Schematic representation of hydraulics in a whole-plant network}\label{fig:unnamed-chunk-41}
\end{figure}

The following figure shows network supply functions (with \(S = 1\)) for
each soil texture and starting from the different soil water potential
configurations (see function
\texttt{hydraulics\_supplyFunctionNetwork()}):

\begin{center}\includegraphics{medfatebook_files/figure-latex/unnamed-chunk-42-1} \end{center}

As with previous representations of the soil-plant continuum, the
derivative of \(dE/d\Psi_{leaf}\) for the network topology is obtained
numerically:

\begin{center}\includegraphics{medfatebook_files/figure-latex/unnamed-chunk-43-1} \end{center}

As with the root system, we can know the water potentials in different
points of the continuum. Here we plot them for the results of silt loam
texture and the first and last soil potential vectors defined above:

\begin{center}\includegraphics{medfatebook_files/figure-latex/unnamed-chunk-44-1} \end{center}

\subsection{Supply function with water
compartments}\label{supply-function-with-water-compartments}

So far we assumed there are no other sources of water than the soil. In
this section we consider the effect of additional water from leaf and
stem tissues on the supply function. The supply function of a network
representing the continuum is build for a given state of soil water
potentials. This approach allows determining flows and water potential
for \emph{steady-state} conditions. In a dynamic context one has to
calculate the supply function for all values of \(E\), then to determine
stomatal conductance (see sections devoted to photosynthesis and
stomatal regulation) and finally to decide on a specific value of \(E\)
and the corresponding steady-state configuration. Because light and
temperature conditions change along the day, steady-state stomatal
conductance and instantaneous \(E\) values have to be determined for
subdaily steps (e.g.~1-hour steps) and then flows need to be scaled and
aggregated. At the end of the day, one may substract water from soil
layers and recalculate the supply function for the next day.

When considering water compartments, plant transpiration can be larger
or smaller than the water extracted from the soil. Moreover, the supply
function of a given time step will be dependent on the status of storage
comparments in the previous time step, because this determines how much
water is added or removed to the stemflow due to the effect of water
compartments. For any segment (either a stem segment or the leaf
segment) the flow out of it, \(E_{i,out}\), is not necessarily equal to
the flow entering the segment, \(E_{i, in}\). Thus, instead of eq.
\ref{eq:generalsupply}, the equation governing the flow is (Sperry et
al. 1998; Steppe et al. 2005):

\begin{equation}
E_{i,out} = E_{i, in} - \frac{\Delta S_{i}}{\Delta t} = \int_{\Psi_{up}}^{\Psi_{down}}{k_i(\Psi) d\Psi} - \frac{\Delta S_{i}}{\Delta t}
\end{equation}

\begin{figure}

{\centering \includegraphics[width=0.8\linewidth]{hydraulics_full} 

}

\caption{Schematic representation of hydraulics in a whole-plant network with water compartments}\label{fig:unnamed-chunk-45}
\end{figure}

When building the supply function from the root crown to the leaf, one
has to consider changes in storage water of the current segment before
processing the next segment. Specifically, for every segment \(i\) we
take the current downstream water potential of the previous segment as
upstream water potential (i.e. \(\Psi_{up} = \Psi_{rootcrown, t}\) when
processing the first stem segment, \(\Psi_{up} = \Psi_{i-1, t}\) when
processing intermediate stem segments or \(\Psi_{up} = \Psi_{S, t}\)
when processing the leaf segment). Then we calculate the steady-state
drop in water potential (i.e., determine \(\Psi_{down} = \Psi_{i, t}\))
corresponding to the input flow (\(E_{i,in}\)) using the inverse of the
integral transform. If stem cavitation has occurred previously then
\(\Psi_{cav,i}\) will limit the maximum conductance. To determine
\(E_{i,out}\) we calculate the additional instantaneous flow due to
changes in storage water volume (\(\Delta S_{i}/\Delta t\)) using:

\begin{equation}
\Delta S_{i} = S_i(\Psi_{i, t}) - S_i(\Psi_{i, t-1})
\end{equation}

As mentioned when describing water content functions, we consider two
sources of water in plant segments \citep{Tyree1990}. The amount of
water absorbed by or released from the storage compartment will depend
on the properties of the apoplastic and symplastic tissues and the
fraction of the segment that corresponds to each kind of tissue. For
example, let us take the following storage capacities, fractions of
apoplastic tissue and parameters of the pressure-volume curves for
symplastic tissue.

\begin{Shaded}
\begin{Highlighting}[]
\NormalTok{Vmaxstem =}\StringTok{ }\FloatTok{0.001005046}
\NormalTok{Vmaxleaf =}\StringTok{ }\FloatTok{0.0001327463}
\NormalTok{stemfapo =}\StringTok{ }\FloatTok{0.8}
\NormalTok{leaffapo =}\StringTok{ }\FloatTok{0.15}
\NormalTok{stempi0 =}\StringTok{ }\OperatorTok{-}\DecValTok{2}
\NormalTok{stemeps =}\StringTok{ }\DecValTok{16}
\NormalTok{leafpi0 =}\StringTok{ }\OperatorTok{-}\FloatTok{1.5}
\NormalTok{leafeps =}\StringTok{ }\DecValTok{8}
\end{Highlighting}
\end{Shaded}

The following figures show the change in supply function caused by stem
and leaf water compartments, assuming a 1-hr time step and either 0 MPa
or -1.5 MPa as previous water potential for all segments (see function
\texttt{hydraulics\_supplyFunctionNetworkCapacitance()}):

\begin{center}\includegraphics{medfatebook_files/figure-latex/unnamed-chunk-47-1} \end{center}

For the same water potential drop, the effect of the water compartment
results in a larger transpiration flow. If the previous water potential
is more negative than the root crown, negative flows may occur, because
of the need to replenish the water compartment. Note, in addition, that
\(dE/d\Psi\) is no longer a non-increasing function of \(\Psi\). This
has important consequences for the definition of the cost function (see
subsection `Stomatal regulation'). In our opinion, the effect of water
compartments on the transpiration flow are important for subdaily
variations in transpiration but are less necessary for seasonal to
multi-year simulations. However, we will come back to the importance of
compartments for plants being disconnected from the soil.

\chapter{Plant photosynthesis and stomatal
regulation}\label{plant-photosynthesis-and-stomatal-regulation}

\section{Leaf energy balance, gas exchange and
photosynthesis}\label{leaf-energy-balance-gas-exchange-and-photosynthesis}

\subsection{Leaf VPD, conductance to water vapor and
photosynthesis}\label{leaf-vpd-conductance-to-water-vapor-and-photosynthesis}

The water supply function specifies the flow rate, as per leaf area, for
values of leaf water potential. If we know air temperature, air vapour
pressure and the light conditions in which leaves are, we can be
translate the supply function into a photosynthesis function
\citep{Sperry2016}. In a nutshell, \(E\) from the supply function is
used to calculate leaf temperature from an evaluation of the leaf energy
balance. The diffusive conductances of the leaf to water and \(CO_{2}\)
are obtained from water supply and water vapor deficit. The gross
assimilation rate is then obtained from the diffusive conductance and a
modelled curve between assimilation and leaf internal \(CO_{2}\)
concentration. Gross assimilation is calculated, without subtracting
autotrophic respiration, because the purpose is to represent the
instantaneous gain of opening the stomata. Nevertheless autotrophic
respiration is included when calculating leaf net photosynthesis.

\begin{Shaded}
\begin{Highlighting}[]
\NormalTok{Tmin =}\StringTok{ }\DecValTok{15}
\NormalTok{Tmax =}\StringTok{ }\DecValTok{30}
\NormalTok{RHmin =}\StringTok{ }\DecValTok{60}
\NormalTok{RHmax =}\StringTok{ }\DecValTok{75}
\NormalTok{Tcan =}\StringTok{ }\NormalTok{meteoland}\OperatorTok{::}\KeywordTok{utils_averageDaylightTemperature}\NormalTok{(Tmin, Tmax)}
\NormalTok{vpa =}\StringTok{ }\NormalTok{meteoland}\OperatorTok{::}\KeywordTok{utils_averageDailyVP}\NormalTok{(Tmin, Tmax, RHmin, RHmax)}
\NormalTok{Patm =}\StringTok{ }\NormalTok{meteoland}\OperatorTok{::}\KeywordTok{utils_atmosphericPressure}\NormalTok{(}\DecValTok{100}\NormalTok{)}

\NormalTok{Q =}\StringTok{ }\DecValTok{2000}
\NormalTok{Catm =}\StringTok{ }\DecValTok{386}

\NormalTok{Vmax298 =}\StringTok{ }\DecValTok{100}
\NormalTok{Jmax298 =}\StringTok{ }\FloatTok{1.67}\OperatorTok{*}\NormalTok{Vmax298}
\NormalTok{Gmin =}\StringTok{ }\FloatTok{0.00001}\NormalTok{;}
\NormalTok{Gmax =}\StringTok{ }\FloatTok{0.3}

\NormalTok{Rabs =}\StringTok{ }\DecValTok{740} \CommentTok{#W * m-2}
\end{Highlighting}
\end{Shaded}

\subsection{Leaf temperature and vapor pressure
deficit}\label{leaf-temperature-and-vapor-pressure-deficit}

Leaf temperature (\(T_{leaf}\); in Celsius) can be calculated for any
given flow rate \(E(\Psi_{leaf})\) using \citep{Campbell1998}:

\begin{equation}
T_{leaf}(\Psi_{leaf}) = T_{can}+\frac{I_{abs}-\epsilon\cdot\sigma\cdot(T_{can}+273.15)^4-\lambda_v\cdot E(\Psi_{leaf})}{C_p\cdot(g_r+g_{Ha})}
\end{equation}

where \(I_{abs}\) (in \(W \cdot m^{-2}\)) is the instantaneous shortwave
and longwave radiation absorbed per leaf area unit, \(E(\Psi_{leaf})\)
is the flow (converted to \(mol \cdot s^{-1} \cdot m^{-2}\) per
two-sided leaf area basis), \(\epsilon\) is longwave radiation
emissivity (0.97), \(\sigma\) is the Stephan-Boltzman constant,
\(T_{can}\) is the canopy air temperature (in ºC; see Radiation and
energy balance), \(C_p\) = 29.3 \(J \cdot mol^{-1} \cdot ºC^{-1}\) is
the specific heat capacity of dry air at constant pressure and
\(\lambda_v\) is the latent heat of vaporization (in
\(J \cdot mol^{-1}\)):

\begin{equation}
\lambda_v = (2.5023\cdot 10^6-(2430.54\cdot T_{can}))\cdot 0.018
\end{equation}

Finally, \(g_r\) and \(g_{Ha}\) are the radiative and heat conductance
values (in \(mol \cdot m^{-2} \cdot s^{-1}\)), respectively
\citep{Campbell1998}:

\begin{eqnarray}
g_r &=& \frac{4\cdot \epsilon \cdot \sigma \cdot (T_{can}+273.15)^3}{C_p} \\
g_{Ha} &=& 0.189 \cdot (u/d)^{0.5}
\end{eqnarray}

where \(u\) is wind speed (in \(m \cdot s^{-1}\)), taken as the wind
speed at mid-crown height, and \(d\) is 0.72 times the leaf width
(species parameter \texttt{LeafWidth} in \(cm\)).

The following figures illustrate the value of \(T_{leaf}\) for two leaf
sizes and varying values of wind speed and flow rate, calculated for
24ºC canopy temperature and 740 \(W \cdot m^{-2}\) instantaneous
absorved radiation (see function \texttt{biophysics\_leafTemperature}):

\begin{center}\includegraphics{medfatebook_files/figure-latex/unnamed-chunk-49-1} \end{center}

Let's now fix wind speed to 2 m/s. The application of the above
equations to the \(E(\Psi_{leaf})\) curves corresponding to the complete
hydraulic network yields the following \(T_{leaf}(\Psi_{leaf})\) curves:

\begin{center}\includegraphics{medfatebook_files/figure-latex/unnamed-chunk-50-1} \end{center}

Thus, transpiration decreases leaf temperature (whereas radiation
increases it and wind speed makes it more similar to air temperature).
Vapor pressure deficit in the leaf (\(VPD_{leaf}\), in kPa) is
calculated as:

\begin{equation}
VPD_{leaf} = VP(T_{leaf})-vp_{day}
\end{equation}

Where \(vp_{day}\) is the average daily vapor pressure and \(VP(T)\) is
a function giving the saturated vapor pressure for temperature \(T\).
Let us assume the following values of relative humidity, yielding an
average \(vp_{day}\):

\begin{Shaded}
\begin{Highlighting}[]
\NormalTok{RHmin =}\StringTok{ }\DecValTok{60}
\NormalTok{RHmax =}\StringTok{ }\DecValTok{75}
\NormalTok{vpa =}\StringTok{ }\KeywordTok{utils_averageDailyVP}\NormalTok{(Tmin, Tmax, RHmin, RHmax)}
\NormalTok{vpa}
\end{Highlighting}
\end{Shaded}

\begin{verbatim}
## [1] 1.912181
\end{verbatim}

the application of the above equation to the \(T_{leaf}(\Psi_{leaf})\)
curves yields the following \(VPD_{leaf}(\Psi_{leaf})\) curves:

\begin{center}\includegraphics{medfatebook_files/figure-latex/unnamed-chunk-52-1} \end{center}

Since leaf saturated VP decreases when leaf temperature decreases,
transpiration decreases leaf VPD as a result of decreasing leaf
temperature.

\subsection{Leaf conductance to water
vapor}\label{leaf-conductance-to-water-vapor}

Leaf conductance to water vapor (\(g_{sw}\); in
\(mol H_2O \cdot s^{-1} \cdot m^{-2}\)) and to carbon dioxide
(\(g_{sc}\); in mol \(CO_{2} \cdot s^{-1} \cdot m^{-2}\)) are obtained
for each value of \(E\) (in \(mol \cdot s^{-1} \cdot m^{-2}\)) and
\(VPD_{leaf}\) using:

\begin{eqnarray}
g_{sw} &=& E \cdot \frac{P_{atm}}{VPD_{leaf}}\\
g_{sc} &=& g_{sw}/1.6
\end{eqnarray}

the application of the equation for \(g_{sw}\) to the
\(VPD_{leaf}(\Psi_{leaf})\) curves yields the following
\(g_{sw}(\Psi_{leaf})\) curves:

\begin{center}\includegraphics{medfatebook_files/figure-latex/unnamed-chunk-53-1} \end{center}

Hence, larger values of transpiration require larger values of leaf
water vapour conductance. In the previous figure we have indicated the
thresholds of \(g_{swmin}\) and \(g_{swmax}\), the species-specific
minimum and maximum water vapour conductances (i.e.~conductances when
stomata are fully closed and fully open, respectively; see parameters
\texttt{Gwmin} and \texttt{Gwmax} in \texttt{SpParamsMED}).

\begin{Shaded}
\begin{Highlighting}[]
\NormalTok{Gmin =}\StringTok{ }\FloatTok{0.0045}\NormalTok{;}
\NormalTok{Gmax =}\StringTok{ }\FloatTok{0.3}
\end{Highlighting}
\end{Shaded}

\(g_{sw}\) cannot exceed \(g_{swmax}\) so that some flow rates may not
be possible (see stomatal regulation below). However, \(g_{swmax}\)
should quickly become non-limiting as soil dries (i.e.~reducing \(E\))
or \(VPD_{leaf}\) increases \citep{Sperry2016}. Minimum stomatal
conductance is also used in \textbf{medfate} when building the supply
function, as it specifies the minimum flow rates that will occur for
completely-closed stomata, i.e.~the minimum flow from which supply
function is build.

\subsection{Leaf photosynthesis}\label{leaf-photosynthesis}

Rubisco-limited photosynthesis rate \(A_c\) (in
\(\mu mol CO_2 \cdot s^{-1} \cdot m^{-2}\)) is modelled using
\citep{Collatz1991, Medlyn2002}:

\begin{equation}
A_c=\frac{V_{max}\cdot (C_i- \Gamma*)}{C_i+K_c \cdot (1+ O_a/K_o)}
\end{equation}

where \(V_{max}\) is Rubisco's maximum carboxylation rate (in
\(\mu mol CO_2 \cdot s^{-1} \cdot m^{-2}\)), \(C_i\) is the internal
carbon dioxide concentration (in \(\mu mol \cdot mol^{-1}\)),
\(\Gamma*\) is the compensation point (in \(\mu mol \cdot mol^{-1}\)),
\(K_c\) (in \(\mu mol \cdot mol^{-1}\)) and \(K_o\) (in
\(mmol \cdot mol^{-1}\)) are Michaelis-Menten constants for
carboxylation and oxygenation, respectively, and \(O_a\) is the
atmospheric oxygen concentration (i.e.~209 \(mmol \cdot mol^{-1}\)).
\(\Gamma*\), \(K_c\) and \(K_o\) depend on leaf temperature
(\(T_{leaf}\), in Celsius) \citep{Bernacchi2001}:

\begin{eqnarray}
\Gamma* &=& 42.75\cdot e^{\frac{37830\cdot (T_{leaf}-25)}{298\cdot R \cdot (T_{leaf}-273)}}\\
K_c &=& 404.9\cdot e^{\frac{79430\cdot (T_{leaf}-25)}{298\cdot R \cdot (T_{leaf}-273)}}\\
K_o &=& 278.4\cdot e^{\frac{36380\cdot (T_{leaf}-25)}{298\cdot R \cdot (T_{leaf}-273)}}
\end{eqnarray}

Electron transport-limited photosynthesis \(A_e\) (in
\(\mu mol CO_2 \cdot s^{-1} \cdot m^{-2}\)) was obtained from
\citet{Medlyn2002}:

\begin{eqnarray}
A_e &=& \frac{J}{4}\cdot \frac{C_i-\Gamma*}{C_i+2\cdot \Gamma*} \\
J &=& \frac{(\alpha\cdot Q + J_{max})-\sqrt{(\alpha\cdot Q + J_{max})^2-4.0\cdot c \cdot \alpha \cdot Q \cdot J_{max}}}{2\cdot c}
\end{eqnarray}

where \(\alpha\) is the quantum yield of electron transport (0.3
\(mol electrons \cdot mol photons^{-1}\)), \(Q\) is the PAR photon flux
density (\(\mu mol photons \cdot m^{-2} \cdot s^{-1}\)), which is
calculated from leaf irradiance (\(I_{par}\); in \(W \cdot m^{-2}\)):

\begin{equation}
Q = I_{par}\cdot 546 \cdot 0.836\cdot 10^{-2}
\end{equation}

\(J_{max}\) and \(J\) are the maximum and actual rate of electron
transport (both in \(\mu mol electrons \cdot m^{-2} \cdot s^{-1}\)) and
\(c=0.9\) defines the curvature of the light-response curve. The gross
assimilation rate \(A\) at a given \(C_i\) is the minimum of \(A_e\) and
\(A_c\). To obtain a smooth \(A\)-vs-\(C_i\) curve we used
\citep{Collatz1991}:

\begin{equation}
A = \frac{(A_c+A_e)-\sqrt{(A_c+A_e)^2-4.0\cdot c'\cdot A_e\cdot A_c}}{2\cdot c'}
\end{equation}

where \(c'=0.98\) is a curvature factor. The temperature dependence of
\(J_{max}\) and \(V_{max}\) relative to 25ºC was modelled using
\citet{Leuning2002} (his eq. 1 with parameters from his Table 2). The
internal CO\(_2\) concentration, \(C_i\), needs to be known to calculate
\(A\) using the previous equations. \citet{Sperry2016a} use a second
equation for \(A\) which uses \(g_{cs}\):

\begin{equation}
A = g_{sc} \cdot (C_{atm}-C_i)
\end{equation}

where \(C_{atm}\) is the atmospheric \(CO_{2}\) concentration (in
\(\mu mol \cdot mol^{-1}\); see parameter \texttt{Catm} in function
\texttt{defaultControl()}). Combining the two equations for \(A\) and
finding the root of the resulting equation using Newton-Raphson method
allows determining \(C_i\) and therefore \(A\). Thus, after defining PAR
photon flux density, atmosphere \(CO_{2}\) concentration and maximum
rate parameters:

\begin{Shaded}
\begin{Highlighting}[]
\NormalTok{Q =}\StringTok{ }\DecValTok{2000}
\NormalTok{Catm =}\StringTok{ }\DecValTok{386}
\NormalTok{Vmax298 =}\StringTok{ }\DecValTok{100}
\NormalTok{Jmax298 =}\StringTok{ }\FloatTok{1.67}\OperatorTok{*}\NormalTok{Vmax298}
\end{Highlighting}
\end{Shaded}

one can obtain the following \(A(\Psi_{leaf})\) curves from
\(T_{leaf}(\Psi_{leaf})\) and \(g_{sc}(\Psi_{leaf})\):

\begin{center}\includegraphics{medfatebook_files/figure-latex/unnamed-chunk-56-1} \end{center}

Finally, leaf net photosynthesis (i.e.~accounting for autotrophic
respiration) is calculated as:

\begin{equation}
A_n = A - 0.015 \cdot V_{max}
\end{equation}

\section{Crown photosynthesis}\label{crown-photosynthesis}

In the previous subsection we calculated photosynthesis at the leaf
level. However, the function \(A(\Psi_{leaf})\) can be calculated for a
whole crown. Essentially we need to repeat the calculations of leaf
temperature, leaf VPD, leaf gas conductance and photosynthesis for every
leaf to be considered in the crown. Gross and net photosynthesis values
can be then aggregated across the crown for each value of
\(\Psi_{leaf}\), so that the function \(A(\Psi_{leaf})\) is obtained.
Here we will consider a crown of one species divided into 10 layers,
with constant leaf density:

\begin{Shaded}
\begin{Highlighting}[]
\NormalTok{LAI =}\StringTok{ }\DecValTok{2}
\NormalTok{nlayer =}\StringTok{ }\DecValTok{10}
\NormalTok{LAIlayerlive =}\StringTok{ }\KeywordTok{matrix}\NormalTok{(}\KeywordTok{rep}\NormalTok{(LAI}\OperatorTok{/}\NormalTok{nlayer,nlayer),nlayer,}\DecValTok{1}\NormalTok{)}
\NormalTok{LAIlayermax =}\StringTok{ }\KeywordTok{matrix}\NormalTok{(}\KeywordTok{rep}\NormalTok{(LAI}\OperatorTok{/}\NormalTok{nlayer,nlayer),nlayer,}\DecValTok{1}\NormalTok{)}
\NormalTok{LAIlayerdead =}\StringTok{ }\KeywordTok{matrix}\NormalTok{(}\DecValTok{0}\NormalTok{,nlayer,}\DecValTok{1}\NormalTok{)}
\NormalTok{kb =}\StringTok{ }\FloatTok{0.8}
\NormalTok{kd_PAR =}\StringTok{ }\FloatTok{0.5}
\NormalTok{kd_SWR =}\StringTok{ }\NormalTok{kd_PAR}\OperatorTok{/}\FloatTok{1.35}
\NormalTok{alpha_PAR =}\StringTok{ }\FloatTok{0.9}
\NormalTok{gamma_PAR =}\StringTok{ }\FloatTok{0.04}
\NormalTok{gamma_SWR =}\StringTok{ }\FloatTok{0.05}
\NormalTok{alpha_SWR =}\StringTok{ }\FloatTok{0.7}
\end{Highlighting}
\end{Shaded}

Many aspects may vary across the crown, including environmental
conditions (such as direct/diffuse light or wind speed) and
photosynthesis parameters (e.g. \texttt{Vmax298}). The previous crown
definition and light parameters lead to a percentage of the above-canopy
irradiance reaching each layer \citep{Anten2016}. Furthermore, it is
generally accepted that sunlit and shade leaves need to be treated
separately \citep{DePury1997}.Extinction of direct radiation also
defines the proportion of leaves of each layer that are affected by
direct light beams (i.e.~the proportion of sunlit leaves).

\begin{center}\includegraphics{medfatebook_files/figure-latex/unnamed-chunk-58-1} \end{center}

For simplicity, here we will assume constant windspeed in all layers:

\begin{Shaded}
\begin{Highlighting}[]
\NormalTok{ulayer =}\StringTok{ }\KeywordTok{rep}\NormalTok{(}\DecValTok{2}\NormalTok{, }\DecValTok{10}\NormalTok{)}
\end{Highlighting}
\end{Shaded}

Regarding incoming light, we assume the following direct and diffuse
irradiance at the top of the canopy:

\begin{Shaded}
\begin{Highlighting}[]
\NormalTok{solarElevation =}\StringTok{ }\FloatTok{0.67}
\NormalTok{SWR_direct =}\StringTok{ }\DecValTok{1100}
\NormalTok{SWR_diffuse =}\StringTok{ }\DecValTok{300}
\NormalTok{PAR_direct =}\StringTok{ }\DecValTok{550}
\NormalTok{PAR_diffuse =}\StringTok{ }\DecValTok{150}
\end{Highlighting}
\end{Shaded}

Solar elevation is the angle between the sun and the horizon (i.e.~the
complement of the zenith angle). Under these conditions, the amount of
shortwave and PAR radiation absorbed per unit of leaf area at each
canopy layer is \citep{Anten2016}:

\begin{center}\includegraphics{medfatebook_files/figure-latex/unnamed-chunk-61-1} \end{center}

Following \citet{DePury1997}, we further assume that maximum
assimilation rates are highest for leaves at the top of the canopy and
there is a exponential decrease from there towards the bottom, where
maximum rates are 50\% of those at the top:

\begin{equation}
V_{max,298}(L_i) =V_{max,298}\cdot exp(-0.713\cdot L_i/LAIc)   
\end{equation}

where \(L_i\) is the cumulative LAI value at a given canopy layer \(i\)
and \(LAIc\) is the canopy LAI.

Multilayer canopy models allow evaluating leaf conditions, stomatal
conductance and photosynthesis for different points of the canopy.
However, this comes at high computational cost. While big-leaf canopy
models are known to be unaccurate under some situations, sun-shade
canopy models \citep{DePury1997} provide estimates that are close to
multiple layer models \citep{Hikosaka2016}. Sun-shade models involve:
(a) aggregating the leaf area of sunlit/shade leaves across layers; (b)
aggregating the light absorbed by leaves of each kind across layers; and
(c) aggregating maximum assimilation rates across layers, again
separating sunlit and shade leaves. One then calls the photosynthesis
model twice (i.e.~once for shade leaves and once for sunlit leaves),
using the aggregated maximum assimilation rates. Separating the two
kinds of leaves acknowledges that they operate at different parts of the
light-saturation curve. The following figure provides the canopy
photosynthesis functions obtained, under different situations, using a
full 10-layer canopy description (top), a sunshade canopy model (center)
or a big-leaf model (bottom). These were generated using functions
\texttt{photo\_multilayerPhotosynthesisFunction()},
\texttt{photo\_sunshadePhotosynthesisFunction()} and
\texttt{photo\_leafPhotosynthesisFunction()}, respectively. Note the
coincidence between the multi-layer and the sun-shade models.

\begin{center}\includegraphics{medfatebook_files/figure-latex/unnamed-chunk-64-1} \end{center}

\section{Stomatal regulation}\label{stomatal-regulation}

\citet{Sperry2016} presented a profit maximization approach where
hydraulic costs of opening the stomata are compared against
photosynthetic gain. Details of their approach, and two suggested
variants, a given in the next two subsections. The final subsection
explains how to scale stomatal regulation (and hence, transpiration and
photosynthesis) from leaf to plant.

\subsection{Cost and gain functions}\label{cost-and-gain-functions}

The hydraulic supply function is used to derive a transpirational
\emph{cost function} \(\theta_1(\Psi_{leaf})\) that reflects the
increasing damage from cavitation and the greater difficulty of moving
water along the continuum \citep{Sperry2016a}:

\begin{equation}
\theta_1(\Psi_{leaf}) = \frac{k_{c,max}-k_{c}(\Psi_{leaf})}{k_{c,max}-k_{crit}}
\end{equation}

where \(k_c(\Psi_{leaf}) = dE/d\Psi(\Psi_{leaf})\) is the slope of the
supply function, \(k_{c,max} = dE/d\Psi(\Psi_{soil})\) and
\(k_{crit} = dE/d\Psi(\Psi_{crit})\) is the slope of the supply function
at \(E = E_{crit}\) the critical flow beyond which hydraulic failure
occurs.

Alternatively, we considered a second cost function
(\(\theta_2(\Psi_{leaf})\)) using the vulnerability curve of the leaf:

\begin{eqnarray}
\theta_2(\Psi_{leaf}) &=& \frac{k_{l, max}-k_l(\Psi_{leaf})}{k_{l,max} - k_{l,min}}\\
\end{eqnarray}

where \(k_l\) is the leaf conductance function; and \(k_{l,min}\) and
\(k_{l,max}\) are the minimum and maximum leaf conductance values found
in the supply function. Using the leaf vulnerability curve for the cost
function is grounded on the fact that stomatal regulation occurs at
leaves, so that instantaneous regulation should respond to the loss of
hydraulic conductance at this point, independently of what happens to
the rest of the continuum. Hormonal signals from root to leaf are
assumed to regulate stomatal aperture at longer time scales. Obviously,
\(\theta_2\) is the same before irreversible cavitation. The difference
between them may be interpreted as the following. \(\theta_2\) strictly
follows the potential at the leaf level (and hence could be related to a
loss of turgor).

The type of cost function can be specified by the user by setting
parameter \texttt{hydraulicCostFunction} (see function
\texttt{defaultControl()}). The following figures illustrate the
\(\theta_1\) and \(\theta_2\) curves corresponding to the supply
functions:

\begin{center}\includegraphics{medfatebook_files/figure-latex/unnamed-chunk-66-1} \end{center}

The normalized photosynthetic \emph{gain function}
\(\beta(\Psi_{leaf})\) reflects the actual assimilation rate with
respect to the maximum:

\begin{equation}
\beta(\Psi_{leaf}) = \frac{A(\Psi_{leaf})}{A_{max}}
\end{equation}

where \(A_{max}\) is the instantaneous maximum assimilation rate
estimated over the full \(\Psi_{leaf}\) range. The following figures
illustrate the \(\theta(\Psi_{leaf})\) and \(\beta(\Psi_{leaf})\) curves
corresponding to the supply and assimilation functions:

\begin{center}\includegraphics{medfatebook_files/figure-latex/unnamed-chunk-67-1} \end{center}

\subsection{Profit maximization at the leaf
level}\label{profit-maximization-at-the-leaf-level}

According to \citet{Sperry2016}, stomatal regulation can be effectively
estimated by determining the maximum of the \emph{profit function}
(\(Profit(\Psi_{leaf})\)), for which we consider three alternatives
corresponding to the two cost functions:

\begin{eqnarray}
Profit_1(\Psi_{leaf}) &=& \beta(\Psi_{leaf})-\theta_1(\Psi_{leaf})\\
Profit_2(\Psi_{leaf}) &=& \beta(\Psi_{leaf})-\theta_2(\Psi_{leaf})\\
\end{eqnarray}

Once \(\Psi_{leaf}\) that maximizes profit is determined, the values of
the remaining variables are also determined. At this point, it may
happen that \(g_{sw}(\Psi_{leaf})\) is lower than the minimum
(i.e.~cuticular) water vapor conductance (\(g_{swmin}\)) or larger than
the maximum water vapor conductance (\(g_{swmax}\)). These thresholds
need to be taken into account when determining the maximum of the profit
function. The following figures illustrate the \(Profit_1(\Psi_{leaf})\)
and \(Profit_2(\Psi_{leaf})\) curves of corresponding to the previous
cost and gain curves: \textbackslash{}begin\{center\}

\begin{center}\includegraphics{medfatebook_files/figure-latex/unnamed-chunk-68-1} \end{center}

Squares in the previous figures indicate the maximum profit points in
each situation. In the case of non-cavitated system (left panels), the
drier the soil, the closer is the maximum profit \(\Psi_{leaf}\) to soil
water potential as one would expect intuitively. This occurs for all
three profit functions. Unlike \(\theta_1\) which is different for each
soil texture (and soil potential), \(\theta_2\) is the same for all soil
textures. As a result, the regulation points do not differ much among
textures in \(Profit_2\) and \(Profit_3\) because the only difference is
in the gain function. For a system with xylem cavitation (right panel),
the maximum \(Profit_1\) curves behave strangely. In particular may get
a more negative value for \(\Psi_{canopy}\) for wet soils than for dry
soils. This effect does not occur when using \(Profit_2\). \(Profit_2\)
brings plant water potentials to more negative values after cavitation.
Although cavitation did not change the \(\theta_2\) function, the supply
function is flatter and this affects the gain function, making it
increase less steeply with lower potentials.

Differences between profit functions can be more easily seen when
plotting the change from original (uncavitated) regulation to the
cavitated one, in terms of both canopy sap pressure and flow rate:

\begin{center}\includegraphics{medfatebook_files/figure-latex/unnamed-chunk-69-1} \end{center}

In \(Profit_1\) irreversible cavitation often brings, after soil
rewetting, less conservative stomatal regulation that enables higher
flow rates. This does not seem to happen in \(Profit_2\), where despite
irreversible cavitation leads to more negative water potentials,
predicted flow rates after rewetting are not above those predicted
before cavitation.

\subsection{Scaling to the plant
level}\label{scaling-to-the-plant-level}

So far, we have considered stomatal regulation by at the leaf level
only. At the plant level, the gain function could be build from the
crown photosynthesis function \(A(\Psi_{leaf})\) that we defined in
subsection `Crown photosynthesis'. However, using the crown
photosynthesis function would imply the assumption that the same
stomatal aperture occurs in all leaves of the crown, independently of
whether they are in shade or sunlit. A more realistic approach is to
determine stomatal regulation by profit maximization for sunlit and
shade leaves separately; and then determining the average photosynthesis
and flow rate from the leaf area of each leaf type. The gain function
and profit maximization calculations conducted for each leaf type yield
instantaneous water potentials \(\Psi_{sunlit}\) and \(\Psi_{shade}\).
They also yield flow values \(E_{shade}\) and \(E_{sunlit}\), in
\(mmol H_2O \cdot s^{-1} \cdot m^{-2}\) of leaf area unit. The average
flow rate in \(mmol H_2O \cdot s^{-1} \cdot m^{-2}\) per leaf area unit
at the plant level is the weighed average:

\begin{equation}
 E_{plant} = \frac{E_{sunlit} \cdot LAI_{sunlit} + E_{shade} \cdot LAI_{shade}}{LAI_{sunlit} + LAI_{shade}}
\end{equation}

where \(LAI_{sunlit}\) and \(LAI_{shade}\) are the cohorts LAI values
for sunlit and shade leaves, respectively. Net photosynthesis per leaf
area of sunlit and shade leaves (i.e. \(A_{n,sunlit}\) and
\(A_{n,shade}\)) is aggregated similarly:

\begin{equation}
 A_{n, plant} = \frac{A_{n,sunlit} \cdot LAI_{sunlit} + A_{n,shade} \cdot LAI_{shade}}{LAI_{sunlit} + LAI_{shade}}
\end{equation}

Profit maximization calculations for shade and sunlit leaves imply
different amount of water extracted from the soil layers and different
plant water potentials. To overcome this issue, we must use the
hydraulic supply function to find the extraction flows from soil layers,
the water potential at the root crown and the `average' water potential
of the crown all corresponding to the average flow \(E_{plant}\).

\chapter{Transpiration and photosynthesis under Sperry's
model}\label{transpirationsperry}

The model determines transpiration and photosynthesis for each plant
cohort separately as follows. First, it updates the hydraulic supply
function depending on plant hydraulic characteristics and soil moisture
status. Then, transpiration of the plant cohort is estimated for each of
them following the framework of \citet{Sperry2016}, who suggest
estimating stomatal conductance from the instantaneous maximization of
profit, defined as the difference between photosynthesis gain and
hydraulic cost (both normalized for comparability). Since the continuum
representation implies several soil layers in parallel but joining at
the root crown, the hydraulic submodel yields instantaneous water flow
and carbon assimilation rates from (or to) each soil layer. Finally, the
instantaneous transpiration and assimilation rates of each time step are
scaled to the duration of the time step and to the leaf area of the
plant cohort. The following provides details for these processes (see
further details in Appendices).

\section{Water supply function}\label{water-supply-function}

The supply-loss theory of plant hydraulics of \citet{Sperry2015} uses
the physics of flow through soil and xylem to quantify how canopy water
supply declines with drought and ceases with hydraulic failure. The
theory can be applied to different networks representing the soil-plant
continuum, but in our case the continuum is represented using a network
of \((N \times 2 + 2)\) resistance elements, with soil being represented
in \(N\) different layers. For each soil layer there is a rhizosphere
element in series with a root xylem element. The \(N\) different layers
are in parallel up to the root crown. From there there is a stem xylem
element and a final leaf element.

The \emph{supply function} describes the rate of water supply
(i.e.~flow) for transpiration (\(E\)) as a function of the pressure drop
between the soil and the leaf, and incorporates both soil, xylem and
leaf hydraulic constrains \citep{Sperry1998, Sperry2015, Sperry2016a}.
Assuming that maximum conductance values are in mmol
\(H_2O \cdot s^{-1} \cdot m^{-2}\) per leaf area unit, transpiration
rate (\(E(\Psi_{leaf})\); in mmol \(H_2O \cdot s^{-1} \cdot m^{-2}\) per
leaf area unit) is a function of leaf water potential (\(\Psi_{leaf}\);
in MPa). The supply function for the whole continuum contains much
information. The \(\Psi\) intercept at \(E=0\) represents the predawn
canopy sap pressure which integrates the rooted soil moisture profile.
As \(E\) increments from zero, the disproportionately greater drop in
\(\Psi_{leaf}\) results from the loss of conductance. As the soil dries
the differences in flow due to soil texture become more apparent. More
details of the calculation of the supply function are given in
Appendices.

\section{Leaf energy balance}\label{leaf-energy-balance}

Leaf temperature (\(T_{leaf}\); in Celsius) can be calculated for any
given flow rate \(E(\Psi_{leaf})\) using \citep{Campbell1998}:

\begin{equation}
T_{leaf}(\Psi_{leaf}) = T_{can}+\frac{I_{abs}-\epsilon\cdot\sigma\cdot(T_{can}+273.15)^4-\lambda_v\cdot E(\Psi_{leaf})}{C_p\cdot(g_r+g_{Ha})}
\end{equation}

where \(I_{abs}\) (in \(W \cdot m^{-2}\)) is the instantaneous shortwave
and longwave radiation absorbed per leaf area unit, \(E(\Psi_{leaf})\)
is the flow (converted to \(mol \cdot s^{-1} \cdot m^{-2}\) per
two-sided leaf area basis), \(\epsilon\) is longwave radiation
emissivity (0.97), \(\sigma\) is the Stephan-Boltzman constant,
\(T_{can}\) is the canopy air temperature (in ºC; see
`\texttt{Complex model: Radiation and energy balance}'), \(C_p\) = 29.3
\(J \cdot mol^{-1}\cdot ºC^{-1}\) is the specific heat capacity of dry
air at constant pressure and \(\lambda_v\) is the latent heat of
vaporization (in \(J \cdot mol^{-1}\)):

\begin{equation}
\lambda_v = (2.5023\cdot 10^6-(2430.54\cdot T_{can}))\cdot 0.018
\end{equation}

Finally, \(g_r\) and \(g_{Ha}\) are the radiative and heat conductance
values (in \(mol·m^{-2}·s^{-1}\)), respectively (Campbell and Norman
1998):

\begin{eqnarray}
g_r &=& \frac{4\cdot \epsilon \cdot \sigma \cdot (T_{can}+273.15)^3}{C_p} \\
g_{Ha} &=& 0.189 \cdot (u/d)^{0.5}
\end{eqnarray}

where \(u\) is wind speed (in \(m·s^{-1}\)), taken as the wind speed at
mid-crown height, and \(d\) is 0.72 times the leaf width (species
parameter \texttt{LeafWidth} in \(cm\)).

\section{Leaf photosynthesis
functions}\label{leaf-photosynthesis-functions}

Each water supply value implies an energy balance at the leaf level and
a degree of stomatal openness, which ultimately leads to a particular
value of leaf photosynthesis. At this point, the model has not decided
the amount of water transpired. Therefore, it determines curves
depending on leaf water potential for several parameters, as done for
\(E(\Psi_{leaf})\). More specifically, for each \(\Psi_{leaf}\) value,
the model calculates the corresponding leaf temperature (\(T_{leaf}\);
in ºC), leaf-to-air vapor pressure deficit (\(VPD_{leaf}\); in kPa),
leaf water vapor conductance (\(g_{sw}\); in \(mol H_2O·s^{-1}·m^{-2}\))
and, finally the leaf gross and net (i.e.~after accounting for
autotrophic respiration) photosynthesis assimilation rates (\(A_g\) and
\(A_n\); both in \(\mu mol CO_2·s^{-1}·m^{-2}\)). More details of the
calculation of these functions are given in Appendices.

Since the model deals with canopies and not single leaves, different
parts of the crowns of plant cohorts may be in different canopy
positions, which leads to differences in radiation and leaf energy
balance. Moreover radiation, energy balance and photosynthesis of leaves
vary through the day. Therefore, calculating photosynthesis at the
canopy level requires dividing the canopy into \(c\) layers, while
differentiating between \textbf{sunlit} and \textbf{shade} leaves.
Photosynthesis is calculated by separately for sunlit/shade leaves (De
Pury and Farquhar 1997). For each time step, the leaf temperature, leaf
VPD and leaf water vapor conductance functions are determined separately
for the different leaves.

\section{Stomatal regulation}\label{stomatal-regulation-1}

Plants must open their stomata to acquire CO\_2 and perform
photosynthesis, but doing so promotes water loss. This trade-off has
resulted in a tight coordination between capacity to supply and
transpire water (hydraulic conductance and diffusive conductance to
water vapor) and the maximum capacity for photosynthesis (carboxylation
rate and electron-transport rate). For modelling purposes, this
carbon-for-water trade-off means that hydraulics, stomatal conductance,
transpiration and photosynthesis need to be estimated simultaneously.
Here we adopt the framework of \citet{Sperry2016}, who suggest
estimating stomatal conductance from the instantaneous maximization of
profit, defined as the difference between photosynthesis gain and
hydraulic cost (both normalized for comparability).

Stomatal regulation and plant transpiration are determined for each time
step separately. The model transforms the slope of the hydraulic supply
function into a \textbf{cost function} and the cohort's gross
photosynthesis function into a \textbf{gain function}. Then, it finds
the \(\Psi_{leaf}\) that maximizes the difference between gain and cost.
This simultaneously determines \(E\) and \(A_n\) at the plant cohort
level (and also \(T_{leaf}\), \(VPD_{leaf}\), \(g_{sw}\) and \(A_{n}\)
for each sunlit/shade leaf in the canopy). The details of all these
calculations can be found in Appendices.

While the cost function is the same for the whole day, the gain function
and profit maximization calculations are conducted for each of the time
steps, yielding instantaneous flow values \(E_{t, s}\) for each soil
layer \(s\), in mmol H\(_2\)O·s\(^{-1}\)·m\(^{-2}\) of leaf area unit
and instantaneous net assimilation values \(A_{n,t}\) in \(\mu\)mol
C·s\(^{-1}\)·m\(^{-2}\) of ground area (i.e.~at the cohort level). To
obtain daily values of transpiration at the cohort level the
instantaneous flow rates \(E_{t, s}\) need to be scaled to
\(E_{step,s}\) using:

\begin{equation}
E_{step,s} = E_{t,s}\cdot 10^{-3} \cdot 0.01802 \cdot LAI_i^{\phi}\cdot \Delta t
\end{equation}

where \$ 0.01802\$ is the molar weight (in kg = L = mm) of water,
\(LAI_i^{\phi}\) is the leaf area index of plant cohort \(i\) and
\(\Delta t = \tau_{day}/n_t\), being \(n_t\) the number of time steps.
The flow rates \(E_{step,s}\) of all steps are added to yield \(E_s\)
(in mm H\(_2\)O·day\(^{-1}\)):

\begin{equation}
E_{s} = \sum_{n=1}^{n_t} {E_{step,s}}
\end{equation}

and substracted from the water content of the corresponding soil layer.
Daily values of net carbon assimilation for plant cohorts are obtained
similarly. The instantaneous rates \(A_{n, t}\) are scaled to
\(A_{n,step}\) using:

\begin{equation}
A_{n,step} = A_{n, t} \cdot 10^{-6} \cdot 12.01017 \cdot \Delta t
\end{equation}

where \(12.01017\) is the molar weight of carbon (in g). \(A_{n, step}\)
values of all steps are added to obtain \(A_n\), the daily net
assimilation at the cohort level (in g C·m\(^{-2}\)·day\(^{-1}\)):

\begin{equation}
A_{n,step} = \sum_{n=1}^{n_t} {A_{n,step}}
\end{equation}

\section{Plant drought stress and water
potentials}\label{plant-drought-stress-and-water-potentials}

Because the model determines optimum transpiration for each time step,
this leads to a daily sequence of leaf water potential
(\(\Psi_{leaf,t}\)) and root crown water potential
(\(\Psi_{rootcrown,t}\)) values. The model chooses as the leaf water
potential of the day for cohort \(i\) (\(\Psi_{leaf,i}\)) the minimum of
\(\Psi_{leaf,t}\) values. Analogously, the model chooses as the root
crown water potential of the day for cohort \(i\)
(\(\Psi_{rootcrown,i}\)) the minimum of \(\Psi_{rootcrown,t}\) values.
They represent water potential values that would occur at mid-day.
Unlike under the simple transpiration mode, here there is no need to
average water potentials under the Sperry transpiration mode, because
the differences in potential of soil layers are already integrated in
the hydraulic supply function.

In order to have an estimate of daily drought stress for the plant
cohort, the model uses the stem vulnerability curve of the plant to find
the conductance relative to maximum stem conductance and turns it into
its complement:

\begin{equation}
DDS_i = \phi_i \cdot \left( 1.0 - \frac{k_{stem, i}(\Psi_{rootcrown,i})}{k_{\max stem, i}}\right) = \phi_i \cdot \left(1.0 - e^{-(\Psi_{rootcrown,i}/d_{stem})^{c_{stem}}}\right)
\end{equation}

where \(\phi_i\) is the leaf phenological status. Note the use of
\(\Psi_{rootcrown,i}\) (and not \(\Psi_{leaf,i}\)) to determine drought
stress index. Thus the model tracks the degree of conductance decrease
at the beginning of the stem as a measure of drought stress. This choice
makes daily drought stress values of the Simple and Complex
transpiration modes more comparable (because leaf mid-day water
potentials are usually much more negative than soil water potentials)
and is a sensible choice if one wants to run the model in irreversible
cavitation mode (see below).

\section{Irreversible cavitation and hydraulic
disconnection}\label{irreversible-cavitation-and-hydraulic-disconnection-1}

Like with the `Granier' transpiration mode, the water balance model with
`Sperry' transpiration mode is normally run assuming that although soil
drought may reduce transpiration, embolized xylem conduits are
automatically refilled when soil moisture recovers. When setting
\texttt{cavitationRefill = FALSE} the model tracks the maximum value of
drought stress as before:

\begin{equation}
P_{embolized,i}= \max \{P_{embolized,i}, DDS_i \}
\end{equation}

However, the way that previous cavitation levels are taken into account
differs from the `Simple' transpiration mode. In this mode, the stem
xylem vulnerability curve is modified by specifying that the maximum
conductance value is reduced and set to
\(k_{stem,i} \cdot (1.0 - P_{embolized,i})\). This effectively causes
the supply function to reach lower flow values for the same water
potential drop (see details in Appendices).

When running the model using the `Complex' transpiration mode plants may
be allowed to disconnect from the soil when its potential becomes too
negative. Parameter \(P_{rootdisc,i}\) can be used to specify the
minimum relative conductance value that the plant will tolerate without
disconnecting hydraulically from the soil (in normal simulations
\(P_{rootdisc,i} = 0\)). Again, this affects the model in a way slightly
different than when running the model in `Simple' transpiration mode.
Before building the supply function, the model checks if there are
layers where the relative conductance of roots (i.e.
\(k_{root, i, s}(\Psi_{s})/k_{\max root, i, s}\)) is lower than
\(P_{rootdisc,i}\). Those layers where this happens are not considered
in the calculation of the supply function and do not contribute to
transpiration or to the determination of plant water potentials.

\section{Transpiration and photosynthesis after soil
disconnection}\label{transpiration-and-photosynthesis-after-soil-disconnection}

Considering water compartments allows tracking leaf and stem
(i.e.~plant) \emph{dessication}, either when the plant is connected to
the soil or when transpiration flow comes from the plant itself. In
\textbf{medfate}, disconnection from a given soil layer occurs when its
water potential is too low with respect to the root xylem vulnerability
curve. The user can control this behaviour by specifying a \(p_{root}\)
threshold for relative conductance, and the plant will be disconnected
from a soil layer if \(k_{r}(\Psi_{soil})/k_{r,max} < p_{root}\). An
interesting situation arises when a plant is disconnected from all soil
layers. In this case, the steady-state calculations cannot be used to
determine flows, so one is forced to use a full discrete time
approximation, with compartments and flows indicated in the figure
below.

\begin{figure}

{\centering \includegraphics[width=0.8\linewidth]{hydraulics_disc} 

}

\caption{Schematic representation of hydraulics in a whole-plant network after disconnection from soil}\label{fig:unnamed-chunk-70}
\end{figure}

As before, each time step \(\Delta t\) is divided into \(m\) substeps
and instantaneous lateral flows between symplastic and apoplastic
compartments are calculated as before. In this case, however, one needs
to calculate instantaneous flows between stem apoplastic compartments
using:

\begin{equation}
F_{i, i+1} = k_{s}(\min(\Psi_{apo,i}, \Psi_{cav, i})) \cdot (\Psi_{apo, i} - \Psi_{apo, i+1})
\end{equation}

and between the last stem compartment and the leaf using:

\begin{equation}
F_{S, l} = k_{l}(\Psi_{apo, l}) \cdot (\Psi_{apo, S} - \Psi_{apo, l})
\end{equation}

The flow from the leaf to the atmosphere is dictated by the minimum
stomatal conductance and the vapour pressure deficit. For each time
substep all flows are calculated and water content of compartments is
updated. Then the equations of relative water content are inversed to
find the water potentials for the next time substep.

The instantaneous flow rate between symplastic and apoplastic
compartments can be approximated using a lateral conductance \(k_{lat}\)
and the difference in water potentials. For the leaf we have:

\begin{equation}
F_{lat, l} = k_{lat} \cdot (\Psi_{sym, l} - \Psi_{apo,l})
\end{equation}

where \(\Psi_{sym, l}\) is the water potential of the symplastic leaf
compartment and \(\Psi_{apo,l}\) is the water potential in the
apoplastic leaf compartment. If \(\Psi_{sym, l} > \Psi_l\) then the flow
will be positive towards the leaf apoplasm and if
\(\Psi_{sym, l} < \Psi_l\) the flow will instead refill the leaf
symplastic tissue. An analogous equation can be used to calculate the
instantaneous lateral flow between symplastic and apoplastic
compartments in any stem segment:

\begin{equation}
F_{lat, i} = k_{lat} \cdot (\Psi_{sym, i} - \Psi_{apo, i})
\end{equation}

\part{Forest growth
modelling}\label{part-forest-growth-modelling}

\chapter{Forest growth model}\label{forest-growth-model}

\section{Design principles}\label{design-principles-2}

The physical structure of the stand is represented in one (vertical)
dimension. Height (or depth) is the only dimension that matters
(i.e.~the coordinates of plants are not explicit). The model is
cohort-based, meaning that similar plant individuals are represented
using a single entity with average properties (e.g.~tree height or
diameter) and a density variable is used to scale from individual level
to the cohort level. Processes are implemented either at the
cohort-level (water balance and photosynthesis) or at the individual
level (carbon balance and growth). The model has been designed to be
executed on forest inventory plots, but it can be run on other kind of
vegetation (e.g.~shrublands or crops) provided vegetation is described
using the appropriate variables (i.e.~diameter and height for trees,
percent cover and height for shrubs). The model tries to reproduce the
physiological processes that modulate leaf area changes and plant growth
rates. Nevertheless, since the model does not implement all processes
that may affect growth (such as nutrient availability), maximum growth
rates and maximum plant sizes are constrained from user inputs, to
ensure that model can be more easily calibrated and validated with
observations. Consequently, we believe the model is suited to study
variations of plant growth derived from environmental conditions and
competition for light and water.

Leaf area of each plant cohort is divided between live (wether in
resistance buds or unfolded leaves) and dead (standing dead trees).
Expanded leaf area corresponds to the portion of live leaf area that is
unfolded at any given moment through the leaf phenological cycle. Leaf
area density of individuals is considered constant across the crown.
Sapwood area of individuals is another important state variable. The
Pipe model \citep{Shinozaki1964} is adopted to link increments of leaf
area, sapwood area and fine root biomass. Ratios of leaf area to sapwood
area (Huber value) can vary within species, due to environmental
conditions \citep{Mencuccini1995}. The model assumes a constant,
species-specific Huber value, but allows deviations from the pipe model
caused by drought-related leaf area reductions.

Water fluxes, soil water balance and plant photosynthesis processes
follow the design of the soil water balance model and this part of the
model design will not be repeated here. Plant respiration is calculated
at the individual level, by estimating the respiration of leaves, stem
and fine root compartments. While fine root respiration is proportional
to leaf respiration, and hence to expanded leaf area, stem respiration
depends on plant size.

Growth is determined taking into account environmental limitations on
both source (i.e.~carbon assimilation) and sink (i.e.~carbon investment
on plant tissue expansion)
\citep{Fatichi2014, Guillemot2015, Korner2015}. With respect to carbon
availability for growth, the model offers three alternatives. In the
first one, carbon available for growth is simply the daily difference
between net photosynthesis and maintenance respiration (i.e.~no carbon
storage). The second alternatives involves a single (fast) carbon
storage pool that allows decoupling assimilation from growth. Finally,
the third alternative involves two carbon storage pools (`fast' and
`slow') with a transfer rate between them
\citep{Richardson2013, Dietze2014}, which we assume to be regulated by
the need to maintain, as much as possible, a minimum amount of carbon in
the fast pool (i.e.~for metabolic and osmotic purposes). In the second
and third modes, maximum overall C storage capacity is proportional to
plant size.

The LPG model \citep{Sitch2003} applies different turnover rates for
different tissues, but then tries to satisfy the pipe model
\citep{Shinozaki1964} by allocating C where it is more limiting.
Instead, we assume that baseline leaf and fine root turnover rates are
linearly related to conversion from sapwood to heartwood. Similarly to
3-PG \citep{Landsberg1997} we assume that the turnover rate is smallest
for young plants, and it increases up to a maximum value. The model
assumes that plants cannot suffer from cavitation if the leaf water
potential is large enough for growth to occur (i.e.~if cell turgidity is
large enough for cell elongation). Similarly, it also assumes that
growth stops before cavitation starts during drought events. When
sapwood area reduction occurs, this not only reduces leaf area, but also
decreases the amount of fast C reserves available for future growth.
Thus it is assumed that parts of the plant are effectively disconnected.

Tree structural variables are updated as in forest gap models
\citep{Lindner1997}.

\section{State variables}\label{state-variables-2}

\section{Process scheduling}\label{process-scheduling-2}

Every day the growth model first updates the expanded leaf area of
living plants according to the phenology of species and the day of the
year. Then the model performs soil water balance, transpiration and
photosynthesis calculations by calling the soil water balance submodel.
After dealing with water fluxes and photosynthesis, the model determines
the amount of respiratory biomass, the maximum storage value per
individual and maintenance respiration at the individual level. The
comparison between photosynthesis and respiration leads to an amount of
carbon available for growth (if no carbon pools are considered) or a
change in the amount of fast carbon storage pool (if one or two carbon
pools are considered). After that, the model determines variations in
sapwood area, dead leaf area and live leaf area, which can originate due
to conversion from sapwood to heartwood, growth or drought-induced
cavitation. If two carbon storage pools are considered, at the end of
the day the model determines the direction and amount of carbon transfer
between them. Once a year (or by the end of the simulated period) the
model translates sapwood area growth into structural variables (i.e.,
plant height, tree DBH, tree crown ratio and shrub cover).

\subsection{Water balance, plant transpiration and
photosynthesis}\label{water-balance-plant-transpiration-and-photosynthesis}

The growth model calls the soil water balance model as a submodel to
perform soil water balance and photosynthesis calculations. We only
summarize the steps here. The submodel first increases soil moisture due
to precipitation, accounting for canopy interception loss, surface
runoff and deep drainage. To calculate water losses due to
transpiration, the submodel acts differently depending on whether
transpiration mode is set to `Granier' or `Sperry'. Generally speaking,
though, the submodel determines stomatal conductance of each plant
cohort according to the environmental conditions (i.e.~light, leaf
temperature, water vapor deficit and soil moisture) and this leads to an
estimation of transpiration and net photosynthesis. The submodel then
decreases water content due to bare soil evaporation, and plant
transpiration, which completes the daily water balance.

Among other outputs, the soil water balance submodel provides values of
leaf water potential \(\Psi_{leaf}\) (in MPa) and net photosynthesis
calculated at the plant cohort level, \(A_n^{coh}\) (in
\(g C · m^{-2} · day^{-1}\)). \(\Psi_{leaf}\) is used in the growth
model to modulate drought effects on growth and leaf area losses (see
below), whereas \(A_n\) is obviously needed to determine carbon balance
and growth. Since carbon balance is calculated at the individual level,
\(A_n^{coh}\) needs to be scaled to net photosynthesis per individual
(in \(g C · ind^{-1} · day^{-1}\)) using:

\begin{equation}
A_{n}^{ind} = 10000 \cdot A_n^{coh} / N
\end{equation}

where \(N\) is the density of individuals per hectare.

\section{Model input}\label{model-input-2}

\subsection{Vegetation state variables and
parameters}\label{vegetation-state-variables-and-parameters}

Vegetation in the stand is described using a set of plant cohorts,
described in an object of class \texttt{growthInput}. This function
assembles all parameters needed for the simulation of a given stand in a
single list. Model parameters are grouped by category. Regarding
physical aboveground description of the stand, each plant cohort is
defined by its species identity (\(SP\); with R name {[}\texttt{SP}{]}).
In addition, each cohort needs to be defined regarding the following
state variables:

\begin{itemize}
\tightlist
\item
  \(N\) {[}\texttt{N}{]}: The density of individuals (in
  \(ind · ha^{-1}\)).
\item
  \(DBH\) {[}\texttt{DBH}{]}: Tree diameter at breast height (in cm).
\item
  \(Cover\) {[}\texttt{Cover}{]}: Shrub percent cover (in \%).
\item
  \(H\) {[}\texttt{H}{]}: Total tree or shrub height (in cm).\}
\item
  \(CR\) {[}\texttt{CR}{]}: Crown ratio (i.e.~the ratio between crown
  length and total height).\}
\item
  \(LAI^{live}\) {[}\texttt{LAI\_live}{]}: Live leaf area index
  (one-side live leaf area of plants in the cohort per surface area of
  the stand) (in m\(^2\)·m\(^{-2}\)).\}
\item
  \(LAI^{\phi}\) {[}\texttt{LAI\_expanded}{]}: Expanded leaf area index
  (one-side expanded leaf area of plants in the cohort per surface area
  of the stand) (in m\(^2\)·m\(^{-2}\)).\}
\item
  \(LAI^{dead}\) {[}\texttt{LAI\_dead}{]}: Dead leaf area index
  (one-side dead leaf area of plants in the cohort per surface area of
  the stand) (in m\(^2\)·m\(^{-2}\)).\}
\item
  \(LAI^{predrought}\) {[}\texttt{LAI\_predrought}{]}: Live leaf area
  index before the current drought started (one-side dead leaf area of
  plants in the cohort per surface area of the stand) (in
  m\(^2\)·m\(^{-2}\)).\}
\item
  \(SA\) {[}\texttt{SA}{]}: Area of functional sapwood per individual
  (in cm\(^2\)·ind\(^{-1}\)).\}
\item
  \(C_{fast}\) {[}\texttt{fastCstorage}{]}: Amount of C in the fast
  carbon storage pool (in g C·ind\(^{-1}\)).\}
\item
  \(C_{slow}\) {[}\texttt{slowCstorage}{]}: Amount of C in the slow
  carbon storage pool (in g C·ind\(^{-1}\)).\}
\end{itemize}

Excepting \(SP\) (species identity) and \(N\) (density), the remaining
aboveground state variables are modified during growth simulations.
Belowground parameters are the following:

\begin{itemize}
\tightlist
\item
  \(Z\) {[}\texttt{Z}{]}: The rooting depth (in cm).\}
\item
  \(V\) {[}\texttt{V}{]}: A matrix with the proportion of fine roots in
  each soil layer.\}
\end{itemize}

Additional belowground variables are included if `transpirationMode =
``Complex''\}. These, and the parameters needed for water balance
calculations are described in vignette
`\textbf{Soil description and root system architecture}'.

The following physiological parameters are needed for growth
calculations:

\begin{itemize}
\tightlist
\item
  \(SLA\) {[}\texttt{SLA}{]}: Specific leaf area (m\(^2\)·kg\(^{-1}\)).
\item
  \(Hv\) {[}\texttt{Al2As}{]}: Huber value. Leaf area to sapwood area
  ratio (in m\(^2\)·m\(^{-2}\)).\}
\item
  \(W_{dens}\) {[}\texttt{WoodDens}{]}: Wood density (at 0\% humidity)
  (in g·cm\(^{-3}\)).\}
\item
  \(W_{C}\) {[}\texttt{WoodC}{]}: Wood carbon content in relation to dry
  weight (in g·g\(^{-1}\)).\}
\item
  \(C_{p,\max}\) {[}\texttt{Cstoragepmax}{]}: Maximum storage capacity,
  expressed as C per total respiratory C (in gC·gC\(^{-1}\)).
\item
  \(RGR_{\max}\) {[}\texttt{RGRmax}{]}: Maximum daily relative growth
  rate (in sapwood area basis) (in cm\(^2\)·cm\(^{-2}\)).
\end{itemize}

Another set of parameters is needed to transform changes in sapwood area
and leaf area to changes in the structural variables such as tree
height, tree crown ratio or shrub cover:

\begin{itemize}
\tightlist
\item
  \(H_{\max}\) {[}\texttt{Hmax}{]}: Maximum plant height (in cm).
\item
  \(f_{HD,\min}\), \(f_{HD,\max}\) {[}\texttt{fHDmin\},}fHDmax`{]}:
  Minimum and maximum values of the height-diameter ratio (in
  cm·cm\(^{-1}\)).
\item
  \(Z_{\max}\) {[}\texttt{Zmax}{]}: Maximum rooting depth (in mm).
\item
  \(a_{ash}\) {[}\texttt{Aash}{]}: Regression coefficient for the
  quadratic relationship between shrub height and shrub area.
\item
  \(a_{bsh}\), \(b_{bsh}\) {[}\texttt{Absh}, \texttt{Bbsh}{]}:
  Allometric coefficients relating crown phytovolume with dry weight of
  shrub individuals.
\item
  \(CR\) {[}\texttt{cr}{]}: Ratio between crown length and total height
  (constant value for shrubs).
\item
  \(r_{6.35}\) {[}\texttt{r635}{]}: Ratio between the dry weight of
  leaves plus branches and the dry weight of leaves alone for branches
  of 6.35 mm of diameter.
\item
  \(a_{cr}\), \(b_{1cr}\), \(b_{2cr}\), \(b_{3cr}\), \(c_{1cr}\),
  \(c_{2cr}\) {[}\texttt{B1cr}, \texttt{B2cr}, \texttt{B3cr},
  \texttt{C1cr}, \texttt{C2cr}{]}: Regression coefficients used to
  update the crown ratio of trees.
\item
  \(a_{cw}\), \(b_{cw}\) {[}\texttt{Acw}, \texttt{Bcw}{]}: Regression
  coefficients used to calculate the crown width of trees (as
  intermediary step to obtain the crown ratio).
\end{itemize}

\subsection{Metereological input}\label{metereological-input-1}

Weather input data must include variables calculated at the
\textbf{daily} scale. The variables required by function
\texttt{growth()} depend on the transpiration mode, similarly to
function \texttt{spwb()}. We recommend meteorological input to be
generated using package \textbf{meteoland} \citep{DeCaceres2018}.

\section{Model output}\label{model-output-2}

\chapter{Plant compartments, respiration and carbon
balance}\label{plant-compartments-respiration-and-carbon-balance}

\section{Biomass compartments}\label{biomass-compartments}

Biomass of leaves, sapwood and fine roots is needed in the model to
estimate respiratory costs (and, if needed, the size of the C storage
pools). Respiratory leaf biomass per individual (\(B_{leaf}\);
\(g C·ind^{-1}\)) is the result of dividing live expanded leaf area by
\(SLA\) (\(kg dry·m^{-2}\)), the specific leaf area coefficient of the
species, and multypling by a carbon conversion factor (0.3
\(gC·g dry^{-1}\)):

\begin{equation}
B_{leaf} = 0.3 \cdot 1000 \cdot LA^{\phi} / SLA
\end{equation}

where \(LA^{\phi} = 10000 \cdot LAI^{\phi} / N\) is the expanded leaf
area per individual (in m\(^2\)) and factor 1000 is used to convert from
kg to g. Hence, only expanded leaf area has respiratory cost
(i.e.~winter resistance buds do not) and counts for C storage purposes.

Respiratory sapwood biomass per individual (\(B_{stem}\);
\(g C·ind^{-1}\)) represents the sapwood biomass of stems (including
trunks and branches) and coarse roots. It is defined as the product of
sapwood area (\(SA\); in \(cm^2\)) per individual times the sum of
height (\(H\); in cm) and rooting depth (\(Z\); cm), transformed to
carbon biomass using species-specific parameters of wood C density
(\(W_{dens}\); \(g dry·cm^{-3}\)) and carbon content (\(W_{C}\);
\(g C · g dry^{-1}\)):

\begin{equation}
B_{stem} = SA\cdot (H + Z) \cdot W_{dens} \cdot W_{C}
\end{equation}

Finally, the biomass of fine roots per individual (\(B_{root}\); g
C·ind\(^{-1}\)) is simply assumed proportional to expanded leaf biomass
per individual:

\begin{equation}
B_{root} = B_{leaf}/2.5
\end{equation}

where 2.5 is a ratio between leaf biomass and fine root biomass. Hence,
fine-root maintenance respiration costs are also influenced by
leaf-phenological status \citep{Sitch2003}.

\section{Maximum capacity of C pools}\label{maximum-capacity-of-c-pools}

If carbon pools are considered, their maximum capacity is updated at
this point. If there is a single (fast) carbon pool, its maximum storage
capacity per individual (\(C_{fast,\max}\); \(g C·ind^{-1}\)) is defined
proportional to the total living biomass (i.e., easily accessed C
sources like sugars are assumed to be stored in all living parts of the
plant):

\begin{equation}
C_{fast,\max} = C_{p,\max} \cdot (B_{leaf} + B_{stem} + B_{root})
\end{equation}

where \(C_{p,\max}\) is the amount of C storage per plant respiratory C
weight. If two carbon pools are considered, their maximum capacity is
updated assuming that the fast pool corresponds to 5\% of plant
respiratory weight, and the slow pool corresponds to the remaining:

\begin{eqnarray}
C_{fast,\max} &=& 0.05 \cdot (B_{leaf} + B_{stem} + B_{root})\\
C_{slow,\max} &=& \max \left(C_{slow,\max},\, (C_{p, \max}-0.05) \cdot (B_{leaf} + B_{stem} + B_{root})\right)
\end{eqnarray}

Note the maximum function for the slow C pool, which ensures that the
size of the slow pool will not decrease if there is a decrease in leaf
area. Thus, while the slow C pool is still calculated in relation to the
total living biomass, it is assumed to be primarily found in
long-lasting organs (stem, roots, lignotubers, \ldots{}).

\section{Maintenance respiration and C
balance}\label{maintenance-respiration-and-c-balance}

Individual daily maintenance respiration (\(R^{ind}\); in
\(g C·ind^{-1}\)) is calculated for each of the three compartments
(leaves, alive vascular tissues (stem and coarse roots), and fine roots)
(Mouillot et al. 2001). The model uses a \(Q_{10}\) relationship with
temperature, which means that for every 10ºC change in temperature there
is a \(Q_{10}\) factor change in respiration. Baseline respiration rates
(\(r_{leaf}\), \(r_{stem}\) and \(r_{root}\) for leaves, vascular
tissues and fine roots, respectively; in \(gC·gC^{-1}\)) are not
species-specific and all refer to 20ºC:

\begin{eqnarray}
R_{leaf} &=& B_{leaf} \cdot r_{leaf} \cdot Q_{10}^{(T_{mean}-20)/10} \\
R_{stem} &=& B_{stem} \cdot r_{stem} \cdot Q_{10}^{(T_{mean}-20)/10} \\
R_{root} &=& B_{root} \cdot r_{root} \cdot Q_{10}^{(T_{mean}-20)/10} \\
R^{ind} &=& R_{leaf}+R_{stem}+R_{root}
\end{eqnarray}

where \(T_{mean}\) is the average daily temperature (in ºC). Note that
the output of \texttt{growth()} regarding respiration is actually the
result of scaling \(R^{ind}\) to the cohort level (\(R^{coh}\) in
\(g C · m^{-2} · day^{-1}\)), for comparability with photosynthesis and
transpiration:

\begin{equation}
R^{coh} =  R^{ind} \cdot N / 10000
\end{equation}

If no carbon pools are considered, the carbon available for growth is
simply the difference between individual's net photosynthesis and
maintenance respiration:

\begin{equation}
C_{available} = \max(0,\, A_n^{ind} -  R^{ind})
\end{equation}

whereas if carbon pools are considered, the fast C pool is updated with
the result of adding photosynthesis and subtracting respiration; and the
resulting pool size sets the amount of C available for growth:

\begin{eqnarray}
C_{fast} &=& \max(0,\,C_{fast} + A_n^{ind} -  R^{ind})\\
C_{available} &=& C_{fast}
\end{eqnarray}

\chapter{Sapwood conversion to heartwood, embolism and
growth}\label{sapwood-conversion-to-heartwood-embolism-and-growth}

\section{Sapwood conversion to heartwood, embolism and
growth}\label{sapwood-conversion-to-heartwood-embolism-and-growth-1}

\citet{Prentice1993} assumed a constant annual rate of 4\% for the
conversion from sapwood to heartwood. Similarly, \citet{Sitch2003}
assumed a sapwood annual turnover rate of 5\% for all biomes. A
reasonable value for maximum daily turnover rate would be (assuming an
annual rate 4.5\%):

\begin{equation}
1-0.955^{(1/365)} = 0.0001261398
\end{equation}

The actual proportion of sapwood area that is transformed into heartwood
is:

\begin{equation}
p_{heartwood} = \frac{0.0001261398}{1+15\cdot e^{-0.01\cdot H}}
\end{equation}

where 0.01 is a constant causing short plants to have slower turnover
rates.

Sapwood turnover is applied at the same rate in evergreen and deciduous
species. The amount of sapwood that is converted to heartwood every day,
\(\Delta SA_{turnover}\), is thus:

\begin{equation}
\Delta SA_{turnover} = SA \cdot p_{heartwood}
\end{equation}

Before applying either growth or cavitation the model determines the
extent to which cell turgor allows growth using a negative exponential
function:

\begin{equation}
f_{turgor}(\Psi_{leaf}) = 1 - \left[e^{(\Psi_{leaf}/\Psi_{tlp})-1}\right]^5
\end{equation}

where \(f_{turgor}(\Psi_{leaf})=0\) if \(\Psi_{leaf} > \Psi_{tlp}\). The
following figure illustrates the function for \(\Psi_{tlp}=-1.5\) MPa:

\begin{center}\includegraphics{medfatebook_files/figure-latex/unnamed-chunk-71-1} \end{center}

If \(f_{turgor}(\Psi_{leaf})>0\) growth is applied, but there can be
leaf area losses from sapwood turnover, whereas if
\(f_{turgor}(\Psi_{leaf}) = 0\) growth does not occur and cavitation is
possible. The following two subsection detail the behavior of the growth
model in each case.

\section{Growth and turnover during non-drought
periods}\label{growth-and-turnover-during-non-drought-periods}

If (i.e.\(f_{turgor}(\Psi_{leaf})>0\)) the model determines growth,
expressed as formation of new sapwood and leaf area increase. Daily
sapwood growth rate is assumed to depend on the availability of carbon
(i.e., \(C_{available}>0\)), but also requires temperature to be within
acceptable ranges (because it affects biochemical rates) and minimum
turgor for cell elongation.

The adoption of the pipe model (Shinozaki et al. 1964) implies that the
addition of new foliage requires building a proportional amount of xylem
conduits and fine roots. This is represented in the model by a
species-specific Huber value \(Hv\) (in \(m^2·m^{-2}\)). Since all
living biomass equations are linearly related to sapwood area, the total
cost in g C per 1 \(cm^2\) of newly formed sapwood area:

\begin{eqnarray}
C_{cost,leaf} &=& 0.3 \cdot \frac{0.1\cdot Hv}{SLA}\\
C_{cost,stem} &=& (H + Z) \cdot W_{dens} \cdot W_{C}\\
C_{cost,root} &=& C_{cost,leaf}/2.5\\
C_{cost,overall} &=& 1.3 \cdot (C_{cost,leaf}+C_{cost,stem} + C_{cost,root})
\end{eqnarray}

where 0.1 is needed to express \(C_{cost,leaf}\) in units of
\(g · cm^{-2}\) and factor 1.3 is used in the calculation of
\(C_{cost,overall}\) because growth respiration is assumed to be a
constant proportion of all new tissue growth (30\% of new tissue is
respired, Ryan 1990). Note that carbon allocation to the three
compartments does not follow constant proportions for different plants
because the larger the size of a plant, the more C will need to be
allocated in the vascular system per unit of sapwood area increment, and
hence the proportion of C allocated to leaves and fine roots will
decrease. The maximum increase in sapwood area according to the
availability of C is:

\begin{equation}
\Delta SA_{available} = \frac{C_{available}}{C_{cost,overall}}
\end{equation}

Several sink limitations may occur. One is the turgor limitation, which
is represented in the model by \(f_{turgor}\). If carbon pools are
considered, then the C concentration in the fast pool (i.e.
\(C_{fast}/C_{fast, \max}\)) may also limit the rate of growth. We model
this sink limitation by assuming that the maximum rate of growth will
decrease with decreasing concentration, following a sigmoidal function:

\begin{equation}
f_{conc} = \frac{1}{1+\exp \left(-5 \cdot \frac{(C_{fast}/C_{fast, \max})-0.5}{0.5}\right)}
\end{equation}

\begin{center}\includegraphics{medfatebook_files/figure-latex/unnamed-chunk-72-1} \end{center}

Obviously, if carbon pools are not considered \(f_{conc} = 1\).

Growth modulation due to temperature is incorporated through
\(f_{temp}(T_{mean})\), using a truncated parabolic function as in
Poyatos et al. (2007):

\begin{equation}
f_{temp}(T) = \frac{(T-T_{low}) \cdot (T_{high}-T)}{(T_{opt}-T_{low}) \cdot (T_{high}-T_{opt})}
\end{equation}

where \(0 \leq f_{temp}(T) \leq 1\), \(T_{low}\) and \(T_{high}\)
minimum and maximum temperature values for growth and \(T_{opt}\) is the
optimum growth temperature.

\begin{center}\includegraphics{medfatebook_files/figure-latex/unnamed-chunk-73-1} \end{center}

The rate of daily increase in sapwood area taking into account sink
limitations is (\(\Delta SA_{sink}\); \(cm^2\)):

\begin{equation}
\Delta SA_{sink} = RGR_{\max} \cdot SA \cdot f_{turgor}(\Psi_{leaf}) \cdot f_{conc} \cdot f_{temp}(T_{mean})
\end{equation}

where \(RGR_{\max}\) is the user-defined, species-specific maximum
relative growth rate in sapwood area (in \(cm^2·cm^{-2}\)), which can
incorporate nutrient deficiency effects.

The final growth rate of sapwood area per individual,
\(\Delta SA_{growth}\) (in \(cm^2\)), is found by combining the
potential increase in sapwood area according to availability and cost
with the sink limitations:

\begin{equation}
\Delta SA_{growth} = \min(\Delta SA_{available}, \, \Delta SA_{sink})
\end{equation}

If carbon pools are considered the actual carbon growth consumption
(i.e. \(C_{cost,overall} \cdot \Delta SA_{growth}\)) has to be
subtracted from \(C_{fast}\):

\begin{equation}
C_{fast} = C_{fast} - C_{cost,overall} \cdot \Delta SA_{growth}
\end{equation}

Sapwood area is updated considering both new sapwood formation and
sapwood conversion to heartwood (\(\Delta SA_{turnover}\)):

\begin{equation}
SA = SA + \Delta SA_{growth} - \Delta SA_{turnover} 
\end{equation}

After updating sapwood area, the model updates living, expanded and dead
leaf area accordingly:

\begin{eqnarray}
LAI^{live} &=& LAI^{live} + N \cdot (\Delta SA_{growth} - \Delta SA_{turnover}) \cdot Hv \\
LAI^{\phi} &=& LAI^{live}\,\cdot\phi \\
LAI^{dead} &=& LAI^{dead} + N \cdot \Delta SA_{turnover} \cdot Hv 
\end{eqnarray}

During non-drought periods the state variables regulating drought
effects are kept at initial values (i.e. \(\Psi_{\min} = 0\) and
\(LAI^{predrought} = LAI^{live}\)).

\section{Leaf area losses during
drought-periods}\label{leaf-area-losses-during-drought-periods}

During drought-periods (i.e.~if \(f_{turgor}(\Psi_{leaf})=0\))
cavitation may occur. However, cavitation is applied at the leaf area
level and not at the sapwood area level. During drought periods
reductions of live leaf area can come from either sapwood conversion
into heartwood or cavitation. First, the model compares the current
\(\Psi_{leaf}\) value with \(\Psi_{\min}\) the minimum potential
experienced since drought started:

\begin{equation}
\Psi_{\min} = \min(\Psi_{leaf}, \, \Psi_{\min})
\end{equation}

Then the model determines the proportion of embolized conducts as the
complement of hydraulic conductance corresponding to \(\Psi_{\min}\),
relative to the maximum hydraulic conductance. If the transpiration mode
is ``Simple'', this is done using a whole-plant conductance function:

\begin{equation}
P_{embolism}=1- K(\Psi_{\min}) = \exp \left \{\ln{(0.5)}\cdot \left[ \frac{\Psi_{\min}}{\Psi_{extract}} \right] ^r \right \} 
\end{equation}

where \(\Psi_{extract}\) is the potential at which conductance is 50\%
of maximum and \(r=3\). If the transpiration mode is ``Complex'',
\(P_{embolism}\) is calculated using the stem-leaves vulnerability
curve:

\begin{equation}
P_{emb}= 1- \frac{k_{stem}(\Psi_{\min})}{k_{stem}(0)} = 1 - \exp \left \{-\left[ \frac{\Psi_{\min}}{d_{stem}} \right] ^{c_{stem}} \right \} 
\end{equation}

Since leaf area reduction may also come from sapwood conversion into
heartwood, the model determines which process leads to a larger
reduction in leaf area:

\begin{equation}
LAI^{live} = \min(LAI^{live} - N \cdot \Delta SA_{turnover} \cdot Hv , \, LAI^{predrought} \cdot (1 - P_{emb}))
\end{equation}

and expanded leaf area (\(LAI^{\phi}\)) and dead leaf area
(\(LAI^{dead}\)) are modified accordingly.

\section{Transfer between carbon
pools}\label{transfer-between-carbon-pools}

If two carbon pools are considered then carbon can be transferred from
one to the other. The model assumes that the direction of transfer
depends on the C concentration in the fast pool. When the pool is at
full capacity its C should be converted to long-term storage, whereas if
the pool is empty C stored in the slow pool should be mobilised. The
relative rate of transfer (\(r_{transfer}\); in \(g C·g C^{-1}\)) is
modelled using a sigmoidal function:

\begin{equation}
r_{transf} = 0.1 \cdot \frac{2}{1+\exp \left(-5 \cdot \frac{(C_{fast}/C_{fast, \max})-0.5}{0.5}\right)}-1
\end{equation}

\begin{center}\includegraphics{medfatebook_files/figure-latex/unnamed-chunk-74-1} \end{center}

where \(0.1\) indicates that the maximum daily transfer rate is 10\% of
the source C pool size. If \(r_{transf}<0\) then
\(-r_{transf} \cdot C_{slow, \max}\) g of carbon are taken from the slow
C pool and 90\% of this amount is added into the fast C pool (the
remaining 10\% is assumed to be the transfer cost). Similarly, if
\(r_{transf}>0\) then \(r_{transf} \cdot C_{fast, \max}\) g of carbon
are taken from the fast C pool and 90\% of this amount is added into the
slow C pool (again, the remaining 10\% is assumed to be the transfer
cost). The amounts of C transferred are also limited by the amount that
would be needed to fill the sink pool (i.e., if the sink pool is already
full, no carbon is transferred).

\section{Update of maximum stem
conductance}\label{update-of-maximum-stem-conductance}

The inclusion of \(Hv\) in the initialization of \(SA\) and in growth
equations causes sapwood area and leaf area to maintain a constant ratio
equal to \(Hv\). However, this rule may be broken by leaf phenology or
when losing leaves because of drought stress. Leaf area reductions cause
transpirational demand to be reduced accordingly. Moreover, in the case
of Complex's transpiration mode leaf loss causes variations in leaf area
to sapwood area ratio, which may become lower than \(Hv\) and, hence,
maximum stem hydraulic conductance may increase. Tree size controls much
of the variation in stem hydraulic conductance, since hydraulic path
length increases with tree height. We modelled stem conductance per leaf
area unit (\(k_{stem, \max}\); in \(mmol·m^{-2}·s^{-1}·MPa^{-1}\)) as a
function of species-specific xylem conductivity (\(k_{xylem, \max}\); in
\(kg·m^{-1}·s^{-1}·MPa^{-1}\)), leaf area to sapwood area ratio and tree
height \citep{Christoffersen2016}:

\begin{equation}
k_{stem, \max} = \frac{1000}{0.018} \cdot \frac{k_{xylem, \max} \cdot (SA/10000)}{(H/100) \cdot LA^{\phi}} \cdot \chi_{taper}
\end{equation}

where \(\chi_{taper}\) is a factor to account for taper of xylem conduit
with height \citep{Savage2010, Christoffersen2016}, 0.018 is the molar
weight of water (in kg·mol\(^{-1}\)). This way, if the leaf-to-sapwood
area ratio decreases with tree height, as has been documented in several
species (McDowell et al. 2002), the decreasing conductance effects with
height may be partially overcome \citep{Christoffersen2016}. A
drought-induced decrease in \(LA^{\phi}\) alleviate drought effects
(i.e.~a lower decrease in water potential across the stem for the same
flow) because of an increase in conductance.

\chapter{Update of structural
variables}\label{update-of-structural-variables}

Unlike functional variables, structural variables are updated for
simplicity once a year only (or before the end of the simulated period).

\section{Tree diameter, height and crown
ratio}\label{tree-diameter-height-and-crown-ratio}

In the case of tree cohorts, the cumulated new sapwood area
(\(\sum{SA_{growth}}\)) is translated to an increment in DBH
(\(\Delta DBH\), in cm) following:

\begin{equation}
\Delta DBH = 2 \cdot \sqrt{(DBH/2)^2+(\sum{SA_{growth}}/\pi)} - DBH
\end{equation}

Furthermore, the model assumes that increments in height are linearly
related to increments in diameter through a function \(f_{HD}\) (Lindner
et al. 1997):

\begin{equation}
\Delta H = f_{HD} \cdot \Delta DBH
\end{equation}

Hence, \(f_{HD}\) represents the height increment (in cm) per each cm of
diameter increment. It was customary in forest gap models to prevent
height from being larger than a species-specific value \(H_{\max}\), so
that beyond some point trees only grew in size by increasing their
diameter. Moreover, light conditions influence growth in height with
trees living under the shade of others generally showing larger
increases in height than trees living in open conditions. Hence, our
formulation for \(f_{HD}\) is \citep{Lindner1997, Rasche2012}:

\begin{equation}
f_{HD} = \left[f_{HD,\min} \cdot L + f_{HD,\max} \cdot (1-L) \right] \cdot \left( 1 - \frac{H-137}{H_{\max} - 137} \right)
\end{equation}

where \(f_{HD,\min}\) would be the height-diameter ratio for a tree of
137 cm height growing in full light and \(f_{HD,\max}\) would be the
same ratio for a tree of the same height growing in the shadow. This
formulation seems slightly easier to calibrate than that presented in
\citet{Rasche2012}. \(H_{\max}\) could be dependent on environmental
conditions, but we skip this here, because environmental conditions
already affect growth rate and carbon balance.

After updating tree diameter (\(DBH\)) and tree height (\(H\)), the
model updates tree crown ratio (\(CR\)) by applying allometric
relationships that take into account tree size and competition (see
details in vignette XX).

\section{Shrub height and cover}\label{shrub-height-and-cover}

Since shrub structural variables are height and cover, shrub growth is
done in a way somewhat different from trees. Shrubs are often
multi-stemmed (some trees also are), so that increases in sapwood area
are not easily related to diameter growth. Since leaf biomass is related
to sapwood area, one may model shrub growth assuming an allometric
relationship between phytovolume of individual shrub crowns and
photosynthetic biomass. This strategy entails that shrubs may grow or
shrink in size depending on their C balance, in the same way that tree
crowns would become denser or sparser depending on their C balance.
Hence, shrubs can be understood as crowns in the floor.

Starting from live leaf area (\(m^2·m^{-2}\)) we can calculate the
foliar weight per shrub individual (in \(kg · ind^{-1}\)):

\begin{equation}
W_{leaves} = \frac{LAI^{live}} {(N/10000) \cdot SLA}
\end{equation}

An allometric relationship relating the biomass of leaves plus small
branches and crown phytovolume (\(PV\); in \(m^3·ind^{-1}\)) can be
drawn from fuel calculations:

\begin{equation}
W_{leaves+branches} = W_{leaves} \cdot r_{6.35}  = a_{bsh} \cdot PV^{b_{bsh}}
\end{equation}

where \(a_{bsh}\) and \(b_{bsh}\) are allometric relationships and
\(r_{6.35}\) is a species-specific ratio relating the dry weight of
leaves plus small branches to the dry weight of leaves. Inverting this
relationship we obtain an expression of shrub crown phytovolume:

\begin{equation}
PV = \left[\frac{W_{leaves} \cdot r_{6.35}}{a_{bsh}}\right]^{1/b_{bsh}}
\end{equation}

Phytovolume is defined as the volume occupied by the shrub crown, i.e.:

\begin{equation}
PV =  (A_{sh}/10000) \cdot (H/100) \cdot CR
\end{equation}

where \(A_{sh}\) is the area of a single shrub individual (in \(cm^2\)).
If we use the following quadratic relationship between \(A_{sh}\) and
\(H\):

\begin{equation}
A_{sh} = a_{ash} \cdot H^2
\end{equation}

we can calculate shrub height from phytovolume using:

\begin{equation}
H = \left[\frac{10^6 \cdot PV}{a_{ash} \cdot CR}\right]^{1/3}
\end{equation}

Finally, the new value for shrub cover (in percent) can be obtained from
\(H\) and \(N\) (in ind·ha\(^{-1}\)):

\begin{equation}
Cover = 100 \cdot (N/10000) \cdot (A_{sh}/10000) = \frac{N \cdot a_{ash} \cdot H^2}{10^6}
\end{equation}

Note that crown ratio for shrubs is assumed constant in the model. Like
for trees, shrub height is limited to a maximum height \(H_{\max}\).
However, unlike trees, shrubs are not allowed to continue growing once
this maximum size is attained. When the estimated height is over the
maximum value, the exceeding amount of live leaf area is allocated to
dead live area.

\part{Static modules}\label{part-static-modules}

\chapter{Leaf biomass, leaf area and vertical
profiles}\label{leafbiomass}

\section{Input data}\label{input-data}

\subsection{Forest inventory data}\label{forest-inventory-data}

The input data for tree cohorts are:

\begin{itemize}
\tightlist
\item
  \(SP_i\): Species identity\}
\item
  \(DBH_i\): Diameter at breast height (in cm) of the representative
  tree.
\item
  \(H_i\): Height (in cm) of the representative tree.
\item
  \(N_i\): Cohort density (in \(ind.\cdot ha^{-1}\)).
\end{itemize}

The input data for shrub cohorts are:

\begin{itemize}
\tightlist
\item
  \(SP_i\): Species identity
\item
  \(C_i\): Plant cover (in percent).
\item
  \(H_i\): Mean plant height (in cm).
\end{itemize}

\subsection{Allometric species-specific
parameters}\label{allometric-species-specific-parameters}

\begin{itemize}
\tightlist
\item
  \(a_{fbt}, b_{fbt}, c_{fbt}, d_{fbt}\) {[}\texttt{a\_fbt},
  \texttt{b\_fbt}, \texttt{c\_fbt}, \texttt{d\_fbt}{]}: Regression
  coefficients used to calculate foliar biomass of an individual tree
  from its \(DBH\) and the cummulative basal area of larger trees.
\item
  \(a_{ash}\) {[}\texttt{a\_ash}{]}: Regression coefficient relating the
  square of shrub height with shrub area.
\item
  \(a_{bsh}\) and \(b_{bsh}\) {[}\texttt{a\_bsh}, \texttt{b\_bsh}{]}:
  Allometric coefficients relating phytovolume with dry weight of shrub
  individuals.
\item
  \(cr(SP_i)\): {[}\texttt{cr}{]}: Ratio between crown length and total
  height for shrubs.
\end{itemize}

\section{Leaf biomass and area}\label{leaf-biomass-and-area}

\subsection{Leaf biomass for tree
cohorts}\label{leaf-biomass-for-tree-cohorts}

Foliar biomass for a single tree of cohort \(i\) (\(FB_{tree,i}\); in
\(kg\)) is calculated using:

\begin{equation}
FB_{tree,i} = a_{fbt} \cdot DBH_{i}^{b_{fbt}}\cdot e^{c_{fbt}\cdot BAL_i} \cdot DBH_{i}^{d_{fbt} \cdot BA_{sup}}
\label{eq:onetreefoliarbiomass}
\end{equation}

where \(DBH_{i}\) is the diameter of the tree (in \(cm\)), \(BAL_i\) is
the cummulative basal area (\(m^2\cdot ha^{-1}\)) of trees having a
larger diameter, and \(a_{fbt}\), \(b_{fbt}\), \(c_{fbt}\) and
\(c_{fbt}\) are species-specific regression coefficients. The foliar
biomass of the whole tree cohort (\(FB_{i}\); in \(kg\cdot m^{-2}\)) is
obtained multiplying tree foliar biomass by tree density (\(N_{i}\); in
\(ind.\cdot ha^{-1}\)):

\begin{equation}
FB_{i} = FB_{tree,i}\cdot (N_{i}/10000)
\label{eq:treefoliarbiomass}
\end{equation}

\subsection{Leaf biomass for shrub
cohorts}\label{leaf-biomass-for-shrub-cohorts}

To calculate the leaf biomass of a shrub cohort, we first determine
\(A_{sh,i}\), the area (in \(cm^2\)) occupied by one average individual
of height \(H_{i}\) (in \(cm\)), using the quadratic relationship:

\begin{equation}
A_{sh,i} = a_{ash} \cdot H_{i}^2
\end{equation}

where \(a_{ash}\) is a species-specific parameter. The model then
estimates the dry weight of leaves and branches up to 6.35mm in diameter
(\(B_{sh,i}\), in kg) of this average individual (i.e.~fine fuel
biomass), using an allometric relationship with shrub crown phytovolume
assuming a cylinder (in \(cm^3\)):

\begin{equation}
B_{sh,i} = a_{bsh} \cdot (A_{sh,i}\cdot H_{i}\cdot cr(SP_i))^{b_{bsh}}
\end{equation}

where \(a_{bsh}\) and \(b_{bsh}\) are species-specific parameters and
\(cr(SP_i)\) is a species-specific crown ratio (a proportion between 0
and 1, the ratio between crown length and total height). Shrub density
(\(N_{i}\); in \(ind.\cdot m^{-2}\)) can be grossly estimated from
percent cover (\(C_{i}\), in percent) and \(A_{sh,i}\) (in \(cm^{-2}\)):

\begin{equation}
N_{i} = \frac{C_{i}/100}{A_{sh,i}/10000}
\end{equation}

The fine fuel biomass of a shrub cohort (\(W_{i}\), in
\(kg \cdot m^{-2}\)) is simply the product of \(B_{sh,i}\) (\(kg\) of
dry weight) and \(N_{i}\):

\begin{equation}
W_{i} =  B_{sh,i}\cdot N_{i}
\label{eq:shrubloading}
\end{equation}

Foliar biomass (in \(kg \cdot m^{-2}\)) can be obtained using the
species-specific ratio \(r_{6.35}(SP_i)\):

\begin{equation}
FB_{i} =  W_{i}/r_{6.35}(SP_i)\\
\label{eq:shrubfoliarbiomass}
\end{equation}

If not known, \(r_{6.35}(SP_i)\) can be set to a default value of 2
(equivalent to \%50 of weight corresponding to leaves).

\subsection{Leaf area}\label{leaf-area}

\section{Vertical distribution of
leaves}\label{vertical-distribution-of-leaves}

\subsection{Crown base height}\label{crownbaseheight}

Crown base height (\(H_{b,i}\); in cm) is defined as the height
corresponding to the first living branch expressed as a fraction of
total height. It is calculated from the crown ratio (\(CR\); a
proportion between 0 and 1), i.e.~the ratio between crown length and
total height:

\begin{equation}
H_{b,i} = H_{i} \cdot (1 - CR_i)
\label{eq:CrownHeight}
\end{equation}

In the case of shrubs the crown ratio is simply equal by a
species-specific parameter: \(CR_i = cr(SP_i)\). In the case of trees,
the crown ratio is modelled as a function of tree size and stand
competition, following a modification of the logistic equation of
\citet{Hasenauer1996}:

\begin{equation}
CR_i = \frac{1}{1+e^{-(a_{cr}+b_{1cr}\cdot HD +b_{2cr} \cdot (H_i/100)+b_{3cr} \cdot DBH_i^2+c_{1cr} \cdot BAL_i + c_{2cr} \cdot ln(CCF))}}
\label{eq:treecrownratio}
\end{equation}

where \(HD = H_i/(100\cdot DBH_i)\) is the height to diameter ratio (in
\(m\cdot cm^{-1}\)), \(H_i\) is the tree height, \(DBH_i\) is the
diameter, \(CCF\) is the crown competition factor and \(a_{cr}\),
\(b_{1cr}\), \(b_{2cr}\), \(b_{3cr}\), \(c_{1cr}\) and \(c_{2cr}\) are
species-specific parameters. The crown competition factor is in turn
calculated using \citep{Krajicek1961}:

\begin{equation}
CCF = \sum_{i}{N_i \cdot MCA_i}= \sum_{i}{N_i \cdot \pi \cdot (CW_i/2)^2/100}
\end{equation}

where \(N_i\) is the tree density, \(MCA_i\) is the maximum crown area
(in percent of unit area) and \(CW_i\) is the crown width (in m)
assuming an open-grown tree, estimated from an allometric relationship
with tree diameter:

\begin{equation}
CW_i = a_{cw}\cdot BDH_i^{b_{cw}} 
\end{equation}

where again \(a_{cw}\) and \(b_{cw}\) are species-specific parameters.

\subsection{Leaf area density profile}\label{ladprofile}

Leaf area of any plant cohort is assumed to be distributed vertically
following a truncated Gaussian function whose standardized values -1.5
and 1.5 correspond to crown base height and total plant height,
respectively. The following figure illustrates the leaf area density
profile for an example forest stand.

\begin{Shaded}
\begin{Highlighting}[]
\KeywordTok{vprofile_leafAreaDensity}\NormalTok{(exampleforest, SpParamsMED, }\DataTypeTok{z =} \KeywordTok{seq}\NormalTok{(}\DecValTok{0}\NormalTok{,}\DecValTok{1000}\NormalTok{, }\DataTypeTok{by=}\DecValTok{1}\NormalTok{))}
\end{Highlighting}
\end{Shaded}

\begin{center}\includegraphics{medfatebook_files/figure-latex/unnamed-chunk-76-1} \end{center}

This density profile determines the extinction rates of light and wind
through the forest canopy.

\chapter{Fuel characteristics and fire
behaviour}\label{fuel-characteristics-and-fire-behaviour}

\section{Purpose}\label{purpose-1}

Functions \texttt{fuel\_FCCS()} and \texttt{fire\_FCCS()} allow
calculating potential fire behaviour for forest inventory plots.
Formulation of fuel characteristics and fire behaviour is an adaptation
of the Fuel Characteristics Classification System
\citep[FCCS;][]{Prichard2013}. In FCCS, fuelbed is divided into six
strata, including canopy, shrub, herbaceous vegetation, dead woody
materials, leaf litter and ground fuels. All except ground fuels are
considered here. The intensity of burning depends on several factors,
including topography, wind conditions, fuel structure and its moisture
content, which is determined from antecedent and current meteorological
conditions. A modification of the Rothermel's
\citeyearpar{Rothermel1972} model is used to calculate the intensity of
surface fire reaction (in \(kW/m^2\)) and the rate of fire spread (in
\(m/min\)) of surface fires assuming a steady-state fire. Both
quantities are dependent on fuel characteristics, windspeed and
direction, and topographic slope and aspect. The model returns the
following results: (1) Fuel characteristics by stratum; (2) Surface fire
behavior (i.e.~reaction intensity, rate of spread, fireline intensity
and flame length); (3) Crown fire behavior; (4) Fire potential ratings
of surface fire behavior and crown fire behavior.

\section{Input data}\label{input-data-1}

\subsection{Forest inventory plot
data}\label{forest-inventory-plot-data}

The input data for tree cohorts are:

\begin{itemize}
\tightlist
\item
  \(SP_i\): Species identity\}
\item
  \(DBH_i\): Diameter at breast height (in cm) of the representative
  tree.
\item
  \(H_i\): Height (in cm) of the representative tree.
\item
  \(N_i\): Cohort density (in \(ind.\cdot ha^{-1}\)).
\end{itemize}

The input data for shrub cohorts are:

\begin{itemize}
\tightlist
\item
  \(SP_i\): Species identity
\item
  \(C_i\): Plant cover (in percent).
\item
  \(H_i\): Mean plant height (in cm).
\end{itemize}

Cohorts are not distinguished for the herbaceous stratum, and the
variables needed are:

\begin{itemize}
\tightlist
\item
  \(C_{he}\): Herbaceous cover (in percent).
\item
  \(H_{he}\): Mean herb height (in cm).
\end{itemize}

Finally, the model also requires the percent cover of trees in the
canopy (\(C_{ca}\)). This is easily available from forest inventory
data, but could also be derived from the description of tree cohorts.

\subsection{Species-specific
parameters}\label{species-specific-parameters}

\begin{itemize}
\tightlist
\item
  \(r_{6.35}(SP_i)\) {[}\texttt{r635}{]}: Ratio between the weight of
  leaves plus branches and the weight of leaves alone for branches of
  6.35 mm.
\item
  \(\rho_p(SP_i)\) {[}\texttt{PD}{]}: Particle density.
\item
  \(\sigma(SP_i)\) {[}\texttt{SAV}{]}: Surface-area-to-volume ratio of
  the small fuel (1h) fraction (leaves and branches \textless{} 6.35mm).
\item
  \(h(SP_i)\) {[}\texttt{HeatContent}{]}: High fuel heat content.
\item
  \(\eta_{F}(SP_i)\) {[}\texttt{Flammability}{]}: Flammability value
  (either 1 or 2 for normal or high, respectively).
\item
  \(LD(SP_i)\) {[}\texttt{LeafDuration}{]}: Leaf duration (in years).
\item
  \(LI(SP_i)\) {[}\texttt{PercentLignin}{]}: Percentage of lignin in
  leaves.
\end{itemize}

\subsection{Other inputs}\label{other-inputs}

Other inputs may be given by expert opinion or they may be calculated
from another model. Specifically, for each plant cohort (and for any day
of application) the fire behaviour model requires:

\begin{itemize}
\tightlist
\item
  \(P_{dead,i}\): Proportion of the plant that is dead.
\item
  \(M_i\): Foliar moisture value (in percent of dry weight).
\end{itemize}

Analogously, the same variables are needed for the herbaceous stratum.

\begin{itemize}
\tightlist
\item
  \(P_{dead,he}\): Proportion of herb fuels that respond to humidity
  changes as 1-h dead fuels.
\item
  \(M_{live,he}\): Foliar moisture value of live herb fuels (in percent
  of dry weight).
\end{itemize}

The model also needs the following input parameters:

\begin{itemize}
\tightlist
\item
  \(M_{dead}\): the moisture of 1-h dead fuels (in percent of dry
  weight).
\item
  \(U\): Midflame windspeed (in \(m\cdot s^{-1}\)).
\item
  \(S\): Slope (in percent).
\end{itemize}

\section{Fuel characteristics}\label{fuel-characteristics}

\subsection{Fuel strata}\label{fuel-strata}

The Fuel Characteristics Classification System (FCCS) on which this
document is based, defines six fuel strata \citep{Prichard2013}:

\begin{itemize}
\tightlist
\item
  \emph{Canopy}: Trees, snags and ladder fuels.
\item
  \emph{Shrubs}: Primary and secondary layers.
\item
  \emph{Non-woody vegetation (herbs)}: grasses, sedges, rushes and
  forbs.
\item
  \emph{Woody fuels}: All downed and dead wood, sound wood, rotten wood
  and stumps.
\item
  \emph{Litter-lichen-moss}: Lichen, litter and moss layers.
\item
  \emph{Ground fuels}: Duff, basal accumulation and squirrel middens.
\end{itemize}

Shrubs, herbs and woody fuels are constitute the \textbf{upper surface
fuels}, whereas herbs and woody fuels alone consitute the \textbf{lower
surface fuels}. FCCS summarizes and calculates characteristics for each
fuelbed stratum and layer. Our model estimates fuel loading and
characteristics for canopy, shrub, non-woody vegetation, as well as fine
(1h) woody fuels and litter fuels. Larger woody fuels (10h or 100h)
could be considered if information about forest management actions is
available. Ground fuels are not included here.

\subsection{Cohort fuel loading}\label{cohort-fuel-loading}

Here we consider as burnable fuels foliage and branches up to 6.35 mm =
0.25 in in diameter. The same consideration applies to both trees and
shrubs.

Fine fuel loading for a tree cohort (\(W_{i}\); in \(kg\cdot m^{-2}\)),
including its leaves and branches with diameter up to 6.35 mm = 0.25 in,
is calculated from foliar biomass (\(FB_{i}\), see eq.
\eqref{eq:treefoliarbiomass}) using:

\begin{equation}
W_{i} = r_{6.35}(SP_i)\cdot FB_{i}
\end{equation}

where \(r_{6.35}(SP_i)\) is the ratio between the weight of leaves plus
branches and the weight of leaves alone for branches of 6.35 mm in
diameter for the species of cohort \(i\). The biomass corresponding to
branches of less than \textless{} 6.35 mm (\(SBB_{i}\), also in
\(kg\cdot m^{-2}\)) is obtained by subtraction:

\begin{equation}
SBB_{i} = (r_{6.35}(SP_i)-1)\cdot FB_{i}
\end{equation}

Whereas \(W_{i}\) is the cohort loading variable influencing fire
behavior, \(FB_{i}\) and \(SBB_{i}\) are cohort variables used to
estimate fine dead woody and leaf litter loadings.

Our procedure to estimate shrub fuel loading differs from
\citet{Prichard2013} because they calculate first total biomass of the
shrub species and then consider the percentage of total weight that
corresponds to leaves and small branches. In our case, we estimate fine
fuel loading (\(W_{i}\), in \(kg \cdot m^{-2}\)) and foliar biomass
(\(FB_i\), in \(kg \cdot m^{-2}\)) of shrubs from eqs.
\eqref{eq:shrubloading} and \eqref{eq:shrubfoliarbiomass}. Biomass of small
branches (in \(kg \cdot m^{-2}\)) can be obtained from :

\begin{equation}
SBB_{i} = W_{i} - FB_{i}
\end{equation}

\subsection{Vertical distribution of cohort
fuels}\label{vertical-distribution-of-cohort-fuels}

For all tree or shrub cohorts fuels are assumed to be homogeneously
distributed between the crown base height (\(H_{b,i}\); in cm) and the
total height (\(H_i\)). Crown base height of trees is calculated as
explained in \ref{crownbaseheight}. The loading of a cohort that occurs
within a given height interval of limits \(H_1\) and \(H_2\) is
calculated as:

\begin{equation}
W_i(H_1, H_2) = W_i\cdot p_i(H_1, H_2)
\end{equation}

where \(p_i(H_1, H_2)\) is the proportion of the crown of cohort \(i\)
that corresponds to the height interval \((H_1, H_2)\):

\begin{equation}
p_i(H_1, H_2) = \frac{\max(0, \min(H_i, H_2) - \max(H_{b,i}, H_1))}{H_{i}-H_{b,i}}
\end{equation}

\subsection{Fuel bulk density profile}\label{fuel-bulk-density-profile}

Knowing at which height fuels are placed, the \textbf{fuel bulk density
profile} \citep{Reinhardt2006} is defined for any given interval
\((H_1, H_2)\) as the bulk density (\(kg/m^3\)) of fine fuels
corresponding to that interval:

\begin{equation}
BDP(H_1, H_2) = \frac{\sum_{i} W_i(H_1, H_2)}{H_2-H_1}
\end{equation}

Canopy bulk density normally ranges between 0 and 0.4 \(kg/m^3\)
\citep{Scott2002}. \citet{Sando1972} arbitrarily defined canopy base
height as the lower vertical 0.3-m section with a weight greater than
\(0.01124 kg/m^3\). A user-defined threshold \(t_{BDP}\) (in \(kg/m^3\))
in 0.1-m sections is used to differentiate the surface fuelbed from
canopy fuels. Using this threshold the model calculates the following
three heights \citep{Reinhardt2006}:

\begin{itemize}
\tightlist
\item
  \emph{Shrub stratum base height}, \(H_{sb}\) (in \(cm\)): the minimum
  height between 0 and 2 m where fuel bulk density is larger than
  \(t_{BDP}\).
\item
  \emph{Shrub stratum top height}, \(H_{st}\) (in \(cm\)): the maximum
  height between 0 and 2 m where fuel bulk density larger than
  \(t_{BDP}\). With this definition \(h_{s}\) cannot be higher than 2 m
  (corresponding to fuel model 4 in Anderson 1982).
\item
  \emph{Canopy base height}, \(H_{cb}\) (in \(cm\)): In terms of its
  consequences to crown fire initiation, canopy base height can be
  defined as the lowest height above the ground at which there is
  sufficient canopy fuel to propagate fire vertically through the
  canopy. It is calculated as the minimum height over \(H_st\) when the
  bulk density starts again to be larger than \(t_{BDP}\).
\item
  \emph{Canopy top height}, \(H_{ct}\) (in \(cm\)): the maximum height
  where bulk density is larger than \(t_{BDP}\).
\item
  \emph{Canopy gap}, \(H_{gap}\) (in \(cm\)): the difference between
  \(H_{cb}\) and \(H_{st}\). The canopy gap is used to calculate crown
  initiation potential.
\end{itemize}

The following figure illustrates the definition and analysis of the fuel
bulk density profile for a given forest stand. Following
\citet{Mitsopoulos2007}, a threshold \(t_{BDP} = 0.04\) has been used to
determine shrub and canopy heights.

\begin{center}\includegraphics{medfatebook_files/figure-latex/unnamed-chunk-78-1} \end{center}

\subsection{\texorpdfstring{Fuel loading (\(w\)) and fuel depth
(\(\delta\))}{Fuel loading (w) and fuel depth (\textbackslash{}delta)}}\label{fuel-loading-w-and-fuel-depth-delta}

\subsubsection{Canopy stratum}\label{canopy-stratum}

Canopy loading (in \(kg\cdot m^{-2}\)) is the sum of (tree and shrub)
cohort loadings above 2 m (i.e.~200 cm):

\begin{equation}
w_{ca} = \sum_{i}w_{i,ca} =\sum_{i}{W_i(200, \infty})
\end{equation}

where \(w_{i,ca}\) is the canopy stratum loading of cohort \(i\). Canopy
depth (in \(m\)) is defined as the average of tree (or shrub) crown
lengths above 2 m, weighted by the loadings of cohorts in the canopy:

\begin{equation}
\delta_{ca} = \frac{1}{100}\cdot\frac{\sum_{i}{w_{i,ca}\cdot (H_i - H_{b,i})\cdot p_{i,ca} }}{\sum_{i}{w_{i,ca}}}
\end{equation}

where the proportion of a tree (or shrub) cohort in the canopy stratum
is \(p_{i,ca}=p_{i}(200,\infty)\).

\subsubsection{Shrub stratum}\label{shrub-stratum}

Shrub loading (in \(kg\cdot m^{-2}\)) is the sum of (tree and shrub)
cohort loadings between the ground and 2 m (i.e.~200 cm):

\begin{equation}
w_{sh} = \sum_{i}w_{i,sh} =\sum_{i}W_i(0, 200)
\end{equation}

where \(w_{i,sh}\) is the shrub stratum loading of cohort \(i\). The
depth of the shrub stratum (in \(m\)) is defined as the average of tree
(or shrub) crown lengths below 2 m, weighted by the loadings of cohorts
in the shrub stratum:

\begin{equation}
\delta_{sh} = \frac{1}{100}\cdot \frac{\sum_{i}{w_{i,sh}\cdot (H_i - H_{b,i})\cdot p_{i,sh} }}{\sum_{i}{w_{i,sh}}}
\end{equation}

where the proportion of a shrub (or tree) cohort in the shrub stratum is
\(p_{i,sh}=p_{i}(0,200)\).

\subsubsection{Non-woody stratum}\label{non-woody-stratum}

Herb percent cover and average herb height are transformed into
herbaceous loading (\(kg\cdot m^{-2}\)) using (ref piropinos):

\begin{equation}
w_{he} = 0.014 \cdot C_{he} \cdot (H_{he}/100)
\end{equation}

The depth of the herbaceous stratum (in \(m\)) is simply the mean height
of herbs:

\begin{equation}
\delta_{he} = H_{he}/100
\end{equation}

\subsubsection{Woody and litter strata}\label{woody-and-litter-strata}

In FCCS \citep{Prichard2013}, woody surface loading includes several
fuel sizes. However, when calculating surface fire behavior \(w_{wo}\)
includes 100\% of 1h fuels, 25\% of 10h fuels and 12.5\% of 100h fuels,
which represents the material available for flaming combustion.
Obtaining loading estimates for 10h- and 100h-fuels is very difficult
without field fuel sampling. However, we might estimate 1h woody fuels
and leaf litter from standing biomass of small branches (\textless{}
6.35mm) and leaves for trees and shrubs. Hence, our treatment of surface
woody fuels includes only fine (1h) fuels.

Assuming a continuous input of litter, the variation in accumulated
litter is described by a simple differential equation \citep{Birk1980}:

\begin{equation}
\frac{\mathrm{d}X}{\mathrm{d}t} = L - k\cdot X
\end{equation}

where \(k\) is the decay constant, \(L\) is the rate of litterfall and
\(X\) is the litter mass accumulated in the forest floor. Assuming that
litter mass has reached a steady state, \(X\) can be estimated as the
ratio between \(L\) and \(k\). If litterfall is estimated as the total
foliar biomass divided by leaf duration, the amount of steady state leaf
litter corresponding to each tree and srhub cohort can be estimated
using:

\begin{equation}
w_{li, i} = \frac{FB_i}{LD(SP_i) \cdot k_i}
\end{equation}

where \(FB_i\) is the foliar biomass of cohort \(i\), \(LD(SP_i)\) is
the species-specific average leaf duration (in years) and \(k_i\) is the
rate of decay of leaves of cohort \(i\), which is given by the
regreession model of Meentemeyer (1978):

\begin{equation}
k_i = (-0.5365+0.00241\cdot AET) - (-0.01586+0.000056\cdot AET) \cdot LI(SP_i) 
\end{equation}

where \(LI(SP_i)\) is the species-specific percentage of lignin content
in leaves and AET is actual evapotranspiration (default
\(AET = 1000 mm\)). Litter loadings are summed for four litter types
(short pine needles, long pine needles, other conifers, broadleaves). In
the case of fine dead woody materials (small fallen branches), loading
of small branches is taken as woody litter and it is assumed that small
branch litterfall occurs at the same time as leaf litterfall
(i.e.~according to leaf duration):

\begin{equation}
w_{wo} = \sum_{i}{w_{wo, i}} = \sum_{i}{\frac{SBB_i}{LD(SP_i)\cdot k_{wo}} }
\end{equation}

where \(k_{wo} = 0.95 y^{-1}\) is a constant rate of decomposition for
small branches.

In FCCS, the depth of woody and LLM strata are inputs. In our case the
depth of the woody and litter strata are estimated from the
corresponding fuel loadings:

\begin{eqnarray}
\delta_{wo} &= w_{wo}/\rho_{b, wo}\\
\delta_{li} &= w_{li}/\rho_{b, li}
\end{eqnarray}

where \(\rho_{b, wo}\), \(\rho_{b, li}\) are the woody and litter bulk
density (in \(kg\cdot m^{-3}\)), respectively. Litter bulk density
\(\rho_{b,li}\) is calculated as a weighted average of litter types:

\begin{equation}
 \rho_{b,li} = \frac{\sum_{k}{ \rho_{b,k}\cdot w_{li,k}}}{\sum_{k} {\cdot w_{li,k}}}
\end{equation}

where \(k\) indicates litter type. The bulk density for litter types are
{[}\citet{Prichard2013}; Table 1{]}:

\begin{eqnarray}
\rho_{b,shortneedlepine} &= \rho_{b,longneedlepine} = \rho_{b,otherconifer}= 1.65 lb\cdot ft^{-3} = 26.43 kg\cdot m^{-3}\\
\rho_{b,hardwood} &= 0.83 lb\cdot ft^{-3} = 13.30  kg\cdot m^{-3}
\end{eqnarray}

\subsection{Other fuel
characteristics}\label{other-fuel-characteristics}

All the following characteristics are calculated in metric units
(although British units are indicated to qualify specific values for
compatibility).

\subsubsection{\texorpdfstring{Particle density
(\(\rho_{p}\))}{Particle density (\textbackslash{}rho\_\{p\})}}\label{particle-density-rho_p}

Particle density is the ratio of dry weight over volume for fuel
particles (in \(kg\cdot m^{-3}\)). When species have different values,
particle density averages for shrub and canopy strata can be obtained
as:

\begin{eqnarray}
\rho_{p, sh} &= \frac{\sum_{i}{w_{i,sh} \cdot \rho_p(SP_i)}}{\sum_{i}{w_{i,sh}}}\\
\rho_{p, ca} &= \frac{\sum_{i}{w_{i,ca} \cdot \rho_p(SP_i)}}{\sum_{i}{w_{i,ca}}}
\end{eqnarray}

where \(\rho_p(SP_i)\) is the species-specific particle density.
\(\rho_{p, he}\), \(\rho_{p, wo}\) and \(\rho_{p, li}\) are all set to a
default value \(\rho_{p} = 400 kg\cdot m^{-3}= 25 lb\cdot ft^{-3}\)
\citep{Prichard2013}.

\subsubsection{\texorpdfstring{Particle volume
(\(PV\))}{Particle volume (PV)}}\label{particle-volume-pv}

Particle volume is defined as the volume of particles per surface area
(in \(m^3\cdot m^{-2}\)). Is calculated as dry weight loading divided by
particle density. If species have different particle density values, the
particle volume for canopy (\(PV_{ca}\)) and shrub(\(PV_{sh}\)) strata
can be calculated using:

\begin{eqnarray}
PV_{ca} &= \sum_{i}{PV_{i,ca}} = \sum_{i}{w_{i,ca}/\rho_{p}(SP_i)}\\
PV_{sh} &= \sum_{i}{PV_{i,sh}} = \sum_{i}{w_{i,sh}/\rho_{p}(SP_i)}
\end{eqnarray}

where \(PV_{i,ca}\) and \(PV_{i,sh}\) are the particle volume of cohort
\(i\) in the canopy and shrub strata, respectively. The particle volume
for woody and herb strata are simply:

\begin{eqnarray}
PV_{wo} &= w_{wo}/\rho_{p,wo}\\
PV_{he} &= w_{he}/\rho_{p,he}
\end{eqnarray}

The particle volume for the litter stratum is the sum of particle volume
of litter components:

\begin{equation}
PV_{li} = \sum_{i}{PV_{li,k}} = \sum_{i}{w_{li,k}/\rho_{p, li}}
\end{equation}

\subsubsection{\texorpdfstring{Packing ratio
(\(\beta\))}{Packing ratio (\textbackslash{}beta)}}\label{packing-ratio-beta}

The proportion of fuelbed stratum volume occupied by fuel particles is
an important factor to predict fire behavior. At low packing ratios (low
particle density) fire intensity is limited by excessive heat loss. At
high packing ratios (high particle density), lack of oxygen limits
combustion. The packing ratios for the canopy and shrub stratum
(\(\beta_{ca}\) and \(\beta_{sh}\); dimensionless) are given by:

\begin{eqnarray}\\eqref{eq:packingratioshrub}
\beta _{ca} &= \frac{PV_{ca}}{\delta_{ca}}\\
\beta _{sh} &= \frac{PV_{sh}}{\delta_{sh}}
\end{eqnarray}

where \(w_{i,ca}\) and \(w_{i,sh}\) are the contribution of cohort \(i\)
to canopy and shrub strata loading (in \(kg\cdot m^{-2}\)),
respectively, and \(\rho_p(SP_i)\) is the particle density (in
\(kg\cdot m^{-3}\)) of fuels in cohort \(i\). The packing ratio for the
herbaceous, woody and litter strata are:

\begin{eqnarray}
\beta _{he} &= \frac{PV_{he}}{\delta_{he}}\\
\beta _{wo} &= \frac{PV_{wo}}{\delta_{wo}}= \frac{\rho_{b,wo}}{\rho_{p,wo}}\\
\beta _{li} &= \frac{PV_{li}}{\delta_{li}}= \frac{\rho_{b,li}}{\rho_{p,li}}
\end{eqnarray}

Note that the packing ratio expressions for woody and litter strata as a
ratio of bulk and particle density arises as a consequence of how fuel
depth and particle volume are estimated.

\subsubsection{\texorpdfstring{Surface-area-to-volume ratio
(\(\sigma\))}{Surface-area-to-volume ratio (\textbackslash{}sigma)}}\label{surface-area-to-volume-ratio-sigma}

The surface-area-to-volume ratio (in \(m^2\cdot m^{-3}\)) for the canopy
or shrub strata are calculated using weighted averages:

\begin{eqnarray}
\sigma_{ca} &= \frac{\sum_{i}{w_{i,ca} \cdot \sigma(SP_i)}}{\sum_{i}{w_{i,ca}}}\\
\sigma_{sh} &= \frac{\sum_{i}{w_{i,sh} \cdot \sigma(SP_i)}}{\sum_{i}{w_{i,sh}}}
\end{eqnarray}

where \(w_{i,ca}\) and \(w_{i,sh}\) are the contribution of cohort \(i\)
to canopy and shrub strata loading (in \(kg\cdot m^{-2}\)),
respectively, and \(\sigma(SP_i)\) is the species-specific
surface-area-to-volume ratio. The surface-area-to-volume ratio of herbs
is assumed constant
\(\sigma_{he} = 11483 m^2\cdot m^{-3} = 3500 ft^2\cdot ft^{-3}\) and
that of small (1-h) woody fuels is
\(\sigma_{wo} = 1601.05 m^2\cdot m^{-3} = 488 ft^2\cdot ft^{-3}\). The
surface-area-to-volume ratio for the litter stratum is:

\begin{equation}
\sigma_{li} = \frac{\sum_{k}{w_{li,k} \cdot \sigma_{k}}}{\sum_{k}{w_{li,k}}}
\end{equation}

and the surface-area-to-volume ratio for litter types are:

\begin{eqnarray}
\sigma_{shortneedlepine} &= 6562 m^{2}\cdot m^{-3}= 2000 ft^{2}\cdot ft^{-3}\\
\sigma_{longneedlepine} &= 4921 m^{2}\cdot m^{-3}= 1500 ft^{2}\cdot ft^{-3}\\
\sigma_{otherconifer} &= 8202 m^{2}\cdot m^{-3}= 2500 ft^{2}\cdot ft^{-3}\\
\sigma_{hardwood} &= 8202 m^{2}\cdot m^{-3}= 2500 ft^{2}\cdot ft^{-3}
\end{eqnarray}

\subsubsection{Fuel area index (FAI)}\label{fuel-area-index-fai}

The fuel area index (FAI) is the total fuel surface area per unit of
ground area (unitless). It is analogous to leave area index (LAI), and
it is used to calculate FCCS fire potentials \citep{Schaaf2007}. For
shrub and canopy strata, FAI is calculated as:

\begin{eqnarray}
FAI_{ca} &= \sum_{i}{FAI_{i, ca}} = \sum_{i}{PV_{i,ca} \cdot\sigma(SP_i)}\\
FAI_{sh} &= \sum_{i}{FAI_{i, sh}}= \sum_{i}{PV_{i,sh} \cdot \sigma(SP_i)}
\end{eqnarray}

where \(FAI_{i, ca}\) and \(FAI_{i, sh}\) are the FAI of cohort \(i\) in
the canopy and shrub strata, respectively. The FAI of herbs and woody
strata are given by:

\begin{eqnarray}
FAI_{he} &= PV_{he} \cdot \sigma_{he}\\
FAI_{wo} &= PV_{wo} \cdot \sigma_{wo}
\end{eqnarray}

For the litter layer, FAI is calculated as a sum of FAI for litter
components:

\begin{equation}
FAI_{li} = \sum_{k}{FAI_{li,k}} = \sum_{k}{PV_{li,k} \cdot \sigma_{k}}
\end{equation}

\subsubsection{\texorpdfstring{Moisture
(\(M\))}{Moisture (M)}}\label{moisture-m}

Live foliar moisture (in percent of dry weight) is also averaged across
cohorts composing the shrub or canopy strata:

\begin{eqnarray}
M_{live, sh} &= \frac{\sum_{i}{w_{i,sh} \cdot M_i}}{\sum_{i}{w_{i,sh}}} \\
M_{live, ca} &= \frac{\sum_{i}{w_{i,ca} \cdot M_i}}{\sum_{i}{w_{i,ca}}}
\end{eqnarray}

Live foliar moisture of herb stratum (\(M_{live, he}\)) is an input. The
moisture of dead plant in the canopy and shrub layers (\(M_{dead, ca}\)
and \(M_{dead, sh}\)), the moisture of dead herbs (\(M_{dead, he}\)), as
well as that of litter (\(M_{li}\)) and woody (\(M_{wo}\)) strata are
all assumed equal to the moisture of 1-h dead fuels, which is an input
of the model.

\subsubsection{\texorpdfstring{Proportion of dead fuel
(\(P_{dead}\))}{Proportion of dead fuel (P\_\{dead\})}}\label{proportion-of-dead-fuel-p_dead}

Woody and litter strata are dead fuels, but for canopy, shrub and herb
strata the proportion of fuels that are dead are variable. The
proportion of dead fuels in the herbaceous stratum (\(P_{dead,he}\)) is
an input of the model, but for the shrub and canopy strata these are
calculated from the proportion of dead fuels in each cohort:

\begin{eqnarray}
P_{dead,sh} &= \frac{\sum_{i}{w_{i,sh} \cdot P_{dead,i}}}{\sum_{i}{w_{i,sh}}} \\
P_{dead, ca} &= \frac{\sum_{i}{w_{i,ca} \cdot P_{dead,i}}}{\sum_{i}{w_{i,ca}}}
\end{eqnarray}

\subsubsection{\texorpdfstring{Low heat content
(\(h\))}{Low heat content (h)}}\label{low-heat-content-h}

The low fuel heat content of each surface fuel stratum (in
\(kJ\cdot kg^{-1}\)) is used for the calculation of reaction intensity.
Heat content values are adjusted for live foliar moisture content in
canopy, shrub and herb strata; and are left to the default value for
woody and litter strata:

\begin{eqnarray}
h_{ca} &=& h_{ca, def} - (M_{live, ca}/100)\cdot V \\
h_{sh} &=& h_{sh, def} - (M_{live, sh}/100)\cdot V \\
h_{he} &=& h_{def} - (M_{live, he}/100)\cdot V \\
h_{wo} &=& h_{li} = h_{def}
\end{eqnarray}

where \(h_{def} = 18608 kJ\cdot kg^{-1} = 8000 Btu\cdot lb^{-1}\) is the
default low heat content value for herbs, woody and litter strata, and
\(V = 2596 kJ\cdot kg^{-1} = 1116 Btu\cdot lb\) is the latent heat of
vaporisation of water. The default low heat of contents for the canopy
and shrub strata (\(h_{ca, def}\) and \(h_{sh, def}\)) are calculated as
a weighted average across cohorts:

\begin{eqnarray}
h_{ca, def} &=& \frac{\sum_{i}{w_{i,ca} \cdot h(SP_i)}}{\sum_{i}{w_{i,ca}}}\\
h_{sh, def} &=& \frac{\sum_{i}{w_{i,sh} \cdot h(SP_i)}}{\sum_{i}{w_{i,sh}}}
\end{eqnarray}

where \(h(SP_i)\) is a species-specific low heat content value.

\subsubsection{\texorpdfstring{Flammability index
(\(\eta_{F}\))}{Flammability index (\textbackslash{}eta\_\{F\})}}\label{flammability-index-eta_f}

Flammability index (\(\eta_{F} \in [1, 2]\)) is a multiplier of reaction
efficiency based on expert opinion applied to species that burn with
more intensity than others, resulting from differences in fuel
chemistry. Flammability index of canopy and shrub strata are the result
of averaging the flammability of cohorts using loading as weights:

\begin{eqnarray}
\eta_{F, ca} &= \frac{\sum_{i}{w_{i,ca} \cdot \eta_{F}(SP_i)}}{\sum_{i}{w_{i,ca}}}\\
\eta_{F, sh} &= \frac{\sum_{i}{w_{i,sh} \cdot \eta_{F}(SP_i)}}{\sum_{i}{w_{i,sh}}}
\end{eqnarray}

where \(\eta_{F}(SP_i)\) is a species-specific flammability value.
Flamability index for other strata are set to default values:

\begin{eqnarray}
\eta_{F, he} &= \eta_{F, li} = 1.5 \\
\eta_{F, wo} &= 1.0
\end{eqnarray}

\subsubsection{\texorpdfstring{Reactive volume
(\(RV\))}{Reactive volume (RV)}}\label{reactive-volume-rv}

The volume per surface unit (\(m^3\cdot m^{-2}\)) that would be involved
in flaming combustion.

\begin{eqnarray}
RV_{sh} &=& w_{shrub}/\rho_{p, sh}\\
RV_{he} &=& w_{he}/\rho_{p, he}\\
RV_{wo} &=& w_{wo}/\rho_{p, wo}\\
RV_{li} &=& \min(w_{li}, w_{\max,li})/\rho_{p, li}
\end{eqnarray}

In the case of litter, the flame loading is limited by \(w_{\max,li}\),
the maximum loading that would be consumed in the flaming stage of
combustion, calculated as a weighted average of litter types:

\begin{equation}
 w_{\max,li} = \frac{\sum_{k}{ w_{\max,k}\cdot w_{li,k}}}{\sum_{k} {\cdot w_{li,k}}}
\end{equation}

where \(k\) indicates litter type. The maximum combustion loadings for
litter types are {[}\citet{Prichard2013}; Table 2{]}:

\begin{eqnarray}
w_{\max,shortneedlepine} &=& w_{\max,otherconifer} = 0.3248 kg \cdot m^{-2} = 2900 lb\cdot ac^{-1} \\
w_{\max,longneedlepine} &=& 0.6496 kg \cdot m^{-2}= 5800 lb\cdot ac^{-1} \\
w_{\max,hardwood} &=& 0.3472 kg \cdot m^{-2}= 3100 lb\cdot ac^{-1} 
\end{eqnarray}

\subsection{Unit conversion of fuel
characteristics}\label{unit-conversion-of-fuel-characteristics}

FCCS calculations employ empirical equations that were derived in
British units system. Hence, all the fuel characteristics and model
inputs that are in metric units have to be translated into British units
prior to fire behaviour calculations:

\begin{itemize}
\tightlist
\item
  Loading: \(1 kg\cdot m^{-2} = 0.204918 lb\cdot ft^{-2}\)
\item
  Depths: \(1m = 3.2808399ft\)
\item
  Particle density and bulk density:
  \(1 kg\cdot m^{-3} = 0.06242796 lb\cdot ft^{-3}\)
\item
  Particle volume and reactive volume:
  \(1 m^{3}\cdot m^{-2} = 3.2808399 ft^{3}\cdot ft^{-2}\)
\item
  Surface-to-area-volume ratio:
  \(1 m^{2}\cdot m^{-3} = 0.3048 ft^{2}\cdot ft^{-3}\)
\item
  Heat content: \(1kJ\cdot kg^{-1} = 0.429922614 Btu\cdot lb^{-1}\)
\item
  Wind speed: \(1 m \cdot s^{-1} = 2.23693629 mph\)
\end{itemize}

\section{Surface fire behavior}\label{surface-fire-behavior}

\subsection{\texorpdfstring{Surface rate of spread
(\(R\))}{Surface rate of spread (R)}}\label{surface-rate-of-spread-r}

In the \citet{Rothermel1972} model, surface rate of spread is defined as
the ratio of heat source (i.e.~the surface fire energy propagated to
unburned fuels) to surface fuel heat sink (i.e.~the energy required to
preheat fuels). Owing to the difference in packing ratio between the
litter stratum and the other surface fuels, litter-dominated fuelbeds
may have substantially different spread rates than other fuelbeds. For
this reason, in FCCS the rate of spread (in \(ft \cdot min^{-1}\)) is
calculated separately for litter stratum and the final rate of spread is
the maximum of the rate of spread of all surface fuels and that of the
litter stratum. Rate of spread is also limited to a maximum based in
windspeed and slope.

\begin{equation}
R = \min(WindSlopeCap, \max(R_{surf}, R_{litter}))
\end{equation}

The surface fuel and litter fuel rates of spread are given by the
application of Rothermel's \citeyearpar{Rothermel1972} equation to each
case:

\begin{eqnarray} 
R_{surf} &= \frac{I_{R,surf} \cdot \xi_{surf}\cdot (1 + \phi_W + \phi_S)}{q_{surf}}\\
R_{litter} &= \frac{I_{R,litter} \cdot \xi_{litter}\cdot (1 + \phi_W + \phi_S)}{q_{litter}}
\end{eqnarray}

where \(I_{R,surf}\) and \(I_{R,litter}\) are the reaction intensities
(in \(Btu \cdot ft^{-2} \cdot min^{-1}\)), \(\xi_{surf}\) and
\(\xi_{litter}\) are the propagating flux ratios, \(q_{surf}\) and
\(q_{litter}\) are the heat sinks. Finally, \(\phi_W\) and \(\phi_S\)
are the slope and wind modifiers. All of them are explained in the
following sections. The maximum rate of spread calculated from windspeed
and slope is:

\begin{equation}
WindSlopeCap = 88 \cdot U \cdot (1 + \phi_S)
\end{equation}

where \(U\) is windspeed (in \(mph\)) and \(88\) is a conversion factor
(from \(mph\) to \(ft/min\)).

\subsubsection{\texorpdfstring{Reaction intensity
(\(I_R\))}{Reaction intensity (I\_R)}}\label{reaction-intensity-i_r}

Reaction intensity of surface fuels (in
\(Btu \cdot ft^{-2} \cdot min^{-1}\)) is calculated as the sum of
component reaction intensities of the four different surface fuel
strata, whereas the reaction intensity in the litter uses this strata
alone:

\begin{eqnarray} 
I_{R,surf} &= I_{R, sh} + I_{R, he}+ I_{R, wo}+I_{R, li}\\
I_{R,litter} &= I_{R, li}
\label{eq:reactintensity}
\end{eqnarray}

Each component reaction intensity is calculated using:

\begin{eqnarray} 
I_{R,sh} &= (\eta_{\beta_{allsurf}'})^{A_{sh}}\cdot \Gamma_{\max, sh}'\cdot w_{sh} \cdot h_{sh} \cdot \eta_{M,sh}\cdot \eta_{K,sh}\cdot \eta_{F,sh}\\
I_{R,he} &= (\eta_{\beta_{lowsurf}'})^{A_{he}}\cdot \Gamma_{\max, he}'\cdot w_{he} \cdot h_{he} \cdot \eta_{M,he}\cdot \eta_{K,he}\cdot \eta_{F,he}\\
I_{R,wo} &= (\eta_{\beta_{lowsurf}'})^{A_{wo}}\cdot \Gamma_{\max, wo}'\cdot w_{wo} \cdot h_{wo} \cdot \eta_{M,wo}\cdot \eta_{K,wo}\cdot \eta_{F,wo}\\
I_{R,li} &= (\eta_{\beta_{litter}'})^{A_{li}}\cdot \Gamma_{\max, li}'\cdot w_{li} \cdot h_{li} \cdot \eta_{M,li}\cdot \eta_{K,li}\cdot \eta_{F,li}
\label{eq:reactintensitycomp}
\end{eqnarray}

In the above equations, \(w_{sh}\), \(w_{he}\), \(w_{wo}\) and
\(w_{li}\) are the loadings of the corresponding shrub, herb, woody and
litter strata, respectively. These quantities were defined in previous
sections, as were the corresponding low heat fuel contents (\(h_{sh}\),
\(h_{he}\), \(h_{wo}\) and \(h_{li}\)) and flammability indices
(\(\eta_{F,sh}\), \(\eta_{F,he}\), \(\eta_{F,wo}\) and \(\eta_{F,li}\)).
Mineral damping coefficient (\(\eta_{K}\); dimensionless) is set to the
same value (corresponding to the conventional value for silica-free ash
content of 1\%) for all strata:

\begin{equation}
\eta_{K,sh} = \eta_{K,he} =\eta_{K,wo} = \eta_{K,li} = 0.42
\end{equation}

In the following subsections, we describe the calculation of the
remaining variables for each stratum: reaction efficiency
(\(\eta_{\beta'}\)), Rothermel's \(A\) parameter, maximum reaction
velocity (\(\Gamma_{\max}'\)) and moisture damping coefficient
(\(\eta_{M}\)).

\subsubsection{\texorpdfstring{Reaction efficiency
(\(\eta_{\beta'}\))}{Reaction efficiency (\textbackslash{}eta\_\{\textbackslash{}beta'\})}}\label{reaction-efficiency-eta_beta}

Reaction efficiency (between 0 and 1) represents the damping effect of
inefficiently packed fuels in the reaction intensity. Because shrubs
rarely burn without lower surface fuels, the reaction efficiency of the
surface layer (\(\eta_{\beta_{allsurf}'}\)) includes shrubs, herbs and
woody fuels. Low surface fuels may carry flames without involving
shrubs, so are assumed to burn with a single reaction efficiency
(\(\eta_{\beta_{lowsurf}'}\)) determined by the combined characteristics
of herb and woody fuel strata. Both are calculated similarly:

\begin{eqnarray} 
\eta_{\beta_{allsurf}'} &= \beta_{allsurf}'\cdot e^{1- \beta_{allsurf}'}\\
\eta_{\beta_{lowsurf}'} &=\beta_{lowsurf}'\cdot e^{1- \beta_{lowsurf}'}
\label{eq:reacteff}
\end{eqnarray}

where \(\beta_{allsurf}'\) and \(\beta_{lowsurf}'\) are the relative
packing ratios corresponding to all surface fuels and low surface fuels,
respectively. Relative packing ratios (\(\beta'\); dimensionless) are
defined as the ratio of optimum depth (\(\delta_{opt}\)) to effective
depth (\(\delta_{eff}\)):

\begin{eqnarray} 
\beta_{allsurf}' &= \delta_{opt, allsurf} / \delta_{eff, allsurf} \\
\beta_{lowsurf}' &= \delta_{opt, lowsurf} / \delta_{eff, lowsurf}
\label{eq:relpacking}
\end{eqnarray}

Optimum depth is the depth (in \(ft\)) at which fuels are optimally
packed for maximum reaction intensity:

\begin{eqnarray}
\delta_{opt, allsurf} &= PV_{allsurf} +OptAirVol_{allsurf}\\
\delta_{opt, lowsurf} &= PV_{lowsurf} +OptAirVol_{lowsurf}
\end{eqnarray}

where \(PV_{allsurf}\) and \(PV_{lowsurf}\) are the volume of particles
(in \(ft^3 \cdot ft^{-2}\)) for all surface fuels and low surface fuels,
respectively, given by:

\begin{eqnarray}
PV_{allsurf} &= PV_{sh} + PV_{he} + PV_{wo}\\
PV_{lowsurf} &= PV_{he} + PV_{wo}
\end{eqnarray}

\(OptAirVol_{allsurf}\) and \(OptAirVol_{allsurf}\) are the volume of
air space (in \(ft^3 \cdot ft^{-2}\)) between fuel particles that would
result in maximum reaction intensity:

\begin{eqnarray}
OptAirVol_{allsurf} &= 45\cdot (RV_{sh} + RV_{he} + RV_{wo})\\
OptAirVol_{lowsurf} &= 45\cdot (RV_{he} + RV_{wo})
\end{eqnarray}

On the other hand, effective depths of all surface fuels and low surface
fuels (in \(ft\)) are calculated as their depth, weighted by the
reactive volume (and percentage cover in FCCS):

\begin{eqnarray}
\delta_{eff, allsurf} &= \frac{(RV_{sh}\cdot \delta_{sh}) +(RV_{he}\cdot \delta_{he}) + (RV_{wo}\cdot \delta_{wo})}{RV_{sh} +RV_{he}+RV_{wo}}\\
\delta_{eff, lowsurf} &= \frac{(RV_{he}\cdot \delta_{he}) + (RV_{wo}\cdot \delta_{wo})}{RV_{he}+RV_{wo}}
\end{eqnarray}

Reaction efficiency of the litter stratum is determined separately from
the other strata. It is defined as the average of reaction efficiency
across litter types, calculated using loadings as weights:

\begin{equation}
\eta_{\beta_{litter}'} = \frac{\sum_{k} {\eta_{\beta_{k}'}\cdot w_{li,k}}}{\sum_{k} {\cdot w_{li,k}}}
\end{equation}

where \(k\) indicates litter type. The reaction efficiencies of litter
types are {[}\citet{Prichard2013}; Table 2{]}:

\begin{eqnarray}
\eta_{\beta_{shortneedlepine}'} &= \eta_{\beta_{otherconifer}'} = 0.18\\
\eta_{\beta_{longneedlepine}'} &= 0.27\\
\eta_{\beta_{hardwood}'} &= 0.11
\end{eqnarray}

\subsubsection{Rothermel's A}\label{rothermels-a}

A dimensionless coefficient that modifies reaction's efficiency (eq.
\eqref{eq:reacteff}) to account for lower sensitivity of reaction
efficiency to relative packing ratio in flash fuels:

\begin{eqnarray}
A_{wo} &= A_{li} = 1.0\\
A_{sh} &= 133\cdot \sigma_{sh}^{-0.7913}\\
A_{he} &= 133\cdot \sigma_{he}^{-0.7913}
\end{eqnarray}

where \(\sigma_{sh}\) and \(\sigma_{he}\) have to be expressed in
\(ft^2\cdot ft^{-3}\); Values \(133\) and \(-0.7913\) are empirical
constants \citep{Rothermel1972}.

\subsubsection{\texorpdfstring{Maximum reaction velocity
(\(\Gamma_{\max}'\))}{Maximum reaction velocity (\textbackslash{}Gamma\_\{\textbackslash{}max\}')}}\label{maximum-reaction-velocity-gamma_max}

The reaction velocity (in \(min^{-1}\)) that would exist at optimum
fuelbed depth with no fuel moisture or mineral content.

\begin{eqnarray}
\Gamma_{\max, sh}' &= 9.495 \cdot \frac{\sigma_{sh}}{\sigma_{wo}}\\
\Gamma_{\max, he}' &= 9.495 \cdot \frac{\sigma_{he}}{\sigma_{wo}}\\
\Gamma_{\max, wo}' &= 9.495 \\
\Gamma_{\max, li}' &= 15 
\label{eq:maxreactvel}
\end{eqnarray}

where \(\sigma_{wo} = 488 ft^2\cdot ft^{-3} = 1601.05 m^2\cdot m^{-3}\)
is the surface-to-area-volume ratio typical of small woody fuels. In
\citet{Prichard2013} \(\sigma_{sh}\) is defined as the average of shrub
foliar surface-to-area-volume ratio and \(\sigma_{wo}\), but in our case
\(\sigma_(SP_i)\) for each species includes both leaves and small
branches. Eq. \eqref{eq:maxreactvel} represent a significant departure
from \citet{Rothermel1972} maximum reaction velocity, and are also
different from \citet{Sandberg2007}.

\subsubsection{\texorpdfstring{Moisture damping coefficient
(\(\eta_M\))}{Moisture damping coefficient (\textbackslash{}eta\_M)}}\label{moisture-damping-coefficient-eta_m}

Moisture damping reduces reaction velocity and hence reaction intensity
(eq. \eqref{eq:reactintensity}). It is calculated for each stratum using
the following regression equations:

\begin{eqnarray}
\eta_{M, live, sh} &= \left[1-2.59\cdot \left(\frac{M_{live, sh}}{X_{live, sh}}\right)\right] +\left[ 5.11\cdot \left(\frac{M_{live, sh}}{X_{live, sh}}\right)^2\right]-\left[ 3.52\cdot \left(\frac{M_{live, sh}}{X_{live, sh}}\right)^3\right] \\
\eta_{M, dead, sh} &= \left[1-2.59\cdot \left(\frac{M_{dead, sh}}{X_{dead, sh}}\right)\right] +\left[ 5.11\cdot \left(\frac{M_{dead, sh}}{X_{dead, sh}}\right)^2\right]-\left[ 3.52\cdot \left(\frac{M_{dead, sh}}{X_{dead, sh}}\right)^3\right] \\
\eta_{M, live, he} &= \left[1-2.59\cdot \left(\frac{M_{live, he}}{X_{live, he}}\right)\right] +\left[ 5.11\cdot \left(\frac{M_{live, he}}{X_{live, he}}\right)^2\right]-\left[ 3.52\cdot \left(\frac{M_{live, he}}{X_{live, he}}\right)^3\right] \\
\eta_{M, dead, he} &= \left[1-2.59\cdot \left(\frac{M_{dead, he}}{X_{dead, he}}\right)\right] +\left[ 5.11\cdot \left(\frac{M_{dead, he}}{X_{dead, he}}\right)^2\right]-\left[ 3.52\cdot \left(\frac{M_{dead, he}}{X_{dead, he}}\right)^3\right] \\
\eta_{M, wo} &= \left[1-2.59\cdot \left(\frac{M_{wo}}{X_{wo}}\right)\right] +\left[ 5.11\cdot \left(\frac{M_{wo}}{X_{wo}}\right)^2\right]-\left[ 3.52\cdot \left(\frac{M_{wo}}{X_{wo}}\right)^3\right] \\
\eta_{M, li} &= \left[1-2.59\cdot \left(\frac{M_{li}}{X_{li}}\right)\right] +\left[ 5.11\cdot \left(\frac{M_{li}}{X_{li}}\right)^2\right]-\left[ 3.52\cdot \left(\frac{M_{li}}{X_{li}}\right)^3\right] 
\label{eq:moistdamp}
\end{eqnarray}

where moisture contents of extinctions were arbitrarily set to
\(X_{dead, sh} = X_{dead, he} X_{wo} = X_{li} = 25\),
\(X_{live, sh} = 180\) and \(X_{live, he} = 120\) in Sandberg et al.
(2007). As it can be seen in the equations above, in the case of shrub
and herb strata, moisture damping of live and dead fuels are
differentiated. Average values are found after accounting for the
proportion of live and dead material:

\begin{eqnarray}
\eta_{M, sh} &= \eta_{M, live, sh} \cdot (1 - P_{dead, sh})+ \eta_{M, dead, sh} \cdot P_{dead, sh}\\
\eta_{M, he} &= \eta_{M, live, he} \cdot (1 - P_{dead, he})+ \eta_{M, dead, he} \cdot P_{dead, he}
\end{eqnarray}

\subsubsection{\texorpdfstring{Propagating flux ratio
(\(\xi\))}{Propagating flux ratio (\textbackslash{}xi)}}\label{propagating-flux-ratio-xi}

The propagating flux ratio (dimensionless) is the proportion of the
reaction intensity (eq. \eqref{eq:reactintensity}) that contributes to the
forward rate of spread, estimated using an empirical regression:

\begin{eqnarray}
\xi_{surf} &= 0.03 + 2.5 \cdot \min \left[0.06, \frac{RV_{sh}+RV_{he}+RV_{wo}+RV_{li}}{\delta_{surfheatsink}} \right]\\
\xi_{litter} &= 0.03 + 2.5 \cdot \min \left[0.06, \frac{RV_{li}}{\delta_{li}} \right]
\end{eqnarray}

where \(\delta_{surfheatsink}\) is the depth of surface heat sink (in
\(ft\)), which in \citet{Prichard2013} is calculated as the sum of
strata depths weighted by their relative cover. In our case we weighted
stratum depths as in the calculation of effective depth
(\(\delta_{eff, allsurf}\)), but considering all four strata:

\begin{equation}
\delta_{surfheatsink} = \frac{(RV_{sh}\cdot \delta_{sh}) +(RV_{he}\cdot \delta_{he}) + (RV_{wo}\cdot \delta_{wo})+ (RV_{li}\cdot \delta_{li})}{RV_{sh} +RV_{he}+RV_{wo}+RV_{li}}
\end{equation}

\subsection{\texorpdfstring{Heat sink
(\(q\))}{Heat sink (q)}}\label{heat-sink-q}

Like reaction intensity, the heat sink term (in \(Btu \cdot ft^{-3}\))
of the rate of spread equation is calculated in FCCS for each fuel
stratum and then summed:

\begin{eqnarray}
q_{surf} &= q_{sh}+q_{he}+q_{wo}+q_{li}\\
q_{litter} &= q_{li}
\label{eq:heatsink}
\end{eqnarray}

where the heat sink for each stratum is:

\begin{eqnarray}
q_{sh} &= \eta_{\beta_{surf}'}\cdot \frac{RV_{sh}\cdot \rho_{p,sh}\cdot Qig_{sh}}{\min(\delta_{sh}, 1ft)}\\
q_{he} &= \eta_{\beta_{lowsurf}'}\cdot \frac{RV_{he}\cdot \rho_{p,he}\cdot Qig_{he}}{\min(\delta_{he}, 1ft)}\\
q_{wo} &= \eta_{\beta_{lowsurf}'}\cdot \frac{RV_{wo}\cdot \rho_{p,wo}\cdot Qig_{wo}}{\min(\delta_{wo}, 1ft)}\\
q_{li} &= \eta_{\beta_{li}'}\cdot \frac{RV_{li}\cdot \rho_{p,li}\cdot Qig_{li}}{\min(\delta_{li}, 1ft)}
\label{eq:heatsinkstrat}
\end{eqnarray}

Where \(\rho_{p,sh}\), \(\rho_{p,he}\), \(\rho_{p,wo}\) and
\(\rho_{p,li}\) are the particle densities (in \(lb\cdot ft^{-3}\)) of
each fuel stratum; and \(RV_{sh}\), \(RV_{he}\), \(RV_{wo}\), and
\(RV_{li}\) are the reactive volumes of each fuel stratum. Unlike in
Sandberg et al. (2007), the calculated heat sink is corrected by the
reaction-efficiency term (\(\eta_{\beta_{surf}'}\),
\(\eta_{\beta_{lowsurf}'}\) or \(\eta_{\beta_{li}'}\)), and the
effective depth of each stratum included is limited to 1ft, based on the
assumption that it is not necessary to preheat more than one 1ft of
depth within a stratum to achieve ignition.

Heat of pre-ignition (\(Qig\); in \(Btu \cdot lb^{-1}\)) is the amount
of heat required to ignite \(1 lb\) of fuel. It is calculated by stratum
as a weighted average of live and dead fuels in shrubs and herbs.

\begin{eqnarray}
Qig_{sh} &= Qig_{live, sh} \cdot (1 - P_{dead, sh})+ Qig_{dead, sh} \cdot P_{dead, sh}\\
Qig_{he} &= Qig_{live, he} \cdot (1 - P_{dead, he})+ Qig_{dead, he} \cdot P_{dead, he}
\end{eqnarray}

Whereas \(Qig_{live, sh}\) and \(Qig_{live, he}\) are corrected by fuel
moisture, \(Qig_{dead, sh}\), \(Qig_{dead, he}\) and the other strata
(\(Qig_{wo}\) and \(Qig_{li}\)) are assumed a constant value:

\begin{eqnarray}
Qig_{live, sh} &= 250 + (V\cdot (M_{live, sh}/100))\\
Qig_{live, he} &= 250 + (V\cdot (M_{live, he}/100))\\
Qig_{dead, sh} &= Qig_{dead, he} = Qig_{wo} = Qig_{li} = 250
\end{eqnarray}

where \(250 Btu/lb\) is the heat of preignition of dry cellulose and
\(V = 1116 Btu/lb\) is the latent heat of vaporization.
\subsection{Wind and slope coefficients ($\phi_W$ and $\phi_S$)} Wind
and slope coefficients modify the heat source term of the rate of spread
equation. Owing to differences in fuel characteristics and boundary
conditions between the litter stratum and other surface fuel strata, in
FCCS wind and slope coefficients are calculated separately for the
litter stratum. The wind and slope coefficients terms in the rate fo
spread equation are a weighted average of litter and surface wind and
slope coefficients using the relative contribution to reaction intensity
as weights:

\begin{eqnarray}
\phi_W &= (1 - I_{R, litter}/I_{R, surf})\cdot \phi_{W, surf} + (I_{R, litter}/I_{R, surf})\cdot \phi_{W, litter}\\
\phi_S &= (1 - I_{R, litter}/I_{R, surf})\cdot \phi_{S, surf} + (I_{R, litter}/I_{R, surf})\cdot \phi_{S, litter}
\end{eqnarray}

Wind coefficients are calculated using:

\begin{eqnarray}
\phi_{W, surf} &= 8.8 \cdot \beta_{surf}'^{-E}\cdot (U/BMU)^B\\
\phi_{W, litter} &= 8.8 \cdot \beta_{litter}^{-E}\cdot (U/BMU)^B
\end{eqnarray}

where \(U\) is the input midflame windspeed (in \(ft\cdot min^{-1}\)),
\(BMU=352 ft\cdot min^{-1}\) is the benchmark midflame windspeed,
\(\beta_{surf}'\) is the relative packing ratio (eq.
\ref{eq:relpacking}), \(B\) is the exponential response of wind
coefficient to windspeed (\(B=1.2\) in \citet{Sandberg2007}), and \(E\)
is the exponential term representing the mild effect of large fuels in
reducing the accelerating effect of wind on fire spread by attenuating
wind flow, given by:

\begin{equation}
E = 0.55 - 0.2 \cdot \frac{FAI_{sh}+FAI_{he}}{FAI_{sh}+FAI_{he}+FAI_{wo}}
\end{equation}

\(E\) is assumed to be the same for both all surface fuels and litter
fuels.

Slope coefficients are calculated using the empirical equation of
\citet{Rothermel1972}, applied to all surface fuels and litter fuels:

\begin{eqnarray}
\phi_{S, surf} &= 5.275 \cdot (S/100)^{2}\cdot (\beta_{sh}+\beta_{he}+\beta_{wo})^{-0.3}\\
\phi_{S, litter} &= 5.275 \cdot (S/100)^{2}\cdot \beta_{li}^{-0.3}
\end{eqnarray}

where \(S\) is the slope (in percent) and \(\beta\) is the packing ratio
(not relative!) of fuels.

\subsection{\texorpdfstring{Fireline intensity (\(I_B\)) and flame
length
(\(FL\))}{Fireline intensity (I\_B) and flame length (FL)}}\label{fireline-intensity-i_b-and-flame-length-fl}

Byram's fireline intensity (\(I_B\)) is the rate of heat release per
unit of fire edge (in \(Btu\cdot ft^{-1} \cdot min^{-1}\)), and in FCCS
is calculated as \citep{Albini1976}:

\begin{equation}
I_B = I_{R,surf} \cdot (R \cdot t_R)
\end{equation}

where \(I_{R,surf}\) is the surface reaction intensity, \(R\) is the
rate of spread and \(t_R\) is the flame residence time, which is defined
as the time (in \(min\)) fuels contribute to propagating flux and is
estimated as \citet{Albini1976}:

\begin{equation}
t_R = 192 \cdot \frac{(I_{R,sh}\cdot RT_{sh})+(I_{R,he}\cdot RT_{he})+(I_{R,wo}\cdot RT_{wo})+(I_{R,li}\cdot RT_{li})}{I_{R,surf}}
\end{equation}

where \(RT\) is the reaction thickness, the approximate thickness (in
\(ft\)) of a fuel element shell that contributes to reaction intensity.
In FCCS, reaction thickness is estimated as \(RT = 0.0028 ft\) for
thermally thick fuel elements \citep{Sandberg2007}. When the diameter of
a fuel element is less than twice the reaction thickness, the entire
fuel element contributes to reaction intensity. Reaction thickness
values for each stratum are given by:

\begin{eqnarray}
RT_{sh} &= \min(0.0028, 2/\sigma_{sh})\\
RT_{he} &= \min(0.0028, 2/\sigma_{he})\\
RT_{wo} &= \min(0.0028, 2/\sigma_{wo})\\
RT_{li} &= \min(0.0028, 2/\sigma_{li})
\end{eqnarray}

Flame length is defined as the distance (in \(ft\)) between the flame
tip and the midpoint of the flame depth at the base of the flame, and is
calculated \citep{Byram1959}:

\begin{equation}
FL = 0.45 \cdot (I_B/60)^{0.46}
\end{equation}

where 60.0 is a factor to convert from
\(Btu\cdot ft^{-1} \cdot min^{-1}\) to
\(Btu\cdot ft^{-1} \cdot s^{-1}\).

\section{Crown fire behavior}\label{crown-fire-behavior}

Crown fire behavior is difficult to model and actual rates of spread are
not possible to predict. Here we mainly follow the approach given in
FCCS \citep{Prichard2013}, although in our case the canopy is not
subdivided into layers (overstory, midstory and understory).

\subsection{Crown fire rate of spread ($R_{crown}$)}

The rate of spread of crown fires is estimated by using a modification
of Rothermel's equation:

\begin{equation}
R_{crown} = \frac{I_{R,crown} \cdot \xi_{crown}\cdot WAF}{q_{crown}} = \frac{(I_{R,surf}+I_{R,ca}) \cdot \xi_{crown}\cdot WAF}{q_{surf}+q_{ca}}
\end{equation}

where \(I_{R,surf}\) is the surface reaction intensity, \(I_{R,ca}\) is
the canopy reaction intensity, \(\xi_{crown}\) is the propagating flux
ratio in the canopy, \(q_{surf}\) is the surface heat sink and
\(q_{ca}\) is the canopy heat sink. Note that reaction intensities and
heat sinks of canopy and surface fuels are added for the application of
Rothermel's equation. Other modifications include the exclusion of slope
effects and the consideration of wind effects through a wind adjusment
factor (WAF).

Crown propagating flux ratio (\(\xi_{crown}\); in \(Btu \cdot ft^{-3}\))
represents the proportion of the crown reaction intensity that
contributes to crown fire's forward rate of spread:

\begin{equation}
\xi_{crown} = 1 - e^{\left(-\frac{FAI_{ca}}{4 \cdot \delta_{ca}}\right)}
\end{equation}

where \(FAI_{ca}\) is the fuel area index of the canopy, and
\(\delta_{ca}\) is the canopy depth (in \(ft\)). Wind adjustment factor
(\(WAF\)) is defined as:

\begin{equation}
WAF = \frac{U/\sqrt{U^2+VS^2}}{BMU/\sqrt{BMU^2+VS^2}}
\end{equation}

where \(U\) is the input (midflame) windspeed (in
\(ft \cdot min^{-1}\)), \(BMU = 352 ft \cdot min^{-1}\) is the benchmark
windspeed and \(VS = 900 ft \cdot min^{-1}\) is the vertical stack
velocity.

The following two subsections detail the calculation of canopy reaction
intensity (\(I_{R, ca}\)) and canopy heat sink (\(q_{ca}\)).

\subsection{\texorpdfstring{Canopy reaction intensity
(\(I_{R, ca}\))}{Canopy reaction intensity (I\_\{R, ca\})}}\label{canopy-reaction-intensity-i_r-ca}

Reaction intensity of canopy fuels (in
\(Btu \cdot ft^{-2} \cdot min^{-1}\)) is estimated as:

\begin{equation}
I_{R,ca} = (\eta_{\beta_{ca}'})^{A_{ca}}\cdot \Gamma_{\max, ca}'\cdot w_{ca} \cdot h_{ca} \cdot \eta_{M,ca}\cdot \eta_{K,ca}\cdot \eta_{F,ca}
\end{equation}

where \(A_{ca} = 133\cdot \sigma_{ca}^{-0.7913}\) is Rothermel's A
coefficient, \(\Gamma_{\max, ca}' = 15 min^{-1}\) is the maximum
reaction velocity of the canopy, \(w_{ca}\) is the loading of canopy
fuels (in \(lb \cdot ft^{-2}\)), \(h_{ca}\) is the heat content of the
canopy fuels (in \(Btu \cdot lb^{-1}\)), \(\eta_{K,ca}=0.42\) is the
mineral damping coefficient and \(\eta_{F,ca}\) is the flammability
index of the canopy stratum. Moisture damping coefficient for the canopy
(\(\eta_{M,ca}\)) is estimated as done for shrub and herb strata:

\begin{eqnarray}
\eta_{M, live, ca} &= \left[1-2.59\cdot \left(\frac{M_{live, ca}}{X_{live, ca}}\right)\right] +\left[ 5.11\cdot \left(\frac{M_{live, ca}}{X_{live, ca}}\right)^2\right]-\left[ 3.52\cdot \left(\frac{M_{live, ca}}{X_{live, ca}}\right)^3\right] \\
\eta_{M, dead, ca} &= \left[1-2.59\cdot \left(\frac{M_{dead, ca}}{X_{dead, ca}}\right)\right] +\left[ 5.11\cdot \left(\frac{M_{dead, ca}}{X_{dead, ca}}\right)^2\right]-\left[ 3.52\cdot \left(\frac{M_{dead, ca}}{X_{dead, ca}}\right)^3\right] \\
\eta_{M, ca} &= \eta_{M, live, ca} \cdot (1 - P_{dead, ca})+ \eta_{M, dead, ca} \cdot P_{dead, ca}
\end{eqnarray}

where moisture contents of extinctions were arbitrarily set to
\(X_{dead, ca} = 25\) and \(X_{live, ca} = 180\).

The reaction efficiency in the canopy (\(\eta_{\beta_{canopy}'}\))
represents the damping effect of inefficiently packed fuels in the
canopy:

\begin{equation}
\eta_{\beta_{canopy}'} =\beta_{canopy}'\cdot e^{1- \beta_{canopy}'}
\end{equation}

where \(\beta_{canopy}'\) is the relative packing ratio in the canopy:

\begin{equation}
\beta_{canopy}' = \delta_{opt, canopy} / \delta_{eff, canopy}
\end{equation}

where the effective depth is \(\delta_{eff, canopy}=\delta_{ca}\) (in
\(ft\)) and the optimum canopy depth is calculated using:

\begin{equation}
\delta_{opt, canopy} = 0.4 \cdot FAI_{ca} + \beta_{ca} \cdot (\delta_{ca} \cdot C_{ca}/100)
\end{equation}

where \(C_{ca}\) is the percent cover of the canopy, \(FAI_{ca}\) is the
fuel area index of the canopy, \(\beta_{ca}\) is the packing ratio of
canopy fuels and \(\delta_{ca}\) is the canopy depth (in \(ft\)).

\subsection{\texorpdfstring{Canopy heat sink
(\(q_{ca}\))}{Canopy heat sink (q\_\{ca\})}}\label{canopy-heat-sink-q_ca}

Canopy heat sink (in \(Btu \cdot ft^{-3}\)) is estimated using:

\begin{equation}
q_{ca} = \frac{0.5 \cdot FAI_{ca} \cdot RT_{ca} \cdot \rho_{p, ca} \cdot Qig_{ca}}{(C_{ca}/100)\cdot \delta_{ca}}
\end{equation}

where \(C_{ca}\) is the percent cover of the canopy, \(FAI_{ca}\) is the
fuel area index of the canopy, \(RT_{ca} = \min(0.0028, 2/\sigma_{ca})\)
is the reaction thickness of the canopy stratum (in \(ft\)),
\(\rho_{p, ca}\) is the particle density of the canopy (in
\(lb \cdot ft^{-3}\)), \(\delta_{ca}\) is the canopy depth (in \(ft\))
and \(Qig_{ca}\) is the heat of pre-ignition of the canopy stratum (in
\(Btu \cdot lb^{-1}\)), which is calculated as a weighted average of
live and dead fuels:

\begin{eqnarray}
Qig_{live, ca} &= 250 + (V\cdot (M_{live, ca}/100))\\
Qig_{dead, ca} &= 250\\
Qig_{ca} &= Qig_{live, ca} \cdot (1 - P_{dead, ca})+ Qig_{dead, ca} \cdot P_{dead, ca}
\end{eqnarray}

where \(250 Btu/lb\) is the heat of preignition of dry cellulose and
\(V = 1116 Btu/lb\) is the latent heat of vaporization of water.

\subsection{\texorpdfstring{Fireline intensity (\(I_{B,crown}\)) and
flame length
(\(FL_{crown}\))}{Fireline intensity (I\_\{B,crown\}) and flame length (FL\_\{crown\})}}\label{fireline-intensity-i_bcrown-and-flame-length-fl_crown}

Byram's fireline intensity for crown fires is estimated using:

\begin{equation}
I_{B,crown} = I_{R,crown} \cdot (R_{crown} \cdot t_{R,crown})
\end{equation}

where \(I_{R,crown}\) is the crown reaction intensity (i.e.~the sum of
canopy and surface reaction intensities), \(R_{crown}\) is the rate of
crown fire spread and \(t_{R,crown}\) is the flame residence time,
estimated as:

\begin{equation}
t_R = 192 \cdot RT_{ca}
\end{equation}

where \(RT_{ca} = \min(0.0028, 2/\sigma_{ca}\) is the reaction thickness
of the canopy. As for surface fires, flame length is calculated using:

\begin{equation}
FL_{crown} = 0.45 \cdot (I_{B,crown}/60)^{0.46}
\end{equation}

where 60.0 is a factor to convert from
\(Btu\cdot ft^{-1} \cdot min^{-1}\) to
\(Btu\cdot ft^{-1} \cdot s^{-1}\).

\section{Fire potentials}\label{fire-potentials}

\subsection{Surface fire behavior
potentials}\label{surface-fire-behavior-potentials}

The \textbf{surface fire behavior potential} (\(SFP\); between 0 and 9)
is an index defined as the maximum of spread potential (\(SP\)) and
flame length (\(FL\)) potential indices (both between 0 and 9):

\begin{equation}
SFP = \max(SP, FP)
\end{equation}

\emph{Spread potential} is derived from \(R\) (in \(ft\cdot min^{-1}\)),
and \emph{flame length potential} is derived from \(FL\) (in \(ft\)),
both quantities being calculated at benchmark environmental conditions:

\begin{eqnarray}
SP &= \min \left[ 9, R^{1/2}\right] \\
FP &= \min \left[ 9, FL^{1/2}\right] 
\end{eqnarray}

\subsection{Crown fire behavior
potentials}\label{crown-fire-behavior-potentials}

The \emph{crown fire summary potential} (\(CPF\)) combines three
subpotentials into a single index value between 0 and 9. It places more
emphasis on crown fire initiation (\(IC\)) and rate of spread (\(RC\))
than to crown-to-crown transmissivity (\(TC\)):

\begin{equation}
CFP = 0.4286 \cdot (IC+(TC/3)+RC)
\end{equation}

where \(0.4286\) is used to limit \(CPF\) between 0 and 9.

The \emph{crown fire initiation potential} (\(IC\)) represents the
likelihood of a surface fire torching into single or multiple trees.
\(IC\) is based on the work by Van Wagner. If \(FAI_{ca} = 0\) then
\(IC = 0\). Otherwise it is calculated as:

\begin{equation}
IC = \min \left[ 9, 4 \cdot \left(\frac{I_B/60}{I_c}\right)^{0.2}\right]
\end{equation}

where \(I_B\) is the surface fireline intensity (60 is used to convert
it to \(Btu\cdot ft^{-1} \cdot s^{-1}\)) and \(I'\) is Van Wagner's
critical fireline intensity \citep{Scott2002}:

\begin{equation}
I_c = 0.288894658 \cdot \left[ 0.01\cdot (H_{gap}/100) \cdot (460 +25.9\cdot M_{live, ca})\right]^{1.5}
\end{equation}

where \(M_{live, ca}\) is the moisture content of the canopy (in percent
of dry weight), 0.288894658 is used to convert from
\(kJ \cdot m^{-1}\cdot s^{-1}\) to \(Btu \cdot ft^{-1}\cdot s^{-1}\) and
\(H_{gap}\) is the canopy gap (in \(cm\)) determined from the analysis
of the bulk density profile.

The \emph{crown-to-crown transmittivity potential} (\(TC\)) is set to
zero if \(FAI_{ca} < TFAI/(3\cdot \pi)\), where \(TFAI\) is a threshold
for FAI calculated as:

\begin{equation}
TFAI = A_{q}\cdot e^{-0.0019 \cdot U}
\end{equation}

where \(A_{q} = 3.2868\) if \(\sigma_{ca} > 2000 ft^{2}\cdot ft^{-3}\)
and \(A_{q} = 2.6296\) otherwise. If \(FAI_{ca} > TFAI/(3\cdot \pi)\),
then \(TC\) is calculated as:

\begin{equation}
TC = \min \left[ 9, 10 \cdot TC_q \right]
\end{equation}

where \(TC_q\) is the efficiency of crown-to-crown heat transfer, as a
proportion of maximum efficiency at 100 \% canopy cover:

\begin{equation}
TC_q = \frac{(\max(0.0, C_{ca}\cdot WAF - 40))^{0.3}}{(100 \cdot WAF -40)^{0.3}}
\end{equation}

In this last equation 40 represents the threshold of canopy cover
necessary to initiate dependent crown spread, and 0.3 is a coefficient
describing the assumed effect of crown cover on transmittivity at
benchmark windspeed. The canopy adjustment ratio \(WAF\) is added to
modulate transmittivity depending on windspeed.

Finally, the \emph{crown fire rate of spread potential} (\(RC\)) is
defined as:

\begin{equation}
RC = \min \left[ 9, 2.5 \cdot R_{crown}^{1/e} \right]
\end{equation}

where \(R_{crown}\) is the rate of spread (in \(ft \cdot min^{-1}\)) of
the crown fire.

\section{Unit conversion of outputs}\label{unit-conversion-of-outputs}

The following factors are used to express fire behavior outputs to
metric units:

\begin{itemize}
\tightlist
\item
  \emph{Fire spread rates}:
  \(1 ft\cdot min^{-1} = 0.3048 m\cdot min^{-1}\)
\item
  \emph{Flame length}: \(1 ft = 0.3048 m\)
\item
  \emph{Reaction intensity}:
  \(1 Btu\cdot ft^{-2} \cdot min^{-1} = 11.3484 kJ \cdot m^{-2}\cdot min^{-1}\)
\item
  \emph{Heat sink}: \(1 Btu\cdot ft^{-3} = 37.2589458 kJ \cdot m^{-3}\)
\item
  \emph{Fireline intensity}:
  \(1 Btu\cdot ft^{-1} \cdot min^{-1} = 0.0576911555 kW\cdot m^{-1}\)
\end{itemize}

\appendix


\chapter{Model parametrization}\label{model-parametrization}

Package \textbf{medfate} provides routines to estimate them from a
minimum set of input parameters. The whole process of estimation of
those parameters is done automatically in functions \texttt{spwbInput()}
and \texttt{forest2spwbInput()}, with the user controlling the process
through the species parameter table input (e.g., \texttt{SpParamsMED})
an object \texttt{control} (see default values in
\texttt{defaultControl()}). In the following we detail the calculations
and present individual functions that perform partial calculations.

\section{Plant Hydraulics}\label{plant-hydraulics-1}

\subsection{Vulnerability curves}\label{vulnerability-curves-1}

Leaf and xylem vulnerability curves are often described using
\(\Psi_{50}\), the water potential at which hydraulic conductance is
half of maximum. As noted above, parameter \(d\) in eq.
\eqref{eq:xylemvulnerability} is the water potential \(\Psi\) at which
\(k_{x}(\Psi)/k_{x,max} = e^{-1} = 0.367\) (and the same for eq.
\eqref{eq:leafvulnerability}). Hence, the two definitions do not match.
Using the definition of \(\Psi_{50}\) in eq. \eqref{eq:xylemvulnerability}
we have:

\begin{equation}
0.5 = e^{-((\Psi_{50}/d)^c)}
\end{equation}

from which we obtain that the value for parameter \(d\) should be:

\begin{equation}
d = \frac{\Psi_{50}}{(-ln(0.5))^{1/c}}= \frac{\Psi_{50}}{0.69314^{1/c}}
\end{equation}

Hence, this operation should be used when specifying this parameter from
\(\Psi_{50}\). Vulnerability curves for root xylem are less common than
for stem xylem. If these values are missing, functions
\texttt{spwbInput()} and \texttt{forest2spwbInput()} will use for \(c\)
the same value as in stems, and for \(d\) half the value of that of
stems \citep{Sperry2016}. If the values for leaves are missing,
initialization functions will use for \(c\) the same value as in stems,
and for \(d\) 66\% of the value for stems.

Rhizosphere conductance is regulated in the model using the van
Genuchten function given in eq. \eqref{eq:rhizovulnerability}, and
parameters \(n\) and \(\alpha\) for each soil layer were already
available from soil initialization (i.e.~function \texttt{soil()}):

\begin{Shaded}
\begin{Highlighting}[]
\NormalTok{s =}\StringTok{ }\KeywordTok{soil}\NormalTok{(}\KeywordTok{defaultSoilParams}\NormalTok{(}\DecValTok{3}\NormalTok{))}
\NormalTok{s}\OperatorTok{$}\NormalTok{VG_n}
\end{Highlighting}
\end{Shaded}

\begin{verbatim}
## [1] 1.267359 1.303861 1.303861
\end{verbatim}

\begin{Shaded}
\begin{Highlighting}[]
\NormalTok{s}\OperatorTok{$}\NormalTok{VG_alpha}
\end{Highlighting}
\end{Shaded}

\begin{verbatim}
## [1] 147.32468  89.16112  89.16112
\end{verbatim}

Aboveground and belowground stem maximum conductance values at the plant
level (\(k_{s, max}\) and \(k_{r, max}\)) will not be normally available
and the same for the rhizosphere (\(k_{rh, max}\)).

\subsection{Leaf maximum conductance}\label{leaf-maximum-conductance}

Leaf maximum conductance (\(k_{l, max}\), in
\(mmol \cdot m^{-2} \cdot s^{-1} \cdot MPa^{-1}\)) is an input parameter
that should be provided for each species. When missing, leaf maximum
hydraulic conductance is assumed \(k_{l, max}=6\) for conifers and
\(k_{l, max}=8\) for angiosperms \citep{Sack2006}.

\subsection{Stem xylem maximum
conductance}\label{stem-xylem-maximum-conductance}

Estimation of maximum stem conductance (\(k_{s,max}\), in
\(mmol \cdot m^{-2} \cdot s^{-1} \cdot MPa^{-1}\)) is done by function
\texttt{hydraulics\_maximumStemHydraulicConductance()} and follows the
work by \citet{Savage2010}, \citet{Olson2014} and
\citet{Christoffersen2016}. Calculations are based on tree height and
two species-specific parameters: maximum sapwood reference conductivity
(\(K_{s,max,ref}\)) and the ratio of leaf area to sapwood area
(\(A_{l}/A_{s}\); \texttt{Al2As} in \texttt{SpParamsMED}), i.e.~the
inverse of the Huber value \(H_v\).

The reference value for maximum sapwood conductivity \(K_{s,max,ref}\)
is assumed to have been measured on a \emph{terminal branch} of a plant
of known height \(H_{ref}\). If our target plant is very different in
height, the conduits of terminal branches will have different radius and
hence conductivity. We correct the reference conductivity to the target
plant height using the following empirical relationship, developed by
\citet{Olson2014} between tree height and diameter of conduits for
angiosperms and the equation described by \citet{Christoffersen2016}:

\begin{eqnarray}
2 \cdot r_{int,H}&=& 10^{1.257+(0.24\cdot log_{10}(H))} \\
2 \cdot r_{int,ref}&=&10^{1.257+(0.24\cdot log_{10}(H_{ref}))}\\
K_{s,max,cor}&=&K_{s,max,ref}\cdot (r_{int,H}/r_{int,ref})^{2}
\end{eqnarray}

Where \(r_{int,H}\) is the radius of conduits for a terminal branch of a
tree of height \(H\) and \(r_{int,ref}\) is the corresponding radius for
a tree of height \(H_{ref}\) (\(H\) and \(H_{ref}\) are measured in m).
The form of the empirical relationship by \citet{Olson2014} is:

\begin{center}\includegraphics{medfatebook_files/figure-latex/unnamed-chunk-80-1} \end{center}

Let's consider an example for a \emph{Quercus ilex} target tree of 4m
height and where species-specific conductivity \(K_{s,max,ref} = 0.77\)
is the apical value for trees of \(H_{ref} = 6.6\) m (in
\texttt{medfate}, values of \(H_{ref}\) are taken from median height
values; see parameter \texttt{Hmed} in \texttt{SpParamsMED}). The
corrected conductivity for a tree of height 4 m will be a bit lower than
that of the reference height:

\begin{Shaded}
\begin{Highlighting}[]
\NormalTok{xylem_kmax =}\StringTok{ }\FloatTok{0.77}
\NormalTok{H =}\StringTok{ }\DecValTok{400} \CommentTok{# in cm}
\NormalTok{Href =}\StringTok{ }\DecValTok{660} \CommentTok{# in cm}
\NormalTok{f =}\StringTok{ }\KeywordTok{hydraulics_referenceConductivityHeightFactor}\NormalTok{(Href, H);}
\NormalTok{f}
\end{Highlighting}
\end{Shaded}

\begin{verbatim}
## [1] 0.7863352
\end{verbatim}

\begin{Shaded}
\begin{Highlighting}[]
\NormalTok{xylem_kmax_cor =}\StringTok{ }\NormalTok{xylem_kmax }\OperatorTok{*}\StringTok{ }\NormalTok{f}
\NormalTok{xylem_kmax_cor}
\end{Highlighting}
\end{Shaded}

\begin{verbatim}
## [1] 0.6054781
\end{verbatim}

Once the reference conductivity is corrected, the maximum stem
conductance without accounting for conduit taper is:

\begin{equation}
k_{s,max, notaper}=\frac{1000}{0.018} \frac{K_{s,max,cor}\cdot A_{s}}{H\cdot A_{l}}
\end{equation}

where \(H\) is the tree height (here in m), \(A_{s}\) is the sapwood
area, \(A_{l}\) is the leaf area and 1000/0.018 is a factor used to go
from kg to mmol. The ratio \(A_{l}/A_{s} = 1/H_v\) is a fixed species
parameter in soil water balance calculations (see parameter
\texttt{Al2As}), but becomes variable when simulating plant growth.
Let's assume that \emph{Quercus ilex} the leaf to sapwood area ratio is
\(A_{l}/A_{s} = 2512\). The maximum (leaf-specific) stem conductance
without taper (\(k_{s, max, notaper}\)) for the tree of 4 m height is
then:

\begin{Shaded}
\begin{Highlighting}[]
\NormalTok{Al2As =}\StringTok{ }\DecValTok{2512} 

\NormalTok{kstemmax =}\StringTok{ }\KeywordTok{hydraulics_maximumStemHydraulicConductance}\NormalTok{(xylem_kmax, }
\NormalTok{                  Href, Al2As, H, }\DataTypeTok{angiosperm =} \OtherTok{TRUE}\NormalTok{,}\DataTypeTok{taper =} \OtherTok{FALSE}\NormalTok{)}
\NormalTok{kstemmax}
\end{Highlighting}
\end{Shaded}

\begin{verbatim}
## [1] 3.347698
\end{verbatim}

In order to consider taper of xylem conduits we calculate the whole-tree
conductance per unit leaf area (\(k_{s, max, taper}\)) as described in
\citet{Christoffersen2016}:

\begin{equation}
k_{s, max, taper}=\frac{1000}{0.018} \cdot \frac{K_{s,max,pet}\cdot A_{s}}{H\cdot A_{l}}\cdot \chi_{tap:notap,ag}(H)
\end{equation}

where \(K_{s,max,pet}\) is the conductivity at the petiole level and
\(\chi_{tap:notap,ag}(H)\) is the taper factor accounting for the
decrease in the xylem conduits diameter with the height, from the
petiole to base of the trunk, which mitigates the negative effects of
height on the hydraulic safety. The conductivity at the petiole level is
obtained from \(K_{s,max,ref}\) using again:

\begin{equation}
K_{s,max,pet} = K_{s,max,ref}\cdot (r_{int, pet}/r_{int,ref})^{2}
\end{equation}

where \(r_{int, pet}\) is the radius of the petiole in the model of
\citet{Savage2010}. \citet{Christoffersen2016} use \(r_{int, pet} = 10\)
\(\mu m\) but we define it as the radius of apical conduits in a tree of
1 m height:

\begin{Shaded}
\begin{Highlighting}[]
\KeywordTok{hydraulics_terminalConduitRadius}\NormalTok{(}\FloatTok{100.0}\NormalTok{)}
\end{Highlighting}
\end{Shaded}

\begin{verbatim}
## [1] 9.035871
\end{verbatim}

\(\chi_{tap:notap,ag}(H)\) is calculated as described in the Appendix 1
section of \citet{Christoffersen2016} (see also \citet{Savage2010}). The
following figure shows the value of \(\chi_{tap:notap,ag}\) for
different heights:

\begin{center}\includegraphics{medfatebook_files/figure-latex/unnamed-chunk-84-1} \end{center}

Note that, since \(\chi_{tap:notap,ag}(1) = 3.82\) (indicated using grey
dashed lines in the last figure), the equation of maximum conductance
with taper would give a higher conductance than the equation without
taper for a tree of 1 m height, which is supposed to have a conductance
equal to conductivity. To solve this issue we define the taper factor as
\(\chi_{tap:notap,ag}(H)/\chi_{tap:notap,ag}(1)\):

\begin{equation}
k_{s, max, taper}=\frac{1000}{0.018} \cdot \frac{K_{s,max,pet}\cdot A_{s}}{H\cdot A_{l}}\cdot \frac{\chi_{tap:notap,ag}(H)}{\chi_{tap:notap,ag}(1)}
\end{equation}

The maximum stem conductance with taper (\(k_{s, max, taper}\)) of a
\emph{Q. ilex} tree of 4 m height, calculated with this second equation,
is:

\begin{Shaded}
\begin{Highlighting}[]
\NormalTok{kstemmax_tap =}\StringTok{ }\KeywordTok{hydraulics_maximumStemHydraulicConductance}\NormalTok{(xylem_kmax, }
\NormalTok{                      Href, Al2As, H, }\DataTypeTok{angiosperm =} \OtherTok{TRUE}\NormalTok{, }\DataTypeTok{taper =} \OtherTok{TRUE}\NormalTok{)}
\NormalTok{kstemmax_tap}
\end{Highlighting}
\end{Shaded}

\begin{verbatim}
## [1] 4.764396
\end{verbatim}

The next two plots show the variation of \(k_{s,max}\) for \emph{Q.
ilex} depending on the tree height and with/without considering taper of
conduits. The plot on the right (both axes in log) show the slope of the
dependency of conductance with height in both cases:

\begin{center}\includegraphics{medfatebook_files/figure-latex/unnamed-chunk-86-1} \end{center}

\subsection{Root xylem maximum hydraulic
conductance}\label{root-xylem-maximum-hydraulic-conductance}

To obtain maximum root xylem conductance (\(k_{r, max}\), in
\(mmol \cdot m^{-2} \cdot s^{-1} \cdot MPa^{-1}\)), one option taken by
\citet{Christoffersen2016} is to assume that minimum stem resistance
(inverse of maximum conductance) represents a fixed proportion of the
minimum total tree (stem+root) resistance. A value 0.625 (i.e.~62.5\%)
suggested by these authors leads to maximum total tree conductance for
our \emph{Q. ilex} tree being:

\begin{Shaded}
\begin{Highlighting}[]
\NormalTok{ktot =}\StringTok{ }\NormalTok{kstemmax}\OperatorTok{*}\FloatTok{0.625}
\NormalTok{ktot}
\end{Highlighting}
\end{Shaded}

\begin{verbatim}
## [1] 2.092311
\end{verbatim}

and the maximum root xylem conductance would be therefore:

\begin{Shaded}
\begin{Highlighting}[]
\NormalTok{krootmax =}\StringTok{ }\DecValTok{1}\OperatorTok{/}\NormalTok{((}\DecValTok{1}\OperatorTok{/}\NormalTok{ktot)}\OperatorTok{-}\NormalTok{(}\DecValTok{1}\OperatorTok{/}\NormalTok{kstemmax))}
\NormalTok{krootmax}
\end{Highlighting}
\end{Shaded}

\begin{verbatim}
## [1] 5.579497
\end{verbatim}

Now, we need to divide total maximum conductance of the root system
xylem among soil layers we need weights inversely proportional to the
length of transport distances \citep{Sperry2016}. Vertical transport
lengths can be calculated from soil depths and radial spread can be
calculated assuming cylinders with volume proportional to the
proportions of fine root biomass. Let's assume a soil with three layers:

\begin{Shaded}
\begin{Highlighting}[]
\NormalTok{d =}\StringTok{ }\NormalTok{s}\OperatorTok{$}\NormalTok{dVec}
\NormalTok{d}
\end{Highlighting}
\end{Shaded}

\begin{verbatim}
## [1]  300  700 1000
\end{verbatim}

The proportion of fine roots in each layer, assuming a linear dose
response model, will be:

\begin{Shaded}
\begin{Highlighting}[]
\NormalTok{Z50 =}\StringTok{ }\DecValTok{200}
\NormalTok{Z95 =}\StringTok{ }\DecValTok{1500}
\NormalTok{v1 =}\StringTok{ }\KeywordTok{root_ldrDistribution}\NormalTok{(Z50,Z95, d)}
\NormalTok{v1}
\end{Highlighting}
\end{Shaded}

\begin{verbatim}
##           [,1]      [,2]       [,3]
## [1,] 0.6661036 0.2784153 0.05548106
\end{verbatim}

Having this information, the calculation of root length (i.e.~the sum of
vertical and radial lengths) to each layer (\(L_j\)) is done using
function \texttt{root\_rootLength()}:

\begin{Shaded}
\begin{Highlighting}[]
\NormalTok{rl =}\StringTok{ }\KeywordTok{root_rootLengths}\NormalTok{(v1, d)}
\NormalTok{rl}
\end{Highlighting}
\end{Shaded}

\begin{verbatim}
## [1] 2150.000 1496.481 1816.149
\end{verbatim}

where lengths are in mm. The proportion of total root xylem conductance
corresponding to each layer (\(w_j\)) is given by
\texttt{root\_xylemConductanceProportions()}:

\begin{Shaded}
\begin{Highlighting}[]
\NormalTok{w1 =}\StringTok{ }\KeywordTok{root_xylemConductanceProportions}\NormalTok{(v1, d)}
\NormalTok{w1}
\end{Highlighting}
\end{Shaded}

\begin{verbatim}
## [1] 0.2762029 0.3968217 0.3269754
\end{verbatim}

Xylem conductance proportions can be quite different than the fine root
biomass proportions. This is because radial lengths are largest for the
first top layers and vertical lengths are largest for the bottom layers.
The maximum root xylem conductances of each layer will be the product of
maximum total conductance of root xylem and weights:

\begin{Shaded}
\begin{Highlighting}[]
\NormalTok{w1}\OperatorTok{*}\NormalTok{krootmax}
\end{Highlighting}
\end{Shaded}

\begin{verbatim}
## [1] 1.541073 2.214065 1.824358
\end{verbatim}

In \textbf{medfate} we calculate maximum root xylem conductance using a
reference root xylem conductivity value (\(K_{r,max,ref}\)):

\begin{equation}
k_{r,max}=\frac{1000}{0.018} \cdot \sum_{j}{\frac{w_j \cdot K_{r,max,ref}\cdot A_{s}}{L_j\cdot A_{l}}}
\end{equation}

where \(w_j\) are root xylem conductance proportion of layer \(j\) and
\(L_j\) is the root length (in m) to layer \(j\). Note that here we use
weights \(w_j\) assuming they represent proportions of total sapwood
area that come from each layer (i.e.~the longer the path the larger the
proportion of sapwood area). This calculation is made available by
function \texttt{hydraulics\_maximumRootHydraulicConductance()}. When
\(K_{r,max,ref}\) is missing, then we assume that
\(K_{r,max,ref} = K_{x,max,ref}\). Let's consider the same \emph{Q.
ilex} tree of 4m height as before. If we specify root xylem specific
conductivity as \(K_{r,max,ref} = K_{s,max,ref} =0.77\) we have:

\begin{Shaded}
\begin{Highlighting}[]
\NormalTok{rootxylem_kmax =}\StringTok{ }\FloatTok{0.77}
\NormalTok{krootmax =}\StringTok{ }\KeywordTok{hydraulics_maximumRootHydraulicConductance}\NormalTok{(rootxylem_kmax, Al2As, }
\NormalTok{                                                      v1, d)}
\NormalTok{krootmax}
\end{Highlighting}
\end{Shaded}

\begin{verbatim}
## [1] 9.769308
\end{verbatim}

The maximum root xylem conductances of each layer would be:

\begin{Shaded}
\begin{Highlighting}[]
\NormalTok{krootmaxvec =}\StringTok{ }\NormalTok{w1}\OperatorTok{*}\NormalTok{krootmax}
\NormalTok{krootmaxvec}
\end{Highlighting}
\end{Shaded}

\begin{verbatim}
## [1] 2.698311 3.876673 3.194324
\end{verbatim}

and the fraction of total xylem resistance due to stem would be:

\begin{Shaded}
\begin{Highlighting}[]
\NormalTok{(}\DecValTok{1}\OperatorTok{/}\NormalTok{kstemmax)}\OperatorTok{/}\NormalTok{((}\DecValTok{1}\OperatorTok{/}\NormalTok{kstemmax)}\OperatorTok{+}\NormalTok{(}\DecValTok{1}\OperatorTok{/}\NormalTok{krootmax))}
\end{Highlighting}
\end{Shaded}

\begin{verbatim}
## [1] 0.7447818
\end{verbatim}

In contrast with the approach of \citet{Christoffersen2016}, in this
approach the root maximum conductance depends root length and
distribution, and is not a fixed fraction of stem maximum conductance.
Assuming constant root length, then the proportion of total resistance
due to the stem will increase with tree height \citep{Magnani2000}:

\begin{center}\includegraphics{medfatebook_files/figure-latex/unnamed-chunk-97-1} \end{center}

where the horizontal gray line indicates the value of 62.5\%. Of course
rooting depth also increases with tree age, but young trees have higher
root-to-shoot ratios than older ones. Hence, a root maximum conductance
that is not fixed but increases with age seems a priori more realistic.
Moreover, \citet{Christoffersen2016} justify the value of 62.5\% from a
study which quantified total aboveground and belowground resistance in
tropical trees \citep{Fisher2006} under near-saturated (wet season)
conditions, but values of belowground resistance reported in this study
for wet conditions and trees of 30 m height are around 13\%, which
equals to 87\% fraction of aboveground resistance. On the other hand,
while rooting depths are limited by soil depth, lateral root length
increases with age and, hence, the model could be made more realistic if
this is taken into account and the curve above would probably saturate
at lower percentages.

\subsection{Rhizosphere maximum hydraulic
conductance}\label{rhizosphere-maximum-hydraulic-conductance}

Maximum rhizosphere conductance (\(k_{rh, max}\), in
\(mmol \cdot m^{-2} \cdot s^{-1} \cdot MPa^{-1}\)) is difficult to
measure directly, as it depends on the rhizosphere (i.e.~fine root)
surface in each soil layer, and will probably always be a parameter to
be calibrated. Instead of trying to estimate rhizosphere surface from
root architecture \citep{Sperry1998}, we follow \citet{Sperry2016a} and
determine the maximum rhizosphere conductance in each layer from an
inputed `average percentage rhizosphere resistance'. The percentage of
continuum resistance corresponding to the rhizosphere is calculated from
the vulnerability curves of stem, root and rhizosphere at the same water
potential. The average resistance is found by evaluating the percentage
for water potential values between 0 and \(\Psi_{crit}\). The following
figure illustrates how the supply function, for different soil water
potentials, is affected by increasing values of the average percentage
of rhizosphere resistance:

\begin{center}\includegraphics{medfatebook_files/figure-latex/unnamed-chunk-98-1} \end{center}

\citet{Sperry2016a} found average percentages of rhizosphere resistance
around 67\%, but these exceptionally-high values were probably a
consequence of using an unsegmented supply function (i.e.~single
vulnerability curve for roots, stem and leaves). If we specify a 15\% of
average resistance in the rhizosphere (see parameter
\texttt{averageFracRhizosphereResistance} in function
\texttt{defaultControl()}), the maximum rhizosphere conductance values
for the three layers are found calling:

\begin{Shaded}
\begin{Highlighting}[]
\NormalTok{krmax =}\StringTok{ }\KeywordTok{rep}\NormalTok{(}\DecValTok{0}\NormalTok{,}\DecValTok{3}\NormalTok{) }
\NormalTok{krmax[}\DecValTok{1}\NormalTok{]=}\StringTok{ }\KeywordTok{hydraulics_findRhizosphereMaximumConductance}\NormalTok{(}\DecValTok{15}\NormalTok{, }
\NormalTok{                     s}\OperatorTok{$}\NormalTok{VG_n[}\DecValTok{1}\NormalTok{],s}\OperatorTok{$}\NormalTok{VG_alpha[}\DecValTok{1}\NormalTok{],}
\NormalTok{                     krootmax, rootc,rootd, }
\NormalTok{                     kstemmax, stemc, stemd,}
\NormalTok{                     kleafmax, leafc, leafd)}
\NormalTok{krmax[}\DecValTok{2}\NormalTok{] =}\StringTok{ }\KeywordTok{hydraulics_findRhizosphereMaximumConductance}\NormalTok{(}\DecValTok{15}\NormalTok{, }
\NormalTok{                      s}\OperatorTok{$}\NormalTok{VG_n[}\DecValTok{2}\NormalTok{],s}\OperatorTok{$}\NormalTok{VG_alpha[}\DecValTok{2}\NormalTok{],}
\NormalTok{                      krootmax, rootc,rootd, }
\NormalTok{                      kstemmax, stemc, stemd,}
\NormalTok{                      kleafmax, leafc, leafd)}
\NormalTok{krmax[}\DecValTok{3}\NormalTok{] =}\StringTok{ }\KeywordTok{hydraulics_findRhizosphereMaximumConductance}\NormalTok{(}\DecValTok{15}\NormalTok{, }
\NormalTok{                      s}\OperatorTok{$}\NormalTok{VG_n[}\DecValTok{3}\NormalTok{],s}\OperatorTok{$}\NormalTok{VG_alpha[}\DecValTok{3}\NormalTok{],}
\NormalTok{                      krootmax, rootc,rootd, }
\NormalTok{                      kstemmax, stemc, stemd,}
\NormalTok{                      kleafmax, leafc, leafd)}
\NormalTok{krmax}
\end{Highlighting}
\end{Shaded}

\begin{verbatim}
## [1] 276438796  97039144  97039144
\end{verbatim}

The values are the same because the texture of the three layers is the
same in this case. If we take into account root distribution, actual
maximum rhizosphere conductance values are:

\begin{Shaded}
\begin{Highlighting}[]
\NormalTok{krmax}\OperatorTok{*}\NormalTok{v1}
\end{Highlighting}
\end{Shaded}

\begin{verbatim}
##           [,1]     [,2]    [,3]
## [1,] 184136883 27017184 5383835
\end{verbatim}

\subsection{Pressure-volume curves}\label{pressure-volume-curves-1}

Parameters of the pressure-volume curve (i.e. \(\pi_{0,stem}\) and
\(\epsilon_{stem}\)) for leaf and stem symplastic tissue are required
for each species. When parameters for stem tissue are missing,
\textbf{medfate} estimates them from wood density following
\citet{Christoffersen2016}:

\begin{equation}
\pi_{0,stem} = 0.52 - 4.16 \cdot \rho_{wood}
\end{equation}

\begin{equation}
\epsilon_{stem} = \sqrt{1.02 \cdot e^{8.5\cdot \rho_{wood}}-2.89}
\end{equation}

\subsection{Plant water storage
capacity}\label{plant-water-storage-capacity}

The water storage capacity of sapwood tissue per leaf area unit
(\(V_{sapwood}\); in \(m^3 \cdot m^{-2}\)) can be estimated as the
product of stem height (\(H\) in m) and Huber value (\(H_v\); ratio of
sapwood area to leaf area in \(m^2 \cdot m^{-2}\)) times a factor to
account for the non-cylindrical shape
(\url{http://www.fao.org/forestry/17109/en/}):

\begin{equation}
V_{sapwood} = 0.48 \cdot H \cdot H_v \cdot \Theta_{sapwood}
\end{equation}

\(\Theta_{sapwood}\) is sapwood porosity (\(cm^3\) of water per \(cm^3\)
of sapwood tissue), which can be estimated from wood density
(\(\rho_{wood}\); in \(g \cdot cm^{-3}\)):

\begin{equation}
\Theta_{sapwood} = 1 - (\rho_{wood} / 1.54)
\end{equation}

where the density of wood substance can be assumed to be fixed and equal
to 1.54 \(g \cdot cm^{-3}\) \citep{Dunlap1914}. For example, wood
densities ranging from 0.443 to 1.000 \(g \cdot cm^{-3}\) result in
sapwood porosity values between 0.35 and 0.71.

Water storage capacity of leaf tissue per leaf area unit (\(V_{leaf}\);
in \(m^3 \cdot m^{-2}\)) can be estimated as the product of specific
leaf area (SLA; in \(m^2 \cdot kg^{-1}\)) and leaf density
(\(\rho_{leaf}\); in \(g \cdot cm^{-3}\)):

\begin{equation}
V_{leaf} = \frac{10^{-3}}{SLA \cdot \rho_{leaf}} \cdot \Theta_{leaf}
\end{equation}

\(\Theta_{leaf}\) is leaf porosity (\(cm^3\) of water per \(cm^3\) of
leaf tissue), which can be estimated from leaf density:

\begin{equation}
\Theta_{leaf} = 1 - (\rho_{leaf} / 1.54)
\end{equation}

where the density of wood substance can be assumed to be fixed and equal
to 1.54 \(g \cdot cm^{-3}\) (Dunlap 1914).

For example, let's calculate the stem and leaf water capacity for a Q.
ilex tree of 15 m height:

\begin{Shaded}
\begin{Highlighting}[]
\NormalTok{wd =}\StringTok{ }\FloatTok{1.0}
\NormalTok{Al2As =}\StringTok{ }\DecValTok{2512} 
\NormalTok{H =}\StringTok{ }\DecValTok{1500} \CommentTok{# 15 m}
\KeywordTok{hydraulics_stemWaterCapacity}\NormalTok{(Al2As, H, wd)}
\end{Highlighting}
\end{Shaded}

\begin{verbatim}
## [1] 0.001005046
\end{verbatim}

\begin{Shaded}
\begin{Highlighting}[]
\NormalTok{ld =}\StringTok{ }\FloatTok{0.7}
\NormalTok{SLA =}\StringTok{ }\FloatTok{5.870} 
\KeywordTok{hydraulics_leafWaterCapacity}\NormalTok{(SLA, ld)}
\end{Highlighting}
\end{Shaded}

\begin{verbatim}
## [1] 0.0001327463
\end{verbatim}

\section{Stomatal regulation and
photosynthesis}\label{stomatal-regulation-and-photosynthesis}

\subsection{Stomatal conductance}\label{stomatal-conductance}

Maximum stomatal conductance (\(g_{swmax}\)) is an input parameter for
each species. When species-specific values are missing, the following
relation with maximum leaf hydraulic conductance (\(k_{l, max}\)) is
used \citep{Mencuccini2003}:

\begin{equation}
g_{swmax} = e^{4.797 + 0.633\cdot \log(k_{l, max})}
\end{equation}

Species values for \(g_{swmin}\) were taken from \citet{Duursma2018}.
Following the same authors, a value of \(g_{swmin}\) = 0.0045
\(mol H_2O \cdot s^{-1} \cdot m^{-2}\) is taken as default, when
species-specific values are missing.

\subsection{Photosynthesis}\label{photosynthesis}

Rubisco's maximum carboxylation rate at 25ºC (\(V_{max, 298}\), in
\(\mu mol CO_2 \cdot s^{-1} \cdot m^{-2}\)) is a required input
parameter for each species (\texttt{Vmax298}), and if its value is
missing a default value of 100
\(\mu mol CO_2 \cdot s^{-1} \cdot m^{-2}\) is used. The maximum rate of
electron transport at the same temperature (\(J_{max, 298}\)) can be
provided by the user (\texttt{Jmax298}) but, if not, it is estimated
from \(V_{max, 298}\) using \citet{Walker2014}:

\begin{equation}
J_{max, 298} = e^{1.197 + 0.847\cdot \log(V_{max,298})}
\end{equation}

\chapter{Symbols}\label{symbols}

The following tables list all symbols used in this document, along with
their units and definition. When symbols are input for \textbf{medfate}
model functions, the R name of those parameters in the package (either
in data frame \texttt{SpParamsMED}, soil input data frame, or the output
of functions such as \texttt{spwbInput()}) is also indicated.

\section{Soils}\label{soils}

\begin{longtable}[]{@{}llll@{}}
\toprule
\begin{minipage}[b]{0.11\columnwidth}\raggedright\strut
Symbol\strut
\end{minipage} & \begin{minipage}[b]{0.10\columnwidth}\raggedright\strut
Units\strut
\end{minipage} & \begin{minipage}[b]{0.12\columnwidth}\raggedright\strut
R param\strut
\end{minipage} & \begin{minipage}[b]{0.45\columnwidth}\raggedright\strut
Description\strut
\end{minipage}\tabularnewline
\midrule
\endhead
\begin{minipage}[t]{0.11\columnwidth}\raggedright\strut
\(d_{s}\)\strut
\end{minipage} & \begin{minipage}[t]{0.10\columnwidth}\raggedright\strut
mm\strut
\end{minipage} & \begin{minipage}[t]{0.12\columnwidth}\raggedright\strut
\texttt{widths}\strut
\end{minipage} & \begin{minipage}[t]{0.45\columnwidth}\raggedright\strut
Soil layer width in soil layer \(s\)\strut
\end{minipage}\tabularnewline
\begin{minipage}[t]{0.11\columnwidth}\raggedright\strut
\(P_{clay,s}\)\strut
\end{minipage} & \begin{minipage}[t]{0.10\columnwidth}\raggedright\strut
\%\strut
\end{minipage} & \begin{minipage}[t]{0.12\columnwidth}\raggedright\strut
\texttt{clay}\strut
\end{minipage} & \begin{minipage}[t]{0.45\columnwidth}\raggedright\strut
Percent of clay in soil layer \(s\)\strut
\end{minipage}\tabularnewline
\begin{minipage}[t]{0.11\columnwidth}\raggedright\strut
\(P_{sand,s}\)\strut
\end{minipage} & \begin{minipage}[t]{0.10\columnwidth}\raggedright\strut
\%\strut
\end{minipage} & \begin{minipage}[t]{0.12\columnwidth}\raggedright\strut
\texttt{sand}\strut
\end{minipage} & \begin{minipage}[t]{0.45\columnwidth}\raggedright\strut
Percent of sand in soil layer \(s\)\strut
\end{minipage}\tabularnewline
\begin{minipage}[t]{0.11\columnwidth}\raggedright\strut
\(OM\)\strut
\end{minipage} & \begin{minipage}[t]{0.10\columnwidth}\raggedright\strut
\%\strut
\end{minipage} & \begin{minipage}[t]{0.12\columnwidth}\raggedright\strut
\texttt{om}\strut
\end{minipage} & \begin{minipage}[t]{0.45\columnwidth}\raggedright\strut
Percentage of organic mater per dry weight\strut
\end{minipage}\tabularnewline
\begin{minipage}[t]{0.11\columnwidth}\raggedright\strut
\(BD_{s}\)\strut
\end{minipage} & \begin{minipage}[t]{0.10\columnwidth}\raggedright\strut
\(g\cdot cm^{-3}\)\strut
\end{minipage} & \begin{minipage}[t]{0.12\columnwidth}\raggedright\strut
\texttt{bd}\strut
\end{minipage} & \begin{minipage}[t]{0.45\columnwidth}\raggedright\strut
Bulk density in soil layer \(s\)\strut
\end{minipage}\tabularnewline
\begin{minipage}[t]{0.11\columnwidth}\raggedright\strut
\(P_{rocks,s}\)\strut
\end{minipage} & \begin{minipage}[t]{0.10\columnwidth}\raggedright\strut
\%\strut
\end{minipage} & \begin{minipage}[t]{0.12\columnwidth}\raggedright\strut
\texttt{rfc}\strut
\end{minipage} & \begin{minipage}[t]{0.45\columnwidth}\raggedright\strut
Percentage of rock fragment content in soil layer \(s\)\strut
\end{minipage}\tabularnewline
\begin{minipage}[t]{0.11\columnwidth}\raggedright\strut
\(\theta_s\)\strut
\end{minipage} & \begin{minipage}[t]{0.10\columnwidth}\raggedright\strut
\(m^3 \cdot m^{-3}\)\strut
\end{minipage} & \begin{minipage}[t]{0.12\columnwidth}\raggedright\strut
\strut
\end{minipage} & \begin{minipage}[t]{0.45\columnwidth}\raggedright\strut
Volumetric moisture in soil layer \(s\)\strut
\end{minipage}\tabularnewline
\begin{minipage}[t]{0.11\columnwidth}\raggedright\strut
\(\Psi_s\)\strut
\end{minipage} & \begin{minipage}[t]{0.10\columnwidth}\raggedright\strut
MPa\strut
\end{minipage} & \begin{minipage}[t]{0.12\columnwidth}\raggedright\strut
\strut
\end{minipage} & \begin{minipage}[t]{0.45\columnwidth}\raggedright\strut
Water potential in soil layer \(s\)\strut
\end{minipage}\tabularnewline
\begin{minipage}[t]{0.11\columnwidth}\raggedright\strut
\(\Psi_{fc}\)\strut
\end{minipage} & \begin{minipage}[t]{0.10\columnwidth}\raggedright\strut
MPa\strut
\end{minipage} & \begin{minipage}[t]{0.12\columnwidth}\raggedright\strut
\strut
\end{minipage} & \begin{minipage}[t]{0.45\columnwidth}\raggedright\strut
Water potential at field capacity\strut
\end{minipage}\tabularnewline
\begin{minipage}[t]{0.11\columnwidth}\raggedright\strut
\(A\), \(B\)\strut
\end{minipage} & \begin{minipage}[t]{0.10\columnwidth}\raggedright\strut
\strut
\end{minipage} & \begin{minipage}[t]{0.12\columnwidth}\raggedright\strut
\strut
\end{minipage} & \begin{minipage}[t]{0.45\columnwidth}\raggedright\strut
Parameters of the Saxton pedotransfer functions\strut
\end{minipage}\tabularnewline
\begin{minipage}[t]{0.11\columnwidth}\raggedright\strut
\(\theta_{sat}\)\strut
\end{minipage} & \begin{minipage}[t]{0.10\columnwidth}\raggedright\strut
\(m^3 \cdot m^{-3}\)\strut
\end{minipage} & \begin{minipage}[t]{0.12\columnwidth}\raggedright\strut
\texttt{VG\_theta\_sat}\strut
\end{minipage} & \begin{minipage}[t]{0.45\columnwidth}\raggedright\strut
Volumetric moisture at saturation\strut
\end{minipage}\tabularnewline
\begin{minipage}[t]{0.11\columnwidth}\raggedright\strut
\(\theta_{fc}\)\strut
\end{minipage} & \begin{minipage}[t]{0.10\columnwidth}\raggedright\strut
\(m^3 \cdot m^{-3}\)\strut
\end{minipage} & \begin{minipage}[t]{0.12\columnwidth}\raggedright\strut
\strut
\end{minipage} & \begin{minipage}[t]{0.45\columnwidth}\raggedright\strut
Volumetric moisture at field capacity (-0.033 MPa)\strut
\end{minipage}\tabularnewline
\begin{minipage}[t]{0.11\columnwidth}\raggedright\strut
\(\theta_{wp}\)\strut
\end{minipage} & \begin{minipage}[t]{0.10\columnwidth}\raggedright\strut
\(m^3 \cdot m^{-3}\)\strut
\end{minipage} & \begin{minipage}[t]{0.12\columnwidth}\raggedright\strut
\strut
\end{minipage} & \begin{minipage}[t]{0.45\columnwidth}\raggedright\strut
Volumetric moisture at wilting point (-1.5 MPa)\strut
\end{minipage}\tabularnewline
\begin{minipage}[t]{0.11\columnwidth}\raggedright\strut
\(\theta_{res}\)\strut
\end{minipage} & \begin{minipage}[t]{0.10\columnwidth}\raggedright\strut
\(m^3 \cdot m^{-3}\)\strut
\end{minipage} & \begin{minipage}[t]{0.12\columnwidth}\raggedright\strut
\texttt{VG\_theta\_res}\strut
\end{minipage} & \begin{minipage}[t]{0.45\columnwidth}\raggedright\strut
Residual volumetric moisture\strut
\end{minipage}\tabularnewline
\begin{minipage}[t]{0.11\columnwidth}\raggedright\strut
\(n\)\strut
\end{minipage} & \begin{minipage}[t]{0.10\columnwidth}\raggedright\strut
\strut
\end{minipage} & \begin{minipage}[t]{0.12\columnwidth}\raggedright\strut
\texttt{VG\_n}\strut
\end{minipage} & \begin{minipage}[t]{0.45\columnwidth}\raggedright\strut
Parameter of the Van Genuchten \citeyearpar{Genuchten1980} model\strut
\end{minipage}\tabularnewline
\begin{minipage}[t]{0.11\columnwidth}\raggedright\strut
\(\alpha\)\strut
\end{minipage} & \begin{minipage}[t]{0.10\columnwidth}\raggedright\strut
\strut
\end{minipage} & \begin{minipage}[t]{0.12\columnwidth}\raggedright\strut
\texttt{VG\_alpha}\strut
\end{minipage} & \begin{minipage}[t]{0.45\columnwidth}\raggedright\strut
Parameter of the Van Genuchten \citeyearpar{Genuchten1980} model\strut
\end{minipage}\tabularnewline
\begin{minipage}[t]{0.11\columnwidth}\raggedright\strut
\(P_{macro, s}\)\strut
\end{minipage} & \begin{minipage}[t]{0.10\columnwidth}\raggedright\strut
\%\strut
\end{minipage} & \begin{minipage}[t]{0.12\columnwidth}\raggedright\strut
\texttt{macro}\strut
\end{minipage} & \begin{minipage}[t]{0.45\columnwidth}\raggedright\strut
Percentage of macroporosity corresponding to each soil layers.\strut
\end{minipage}\tabularnewline
\begin{minipage}[t]{0.11\columnwidth}\raggedright\strut
\(\gamma_{soil}\)\strut
\end{minipage} & \begin{minipage}[t]{0.10\columnwidth}\raggedright\strut
\(mm \cdot day^{-1}\)\strut
\end{minipage} & \begin{minipage}[t]{0.12\columnwidth}\raggedright\strut
\texttt{Gsoil}\strut
\end{minipage} & \begin{minipage}[t]{0.45\columnwidth}\raggedright\strut
The maximum daily evaporation from soil\strut
\end{minipage}\tabularnewline
\begin{minipage}[t]{0.11\columnwidth}\raggedright\strut
\(\kappa_{soil}\)\strut
\end{minipage} & \begin{minipage}[t]{0.10\columnwidth}\raggedright\strut
\strut
\end{minipage} & \begin{minipage}[t]{0.12\columnwidth}\raggedright\strut
\texttt{Ksoil}\strut
\end{minipage} & \begin{minipage}[t]{0.45\columnwidth}\raggedright\strut
Extinction parameter to regulate the amount of water evaporated from
each soil layer\strut
\end{minipage}\tabularnewline
\begin{minipage}[t]{0.11\columnwidth}\raggedright\strut
\(WT\)\strut
\end{minipage} & \begin{minipage}[t]{0.10\columnwidth}\raggedright\strut
mm\strut
\end{minipage} & \begin{minipage}[t]{0.12\columnwidth}\raggedright\strut
\strut
\end{minipage} & \begin{minipage}[t]{0.45\columnwidth}\raggedright\strut
Water table depth\strut
\end{minipage}\tabularnewline
\begin{minipage}[t]{0.11\columnwidth}\raggedright\strut
\(W_{s}\)\strut
\end{minipage} & \begin{minipage}[t]{0.10\columnwidth}\raggedright\strut
{[}0-1{]}\strut
\end{minipage} & \begin{minipage}[t]{0.12\columnwidth}\raggedright\strut
\texttt{W}\strut
\end{minipage} & \begin{minipage}[t]{0.45\columnwidth}\raggedright\strut
Proportion of volumetric moisture in relation to field capacity in soil
layer \(s\)\strut
\end{minipage}\tabularnewline
\begin{minipage}[t]{0.11\columnwidth}\raggedright\strut
\(S_{snow}\)\strut
\end{minipage} & \begin{minipage}[t]{0.10\columnwidth}\raggedright\strut
mm\strut
\end{minipage} & \begin{minipage}[t]{0.12\columnwidth}\raggedright\strut
\texttt{SWE}\strut
\end{minipage} & \begin{minipage}[t]{0.45\columnwidth}\raggedright\strut
Snow water equivalent of the snow pack storage over the soil
surface\strut
\end{minipage}\tabularnewline
\bottomrule
\end{longtable}

\section{Vegetation}\label{vegetation}

\begin{longtable}[]{@{}llll@{}}
\toprule
\begin{minipage}[b]{0.11\columnwidth}\raggedright\strut
Symbol\strut
\end{minipage} & \begin{minipage}[b]{0.10\columnwidth}\raggedright\strut
Units\strut
\end{minipage} & \begin{minipage}[b]{0.12\columnwidth}\raggedright\strut
R param\strut
\end{minipage} & \begin{minipage}[b]{0.45\columnwidth}\raggedright\strut
Description\strut
\end{minipage}\tabularnewline
\midrule
\endhead
\begin{minipage}[t]{0.11\columnwidth}\raggedright\strut
\(H_i\)\strut
\end{minipage} & \begin{minipage}[t]{0.10\columnwidth}\raggedright\strut
cm\strut
\end{minipage} & \begin{minipage}[t]{0.12\columnwidth}\raggedright\strut
\texttt{H}\strut
\end{minipage} & \begin{minipage}[t]{0.45\columnwidth}\raggedright\strut
Total tree or shrub height of cohort \(i\)\strut
\end{minipage}\tabularnewline
\begin{minipage}[t]{0.11\columnwidth}\raggedright\strut
\(CR_i\)\strut
\end{minipage} & \begin{minipage}[t]{0.10\columnwidth}\raggedright\strut
\strut
\end{minipage} & \begin{minipage}[t]{0.12\columnwidth}\raggedright\strut
\texttt{CR}\strut
\end{minipage} & \begin{minipage}[t]{0.45\columnwidth}\raggedright\strut
Crown ratio (i.e.~the ratio between crown length and total height) of
cohort \(i\)\strut
\end{minipage}\tabularnewline
\begin{minipage}[t]{0.11\columnwidth}\raggedright\strut
\(LAI^{live}_i\)\strut
\end{minipage} & \begin{minipage}[t]{0.10\columnwidth}\raggedright\strut
\strut
\end{minipage} & \begin{minipage}[t]{0.12\columnwidth}\raggedright\strut
\texttt{LAI\_live}\strut
\end{minipage} & \begin{minipage}[t]{0.45\columnwidth}\raggedright\strut
(Maximum) leaf area index (one-side leaf area per surface area of the
stand) of cohort \(i\)\strut
\end{minipage}\tabularnewline
\begin{minipage}[t]{0.11\columnwidth}\raggedright\strut
\(LAI^{dead}_i\)\strut
\end{minipage} & \begin{minipage}[t]{0.10\columnwidth}\raggedright\strut
\strut
\end{minipage} & \begin{minipage}[t]{0.12\columnwidth}\raggedright\strut
\texttt{LAI\_dead}\strut
\end{minipage} & \begin{minipage}[t]{0.45\columnwidth}\raggedright\strut
Dead leaf area index (one-side dead leaf area per surface area of the
stand) of cohort \(i\)\strut
\end{minipage}\tabularnewline
\begin{minipage}[t]{0.11\columnwidth}\raggedright\strut
\(LAI^{\phi}_i\)\strut
\end{minipage} & \begin{minipage}[t]{0.10\columnwidth}\raggedright\strut
\strut
\end{minipage} & \begin{minipage}[t]{0.12\columnwidth}\raggedright\strut
\texttt{LAI\_expanded}\strut
\end{minipage} & \begin{minipage}[t]{0.45\columnwidth}\raggedright\strut
Current expanded leaf area index (one-side expanded leaf area per
surface area of the stand) of cohort \(i\)\strut
\end{minipage}\tabularnewline
\begin{minipage}[t]{0.11\columnwidth}\raggedright\strut
\(v_{i,s}\)\strut
\end{minipage} & \begin{minipage}[t]{0.10\columnwidth}\raggedright\strut
{[}0-1{]}\strut
\end{minipage} & \begin{minipage}[t]{0.12\columnwidth}\raggedright\strut
\texttt{V{[}i,s{]}}\strut
\end{minipage} & \begin{minipage}[t]{0.45\columnwidth}\raggedright\strut
The proportion of fine roots of cohort \(i\) in soil layer \(s\)\strut
\end{minipage}\tabularnewline
\begin{minipage}[t]{0.11\columnwidth}\raggedright\strut
\(Z_{50,i}\)\strut
\end{minipage} & \begin{minipage}[t]{0.10\columnwidth}\raggedright\strut
mm\strut
\end{minipage} & \begin{minipage}[t]{0.12\columnwidth}\raggedright\strut
\texttt{Z50}\strut
\end{minipage} & \begin{minipage}[t]{0.45\columnwidth}\raggedright\strut
Depth above which 50\% of the fine root mass of cohort \(i\) is
located\strut
\end{minipage}\tabularnewline
\begin{minipage}[t]{0.11\columnwidth}\raggedright\strut
\(Z_{95,i}\)\strut
\end{minipage} & \begin{minipage}[t]{0.10\columnwidth}\raggedright\strut
mm\strut
\end{minipage} & \begin{minipage}[t]{0.12\columnwidth}\raggedright\strut
\texttt{Z95}\strut
\end{minipage} & \begin{minipage}[t]{0.45\columnwidth}\raggedright\strut
Depth above which 95\% of the fine root mass of cohort \(i\) is
located\strut
\end{minipage}\tabularnewline
\bottomrule
\end{longtable}

\section{Anatomy}\label{anatomy}

\begin{longtable}[]{@{}llll@{}}
\toprule
\begin{minipage}[b]{0.11\columnwidth}\raggedright\strut
Symbol\strut
\end{minipage} & \begin{minipage}[b]{0.10\columnwidth}\raggedright\strut
Units\strut
\end{minipage} & \begin{minipage}[b]{0.12\columnwidth}\raggedright\strut
R param\strut
\end{minipage} & \begin{minipage}[b]{0.45\columnwidth}\raggedright\strut
Description\strut
\end{minipage}\tabularnewline
\midrule
\endhead
\begin{minipage}[t]{0.11\columnwidth}\raggedright\strut
\(H_v\)\strut
\end{minipage} & \begin{minipage}[t]{0.10\columnwidth}\raggedright\strut
\(m^2 \cdot m^{-2}\)\strut
\end{minipage} & \begin{minipage}[t]{0.12\columnwidth}\raggedright\strut
\texttt{1/Al2As}\strut
\end{minipage} & \begin{minipage}[t]{0.45\columnwidth}\raggedright\strut
Huber value (ratio of sapwood area to leaf area)\strut
\end{minipage}\tabularnewline
\begin{minipage}[t]{0.11\columnwidth}\raggedright\strut
\(d\)\strut
\end{minipage} & \begin{minipage}[t]{0.10\columnwidth}\raggedright\strut
\(cm\)\strut
\end{minipage} & \begin{minipage}[t]{0.12\columnwidth}\raggedright\strut
\texttt{LeafWidth}\strut
\end{minipage} & \begin{minipage}[t]{0.45\columnwidth}\raggedright\strut
Leaf width\strut
\end{minipage}\tabularnewline
\begin{minipage}[t]{0.11\columnwidth}\raggedright\strut
\(SLA\)\strut
\end{minipage} & \begin{minipage}[t]{0.10\columnwidth}\raggedright\strut
\(m^2 \cdot kg^{-1}\)\strut
\end{minipage} & \begin{minipage}[t]{0.12\columnwidth}\raggedright\strut
\texttt{SLA}\strut
\end{minipage} & \begin{minipage}[t]{0.45\columnwidth}\raggedright\strut
Specific leaf area\strut
\end{minipage}\tabularnewline
\begin{minipage}[t]{0.11\columnwidth}\raggedright\strut
\(\rho_{leaf}\)\strut
\end{minipage} & \begin{minipage}[t]{0.10\columnwidth}\raggedright\strut
\(g \cdot cm^{-3}\)\strut
\end{minipage} & \begin{minipage}[t]{0.12\columnwidth}\raggedright\strut
\texttt{LeafDens}\strut
\end{minipage} & \begin{minipage}[t]{0.45\columnwidth}\raggedright\strut
Leaf tissue density\strut
\end{minipage}\tabularnewline
\begin{minipage}[t]{0.11\columnwidth}\raggedright\strut
\(\rho_{wood}\)\strut
\end{minipage} & \begin{minipage}[t]{0.10\columnwidth}\raggedright\strut
\(g \cdot cm^{-3}\)\strut
\end{minipage} & \begin{minipage}[t]{0.12\columnwidth}\raggedright\strut
\texttt{WoodDens}\strut
\end{minipage} & \begin{minipage}[t]{0.45\columnwidth}\raggedright\strut
Wood tissue density\strut
\end{minipage}\tabularnewline
\begin{minipage}[t]{0.11\columnwidth}\raggedright\strut
\(\Theta_{sapwood}\)\strut
\end{minipage} & \begin{minipage}[t]{0.10\columnwidth}\raggedright\strut
\(m^3 \cdot m^{-3}\)\strut
\end{minipage} & \begin{minipage}[t]{0.12\columnwidth}\raggedright\strut
\strut
\end{minipage} & \begin{minipage}[t]{0.45\columnwidth}\raggedright\strut
Sapwood porosity (volume of empty spaces over total volume)\strut
\end{minipage}\tabularnewline
\begin{minipage}[t]{0.11\columnwidth}\raggedright\strut
\(\Theta_{leaf}\)\strut
\end{minipage} & \begin{minipage}[t]{0.10\columnwidth}\raggedright\strut
\(m^3 \cdot m^{-3}\)\strut
\end{minipage} & \begin{minipage}[t]{0.12\columnwidth}\raggedright\strut
\strut
\end{minipage} & \begin{minipage}[t]{0.45\columnwidth}\raggedright\strut
Leaf porosity (volume of empty spaces over total volume)\strut
\end{minipage}\tabularnewline
\bottomrule
\end{longtable}

\section{Plant hydraulics}\label{plant-hydraulics-2}

\begin{longtable}[]{@{}llll@{}}
\toprule
\begin{minipage}[b]{0.11\columnwidth}\raggedright\strut
Symbol\strut
\end{minipage} & \begin{minipage}[b]{0.10\columnwidth}\raggedright\strut
Units\strut
\end{minipage} & \begin{minipage}[b]{0.12\columnwidth}\raggedright\strut
R param\strut
\end{minipage} & \begin{minipage}[b]{0.45\columnwidth}\raggedright\strut
Description\strut
\end{minipage}\tabularnewline
\midrule
\endhead
\begin{minipage}[t]{0.11\columnwidth}\raggedright\strut
\(K_{s,max,ref}\)\strut
\end{minipage} & \begin{minipage}[t]{0.10\columnwidth}\raggedright\strut
\(kg \cdot s^{-1} \cdot m^{-1} \cdot MPa^{-1}\)\strut
\end{minipage} & \begin{minipage}[t]{0.12\columnwidth}\raggedright\strut
\texttt{Kmax\_stemxylem}\strut
\end{minipage} & \begin{minipage}[t]{0.45\columnwidth}\raggedright\strut
Maximum stem sapwood reference conductivity per leaf area unit\strut
\end{minipage}\tabularnewline
\begin{minipage}[t]{0.11\columnwidth}\raggedright\strut
\(K_{r,max,ref}\)\strut
\end{minipage} & \begin{minipage}[t]{0.10\columnwidth}\raggedright\strut
\(kg \cdot s^{-1} \cdot m^{-1} \cdot MPa^{-1}\)\strut
\end{minipage} & \begin{minipage}[t]{0.12\columnwidth}\raggedright\strut
\texttt{Kmax\_rootxylem}\strut
\end{minipage} & \begin{minipage}[t]{0.45\columnwidth}\raggedright\strut
Maximum root sapwood reference conductivity per leaf area unit\strut
\end{minipage}\tabularnewline
\begin{minipage}[t]{0.11\columnwidth}\raggedright\strut
\(k_{s, max}\)\strut
\end{minipage} & \begin{minipage}[t]{0.10\columnwidth}\raggedright\strut
\(mmol \cdot s^{-1} \cdot m^{-2} \cdot MPa^{-1}\)\strut
\end{minipage} & \begin{minipage}[t]{0.12\columnwidth}\raggedright\strut
\strut
\end{minipage} & \begin{minipage}[t]{0.45\columnwidth}\raggedright\strut
Maximum whole-stem conductance (per leaf area unit)\strut
\end{minipage}\tabularnewline
\begin{minipage}[t]{0.11\columnwidth}\raggedright\strut
\(k_{r, max}\)\strut
\end{minipage} & \begin{minipage}[t]{0.10\columnwidth}\raggedright\strut
\(mmol \cdot s^{-1} \cdot m^{-2} \cdot MPa^{-1}\)\strut
\end{minipage} & \begin{minipage}[t]{0.12\columnwidth}\raggedright\strut
\strut
\end{minipage} & \begin{minipage}[t]{0.45\columnwidth}\raggedright\strut
Maximum root conductance (per leaf area unit)\strut
\end{minipage}\tabularnewline
\begin{minipage}[t]{0.11\columnwidth}\raggedright\strut
\(k_{rh, max}\)\strut
\end{minipage} & \begin{minipage}[t]{0.10\columnwidth}\raggedright\strut
\(mmol \cdot s^{-1} \cdot m^{-2} \cdot MPa^{-1}\)\strut
\end{minipage} & \begin{minipage}[t]{0.12\columnwidth}\raggedright\strut
\strut
\end{minipage} & \begin{minipage}[t]{0.45\columnwidth}\raggedright\strut
Maximum rhizosphere conductance (per leaf area unit)\strut
\end{minipage}\tabularnewline
\begin{minipage}[t]{0.11\columnwidth}\raggedright\strut
\(k_{l, max}\)\strut
\end{minipage} & \begin{minipage}[t]{0.10\columnwidth}\raggedright\strut
\(mmol \cdot s^{-1} \cdot m^{-2} \cdot MPa^{-1}\)\strut
\end{minipage} & \begin{minipage}[t]{0.12\columnwidth}\raggedright\strut
\texttt{VCleaf\_kmax}\strut
\end{minipage} & \begin{minipage}[t]{0.45\columnwidth}\raggedright\strut
Maximum leaf conductance (per leaf area unit)\strut
\end{minipage}\tabularnewline
\begin{minipage}[t]{0.11\columnwidth}\raggedright\strut
\(c_l\), \(d_l\)\strut
\end{minipage} & \begin{minipage}[t]{0.10\columnwidth}\raggedright\strut
(unitless), MPa\strut
\end{minipage} & \begin{minipage}[t]{0.12\columnwidth}\raggedright\strut
\texttt{VCleaf\_c}, \texttt{VCleaf\_d}\strut
\end{minipage} & \begin{minipage}[t]{0.45\columnwidth}\raggedright\strut
Parameters of the vulnerability curve for leaves\strut
\end{minipage}\tabularnewline
\begin{minipage}[t]{0.11\columnwidth}\raggedright\strut
\(c_r\), \(d_r\)\strut
\end{minipage} & \begin{minipage}[t]{0.10\columnwidth}\raggedright\strut
(unitless), MPa\strut
\end{minipage} & \begin{minipage}[t]{0.12\columnwidth}\raggedright\strut
\texttt{VCroot\_c}, \texttt{VCroot\_d}\strut
\end{minipage} & \begin{minipage}[t]{0.45\columnwidth}\raggedright\strut
Parameters of the vulnerability curve for root xylem\strut
\end{minipage}\tabularnewline
\begin{minipage}[t]{0.11\columnwidth}\raggedright\strut
\(c_s\), \(d_s\)\strut
\end{minipage} & \begin{minipage}[t]{0.10\columnwidth}\raggedright\strut
(unitless), MPa\strut
\end{minipage} & \begin{minipage}[t]{0.12\columnwidth}\raggedright\strut
\texttt{VCstem\_c}, \texttt{VCstem\_d}\strut
\end{minipage} & \begin{minipage}[t]{0.45\columnwidth}\raggedright\strut
Parameters of the vulnerability curve for stem xylem\strut
\end{minipage}\tabularnewline
\begin{minipage}[t]{0.11\columnwidth}\raggedright\strut
\(n\), \(\alpha\)\strut
\end{minipage} & \begin{minipage}[t]{0.10\columnwidth}\raggedright\strut
\strut
\end{minipage} & \begin{minipage}[t]{0.12\columnwidth}\raggedright\strut
\strut
\end{minipage} & \begin{minipage}[t]{0.45\columnwidth}\raggedright\strut
Texture-specific parameters of the van Genuchten equation\strut
\end{minipage}\tabularnewline
\begin{minipage}[t]{0.11\columnwidth}\raggedright\strut
\(\Psi\)\strut
\end{minipage} & \begin{minipage}[t]{0.10\columnwidth}\raggedright\strut
MPa\strut
\end{minipage} & \begin{minipage}[t]{0.12\columnwidth}\raggedright\strut
\strut
\end{minipage} & \begin{minipage}[t]{0.45\columnwidth}\raggedright\strut
Water potential in a given water compartment\strut
\end{minipage}\tabularnewline
\begin{minipage}[t]{0.11\columnwidth}\raggedright\strut
\(\Psi_P\)\strut
\end{minipage} & \begin{minipage}[t]{0.10\columnwidth}\raggedright\strut
MPa\strut
\end{minipage} & \begin{minipage}[t]{0.12\columnwidth}\raggedright\strut
\strut
\end{minipage} & \begin{minipage}[t]{0.45\columnwidth}\raggedright\strut
Turgor water potential in a given water compartment\strut
\end{minipage}\tabularnewline
\begin{minipage}[t]{0.11\columnwidth}\raggedright\strut
\(\Psi_S\)\strut
\end{minipage} & \begin{minipage}[t]{0.10\columnwidth}\raggedright\strut
MPa\strut
\end{minipage} & \begin{minipage}[t]{0.12\columnwidth}\raggedright\strut
\strut
\end{minipage} & \begin{minipage}[t]{0.45\columnwidth}\raggedright\strut
Osmotic (solute) water potential in a given water compartment\strut
\end{minipage}\tabularnewline
\begin{minipage}[t]{0.11\columnwidth}\raggedright\strut
\(\Psi_{cav}\)\strut
\end{minipage} & \begin{minipage}[t]{0.10\columnwidth}\raggedright\strut
MPa\strut
\end{minipage} & \begin{minipage}[t]{0.12\columnwidth}\raggedright\strut
\strut
\end{minipage} & \begin{minipage}[t]{0.45\columnwidth}\raggedright\strut
Minimum water potential experienced by xylem in previous steps
(cavitation)\strut
\end{minipage}\tabularnewline
\begin{minipage}[t]{0.11\columnwidth}\raggedright\strut
\(\Psi_{canopy}\)\strut
\end{minipage} & \begin{minipage}[t]{0.10\columnwidth}\raggedright\strut
MPa\strut
\end{minipage} & \begin{minipage}[t]{0.12\columnwidth}\raggedright\strut
\strut
\end{minipage} & \begin{minipage}[t]{0.45\columnwidth}\raggedright\strut
Canopy water potential\strut
\end{minipage}\tabularnewline
\begin{minipage}[t]{0.11\columnwidth}\raggedright\strut
\(\Psi_{leaf}\)\strut
\end{minipage} & \begin{minipage}[t]{0.10\columnwidth}\raggedright\strut
MPa\strut
\end{minipage} & \begin{minipage}[t]{0.12\columnwidth}\raggedright\strut
\strut
\end{minipage} & \begin{minipage}[t]{0.45\columnwidth}\raggedright\strut
Leaf water potential\strut
\end{minipage}\tabularnewline
\begin{minipage}[t]{0.11\columnwidth}\raggedright\strut
\(\Psi_{rootcrown}\)\strut
\end{minipage} & \begin{minipage}[t]{0.10\columnwidth}\raggedright\strut
MPa\strut
\end{minipage} & \begin{minipage}[t]{0.12\columnwidth}\raggedright\strut
\strut
\end{minipage} & \begin{minipage}[t]{0.45\columnwidth}\raggedright\strut
Water potential at the root crown\strut
\end{minipage}\tabularnewline
\begin{minipage}[t]{0.11\columnwidth}\raggedright\strut
\(\Psi_{stem}\)\strut
\end{minipage} & \begin{minipage}[t]{0.10\columnwidth}\raggedright\strut
MPa\strut
\end{minipage} & \begin{minipage}[t]{0.12\columnwidth}\raggedright\strut
\strut
\end{minipage} & \begin{minipage}[t]{0.45\columnwidth}\raggedright\strut
Water potential at the end (highest part) of the stem\strut
\end{minipage}\tabularnewline
\begin{minipage}[t]{0.11\columnwidth}\raggedright\strut
\(PLC\)\strut
\end{minipage} & \begin{minipage}[t]{0.10\columnwidth}\raggedright\strut
{[}0-1{]}\strut
\end{minipage} & \begin{minipage}[t]{0.12\columnwidth}\raggedright\strut
\strut
\end{minipage} & \begin{minipage}[t]{0.45\columnwidth}\raggedright\strut
Proportion of conductance loss in stem xylem tissue\strut
\end{minipage}\tabularnewline
\begin{minipage}[t]{0.11\columnwidth}\raggedright\strut
\(p_{root}\)\strut
\end{minipage} & \begin{minipage}[t]{0.10\columnwidth}\raggedright\strut
{[}0-1{]}\strut
\end{minipage} & \begin{minipage}[t]{0.12\columnwidth}\raggedright\strut
\texttt{pRootDisc}\strut
\end{minipage} & \begin{minipage}[t]{0.45\columnwidth}\raggedright\strut
Relative root conductance leading to hydraulic disconnection from a soil
layer\strut
\end{minipage}\tabularnewline
\bottomrule
\end{longtable}

\section{Plant water compartments}\label{plant-water-compartments}

\begin{longtable}[]{@{}llll@{}}
\toprule
\begin{minipage}[b]{0.11\columnwidth}\raggedright\strut
Symbol\strut
\end{minipage} & \begin{minipage}[b]{0.10\columnwidth}\raggedright\strut
Units\strut
\end{minipage} & \begin{minipage}[b]{0.12\columnwidth}\raggedright\strut
R param\strut
\end{minipage} & \begin{minipage}[b]{0.45\columnwidth}\raggedright\strut
Description\strut
\end{minipage}\tabularnewline
\midrule
\endhead
\begin{minipage}[t]{0.11\columnwidth}\raggedright\strut
\(\epsilon_{leaf}\)\strut
\end{minipage} & \begin{minipage}[t]{0.10\columnwidth}\raggedright\strut
MPa\strut
\end{minipage} & \begin{minipage}[t]{0.12\columnwidth}\raggedright\strut
\texttt{LeafEPS}\strut
\end{minipage} & \begin{minipage}[t]{0.45\columnwidth}\raggedright\strut
Modulus of elasticity of leaves\strut
\end{minipage}\tabularnewline
\begin{minipage}[t]{0.11\columnwidth}\raggedright\strut
\(\epsilon_{stem}\)\strut
\end{minipage} & \begin{minipage}[t]{0.10\columnwidth}\raggedright\strut
MPa\strut
\end{minipage} & \begin{minipage}[t]{0.12\columnwidth}\raggedright\strut
\texttt{StemEPS}\strut
\end{minipage} & \begin{minipage}[t]{0.45\columnwidth}\raggedright\strut
Modulus of elasticity of symplastic xylem tissue\strut
\end{minipage}\tabularnewline
\begin{minipage}[t]{0.11\columnwidth}\raggedright\strut
\(\pi_{0,leaf}\)\strut
\end{minipage} & \begin{minipage}[t]{0.10\columnwidth}\raggedright\strut
MPa\strut
\end{minipage} & \begin{minipage}[t]{0.12\columnwidth}\raggedright\strut
\texttt{LeafPI0}\strut
\end{minipage} & \begin{minipage}[t]{0.45\columnwidth}\raggedright\strut
Osmotic potential at full turgor of leaves\strut
\end{minipage}\tabularnewline
\begin{minipage}[t]{0.11\columnwidth}\raggedright\strut
\(\pi_{0,stem}\)\strut
\end{minipage} & \begin{minipage}[t]{0.10\columnwidth}\raggedright\strut
MPa\strut
\end{minipage} & \begin{minipage}[t]{0.12\columnwidth}\raggedright\strut
\texttt{StemPI0}\strut
\end{minipage} & \begin{minipage}[t]{0.45\columnwidth}\raggedright\strut
Osmotic potential at full turgor of symplastic xylem tissue\strut
\end{minipage}\tabularnewline
\begin{minipage}[t]{0.11\columnwidth}\raggedright\strut
\(RWC\)\strut
\end{minipage} & \begin{minipage}[t]{0.10\columnwidth}\raggedright\strut
{[}0-1{]}\strut
\end{minipage} & \begin{minipage}[t]{0.12\columnwidth}\raggedright\strut
\strut
\end{minipage} & \begin{minipage}[t]{0.45\columnwidth}\raggedright\strut
Relative water content\strut
\end{minipage}\tabularnewline
\begin{minipage}[t]{0.11\columnwidth}\raggedright\strut
\(RWC_{sym}\)\strut
\end{minipage} & \begin{minipage}[t]{0.10\columnwidth}\raggedright\strut
{[}0-1{]}\strut
\end{minipage} & \begin{minipage}[t]{0.12\columnwidth}\raggedright\strut
\strut
\end{minipage} & \begin{minipage}[t]{0.45\columnwidth}\raggedright\strut
Relative water content in the symplasm fraction of a tissue\strut
\end{minipage}\tabularnewline
\begin{minipage}[t]{0.11\columnwidth}\raggedright\strut
\(RWC_{apo}\)\strut
\end{minipage} & \begin{minipage}[t]{0.10\columnwidth}\raggedright\strut
{[}0-1{]}\strut
\end{minipage} & \begin{minipage}[t]{0.12\columnwidth}\raggedright\strut
\strut
\end{minipage} & \begin{minipage}[t]{0.45\columnwidth}\raggedright\strut
Relative water content in the apoplasm fraction of a tissue\strut
\end{minipage}\tabularnewline
\begin{minipage}[t]{0.11\columnwidth}\raggedright\strut
\(V_{segment}\)\strut
\end{minipage} & \begin{minipage}[t]{0.10\columnwidth}\raggedright\strut
\(m^3 \cdot m^{-2}\)\strut
\end{minipage} & \begin{minipage}[t]{0.12\columnwidth}\raggedright\strut
\strut
\end{minipage} & \begin{minipage}[t]{0.45\columnwidth}\raggedright\strut
Water capacity of a segment (leaf or stem)\strut
\end{minipage}\tabularnewline
\begin{minipage}[t]{0.11\columnwidth}\raggedright\strut
\(V_{leaf}\)\strut
\end{minipage} & \begin{minipage}[t]{0.10\columnwidth}\raggedright\strut
\(m^3 \cdot m^{-2}\)\strut
\end{minipage} & \begin{minipage}[t]{0.12\columnwidth}\raggedright\strut
\strut
\end{minipage} & \begin{minipage}[t]{0.45\columnwidth}\raggedright\strut
Leaf water capacity per leaf area unit\strut
\end{minipage}\tabularnewline
\begin{minipage}[t]{0.11\columnwidth}\raggedright\strut
\(V_{sapwood}\)\strut
\end{minipage} & \begin{minipage}[t]{0.10\columnwidth}\raggedright\strut
\(m^3 \cdot m^{-2}\)\strut
\end{minipage} & \begin{minipage}[t]{0.12\columnwidth}\raggedright\strut
\strut
\end{minipage} & \begin{minipage}[t]{0.45\columnwidth}\raggedright\strut
Sapwood water capacity per leaf area unit\strut
\end{minipage}\tabularnewline
\bottomrule
\end{longtable}

\section{Stomatal regulation and
photosynthesis}\label{stomatal-regulation-and-photosynthesis-1}

\begin{longtable}[]{@{}llll@{}}
\toprule
\begin{minipage}[b]{0.11\columnwidth}\raggedright\strut
Symbol\strut
\end{minipage} & \begin{minipage}[b]{0.10\columnwidth}\raggedright\strut
Units\strut
\end{minipage} & \begin{minipage}[b]{0.12\columnwidth}\raggedright\strut
R param\strut
\end{minipage} & \begin{minipage}[b]{0.45\columnwidth}\raggedright\strut
Description\strut
\end{minipage}\tabularnewline
\midrule
\endhead
\begin{minipage}[t]{0.11\columnwidth}\raggedright\strut
\(g_{swmin}\)\strut
\end{minipage} & \begin{minipage}[t]{0.10\columnwidth}\raggedright\strut
\(mol H_2O \cdot s^{-1} \cdot m^{-2}\)\strut
\end{minipage} & \begin{minipage}[t]{0.12\columnwidth}\raggedright\strut
\texttt{Gwmin}\strut
\end{minipage} & \begin{minipage}[t]{0.45\columnwidth}\raggedright\strut
Minimum stomatal conductance to water vapour\strut
\end{minipage}\tabularnewline
\begin{minipage}[t]{0.11\columnwidth}\raggedright\strut
\(g_{swmax}\)\strut
\end{minipage} & \begin{minipage}[t]{0.10\columnwidth}\raggedright\strut
\(mol H_2O \cdot s^{-1} \cdot m^{-2}\)\strut
\end{minipage} & \begin{minipage}[t]{0.12\columnwidth}\raggedright\strut
\texttt{Gwmax}\strut
\end{minipage} & \begin{minipage}[t]{0.45\columnwidth}\raggedright\strut
Maximum stomatal conductance to water vapour\strut
\end{minipage}\tabularnewline
\begin{minipage}[t]{0.11\columnwidth}\raggedright\strut
\(J_{max, 298}\)\strut
\end{minipage} & \begin{minipage}[t]{0.10\columnwidth}\raggedright\strut
\(\mu mol electrons \cdot m^{-2} \cdot s^{-1}\)\strut
\end{minipage} & \begin{minipage}[t]{0.12\columnwidth}\raggedright\strut
\texttt{Jmax298}\strut
\end{minipage} & \begin{minipage}[t]{0.45\columnwidth}\raggedright\strut
Maximum rate of electron transport at 298K\strut
\end{minipage}\tabularnewline
\begin{minipage}[t]{0.11\columnwidth}\raggedright\strut
\(V_{max, 298}\)\strut
\end{minipage} & \begin{minipage}[t]{0.10\columnwidth}\raggedright\strut
\(\mu mol CO_2 \cdot s^{-1} \cdot m^{-2}\)\strut
\end{minipage} & \begin{minipage}[t]{0.12\columnwidth}\raggedright\strut
\texttt{Vmax298}\strut
\end{minipage} & \begin{minipage}[t]{0.45\columnwidth}\raggedright\strut
Rubisco's maximum carboxylation rate at 298K\strut
\end{minipage}\tabularnewline
\bottomrule
\end{longtable}

\bibliography{medfatebook.bib}


\end{document}
